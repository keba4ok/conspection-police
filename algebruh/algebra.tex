\documentclass[a4paper,100pt]{article}

\usepackage[utf8]{inputenc}
\usepackage[unicode, pdftex]{hyperref}
\usepackage{cmap}
\usepackage{mathtext}
\usepackage{multicol}
\setlength{\columnsep}{1cm}
\usepackage[T2A]{fontenc}
\usepackage[english,russian]{babel}
\usepackage{amsmath,amsfonts,amssymb,amsthm,mathtools}
\usepackage{icomma}
\usepackage{euscript}
\usepackage{mathrsfs}
\usepackage{geometry}
\usepackage[usenames]{color}
\hypersetup{
     colorlinks=true,
     linkcolor=red,
     filecolor=red,
     citecolor=black,      
     urlcolor=cyan,
     }
\usepackage{fancyhdr}
\pagestyle{fancy} 
\fancyhead{} 
\fancyhead[LE,RO]{\thepage} 
\fancyhead[CO]{\hyperlink{t2}{к списку объектов}}
\fancyhead[LO]{\hyperlink{t1}{к содержанию}} 
\fancyhead[CE]{текст-центр-четные} 
\fancyfoot{}
\newtheoremstyle{indented}{0 pt}{0 pt}{\itshape}{}{\bfseries}{. }{0 em}{ }

%\geometry{verbose,a4paper,tmargin=2cm,bmargin=2cm,lmargin=2.5cm,rmargin=1.5cm}

\title{Алгебра}
\author{Кабашный Иван\\ (по материалам конспекта старшекурсников, \\ написанном на основе лекций В. А. Петрова)}
\date{22 января 2020 г.}

\theoremstyle{indented}
\newtheorem{theorem}{Теорема}
\newtheorem{lemma}{Лемма}

\theoremstyle{definition} 
\newtheorem{defn}{Определение}
\newtheorem{exl}{Пример(ы)}

\theoremstyle{remark} 
\newtheorem{remark}{Примечание}
\newtheorem{cons}{Следствие}

\DeclareMathOperator{\Ker}{Ker}
\DeclareMathOperator{\Frac}{Frac}
\DeclareMathOperator{\Imf}{Im}
\DeclareMathOperator{\cont}{cont}
\DeclareMathOperator{\id}{id}
\DeclareMathOperator{\ev}{ev}
\DeclareMathOperator{\lcm}{lcm}
\DeclareMathOperator{\chard}{char}
\DeclareMathOperator{\CC}{\mathbb{C}}
\DeclareMathOperator{\RR}{\mathbb{R}}
\DeclareMathOperator{\NN}{\mathbb{N}}
\DeclareMathOperator{\codim}{codim}
\DeclareMathOperator{\rank}{rank}

\begin{document}

\newcommand{\resetexlcounters}{%
  \setcounter{exl}{0}%
} 

\newcommand{\resetremarkcounters}{%
  \setcounter{remark}{0}%
} 

\newcommand{\reseconscounters}{%
  \setcounter{cons}{0}%
} 

\newcommand{\resetall}{%
    \resetexlcounters
    \resetremarkcounters
    \reseconscounters%
}

\maketitle 

\newpage

\hypertarget{t1}{Честно говоря, ненависть} к этой вашей топологии просто невообразимая.
\tableofcontents

\newpage



\section{Билеты}



\subsection{Определение кольца. Простейшие следствия из аксиом. Примеры. Области целостности}

\medskip

\begin{defn}
    \hypertarget{n1}{\textcolor{red}{\textit{Кольцом}}} называется множество $R$ вместе с бинарными операциями $+$ и $\cdot$ (которые называются сложением и умножением соответственно), удовлетворяющим аксиомам:\

    \begin{itemize}
        \item операция сложения ассоциативна;
        \item по отношению к сложению существует нейтральный элемент;
        \item у каждого элемента есть обратный по сложению
        \item операция сложения коммутативна;
        \item умножение ассоциативно;
        \item умножение дистрибутивно по сложеиню.
    \end{itemize}
\end{defn}

Также можно добавить, что если на множестве выполныны три первые аксиомы, то оно будет называться \textit{группой}, а если выполнены первые четыре, то это уже \textit{абелева группа}. Нейтральный по сложению элемент кольца называют \textit{нулём}.

\begin{exl}
    Кольцо называется:\

    \begin{itemize}
        \item \textit{коммутативным}, если оно коммутативно по умножению;
        \item \textit{кольцом с единицей}, если оно содержит нейтральный элемент по умножению (единица);
        \item \textit{телом}, если в нём есть 1, и для любых $a\neq 0 \rightarrow a \cdot a^{-1}=a^{-1}\cdot a=1$;
        \item \textit{полем}, если это коммутативное тело;
        \item \textit{полукольцом}, если нет требования противоположного элемента по сложению.
    \end{itemize}
\end{exl}

\begin{cons}
    Некоторые следствия из аксиом:\

    \begin{itemize}
        \item $0\cdot a = 0$
        \begin{proof}
            \[
               0\cdot a = (0+0)\cdot a = 0\cdot a + 0\cdot a
            \]
            Прибавим к обеим частям $-0\cdot a$ и получим требуемое.
        \end{proof}
        \item Нейтральный элемент по сложению единственный
        \begin{proof}
            Рассмотрим их сумму справа и слева.
        \end{proof}
        \item $a\cdot 0 = 0$
        \begin{proof}
            \[
                a\cdot 1 = a \Longrightarrow (0+1)a = a \Longrightarrow 0\cdot a+1\cdot a = a \Longrightarrow 0\cdot a = 0
            \]
        \end{proof}
    \end{itemize}
\end{cons}

\begin{defn}
    Коммутативное кольцо $R$ с единицей, обладающее свойством
    \[
        xy=0 \Longrightarrow x=0 \vee y=0 \text{ }(\forall x, y\in R)
    \]
    называется \hypertarget{n2}{\textcolor{red}{\textit{областью целостности}}} или просто \textit{областью}.
\end{defn}

\begin{defn}
    Число $d\neq 0$ называется \hypertarget{n3}{\textcolor{red}{\textit{делителем нуля}}}, если существует такое $d'\neq 0$, что $dd'=0$.
\end{defn}

Нетрудно понять, что область целостности - в точности коммутативное кольцо с единицей без делителей нуля.

\resetall

\subsection{Евклидовы кольца. Евклидовость $\mathbb{Z}$. Неприводимые и простые элементы.}

\medskip

Для начала, некоторые связанные понятия, не упомянутые в билетах.

\begin{defn}
    Говорят, что $d$ \textit{делит} $p$ и пишут $d|p$, если $p=dq$ для некоторго $q\in R$.
\end{defn}

\begin{defn}
    Элемент $\varepsilon$ называется \textit{обратимым}, если он делит единицу, то есть существует такое $\varepsilon^{-1} \in R$, что $\varepsilon^{-1} \cdot \varepsilon = 1$. 
\end{defn}

\begin{defn}
    Будем говорит, что элементы $a$ и $b$ \hypertarget{n4}{\textcolor{red}{\textit{ассоциированы}}} и писать $a\sim b$, если выполнено одно из двух эквивалентных условий:
    \begin{itemize}
        \item существует обратимый элемент $\varepsilon$, для которого $a=\varepsilon b$;
        \item $a|b$ и $b|a$.
    \end{itemize}
\end{defn}

Покажем, что эти условия действительно эквивалентны.

\begin{proof}
    Докажем в обе стороны:\

    \textcolor{red}{$\Rightarrow$} Если $a=\varepsilon b$, то $\varepsilon^{-1}a=b$. Это и есть второе условие.\

    \textcolor{red}{$\Leftarrow$} Пусть $a=bc$ и $b=ac'$ для каких-то $c, c'$. Тогда $a = (ac')c=a(cc')\leftrightarrow a(1-cc')=0$. Тогда либо $a=0$, либо $cc'=1$, потому что делителей нуля в нашем кольце нет. В любом случае, $a$ и $b$ отличаются на обратимый: либо они оба равны нулю, либо $c$ - обратимый.
\end{proof}

А теперь, что касается самого билета.

\begin{defn}
    Область целостности  $R$ называется \hypertarget{n5}{\textcolor{red}{\textit{евклидовым кольцом}}}, если существует евклидова норма $N: R\rightarrow \mathbb{N}_0$ такая, что $N(0)=0$ и для любых элементов $a, b \in R$, где $b \neq 0$, существует меньший чем $b$ по норме элемент $r\in R$ такой, что выполнено равенство $a=bq+r$.
\end{defn}

\begin{exl}
    Кольцо целых чисел $\mathbb{Z}$ евклидово.
\end{exl}

\begin{proof}
    Пусть у нас имеются целое число $a$ и ненулевое целое $b$. Тогда существуют такие целые числа $q$ и $r$, что модуль $r$ меньше модуля $b$, а также $a=bq+r$. Отметим на оси все ератные $b$. Тогда если число $a$ попало на отрезок $[kb, (k+1)b]$, $k$ будет частным, а $a-kb$ - остатком. Дальнейшую формализация можно провести индукцией.
\end{proof}

Опять несколько небольших новых определений перед тем как перейти к последнему пункту билета (их можно упустить).

\begin{defn}
    Пусть $R$ - область целостности; $a,b\in R$. Элемент $d\in R$ называется \hypertarget{n6}{\textcolor{red}{\textit{наибольшим общим делителем}}} $a$ и $b$, если 
    \begin{itemize}
        \item $d|a$ и $d|b$;
        \item для любого $d'\in R$, который также делит $a$ и $b$, выполнено также, что он делит $d$.
    \end{itemize}
\end{defn}

\begin{theorem}
    (О линейном представлении НОД в евклидовых кольцах). Пусть $R$ - евклидово кольцо, $a, b\in R$.Тогда существуют $d:=\gcd (a,b)$ и такие $x,y\in R$, что $d=ax+by$.
\end{theorem}\

Теперь про простые и неприводимые.

\begin{defn}
    Пусть $R$ - область. Необратимый элемент $p\in R$ - \hypertarget{n7}{\textcolor{red}{\textit{неприводимый}}}, если 
    \[
        \forall d\in R:d\vert p \Longrightarrow d\sim 1 \vee d\sim p
    \]
\end{defn}

\begin{defn}
    Пусть $R$ - область. Ненулевой необратимый элемент $p\in R\backslash 0$ называется \hypertarget{n8}{\textcolor{red}{\textit{простым}}}, если $\forall a,b\in R: p\vert ab \Longrightarrow p\vert a \vee p\vert b$.
\end{defn}

\begin{lemma}
    (Простые $\subset$ неприводимые). Если $p$ - простой элемент произвольного коммутативног кольцв с единицей, то $p$ - неприводим.
\end{lemma}

\begin{proof}
    Пусть $d$ - какой-то делитель $p$, что эквивалентно равенству $p=da$ для какого-то $a$. Проверим, что либо $d\sim 1$, либо $d\sim p$. Раз $p$ - простой, то либо он делит $d$, либо он делит $a$. Если первое, что сразу $d\sim p$.Если второе, перепишем в виде $da=p\vert a$. Это то же самое, что $bda=a$ для некоторого $b$. Здесь либо $a=0$, то тогда $p=o$, что невозможно по определению простого, либо мы можем сократить на $a$ и получим $bd=1$, тогда $d$ ассоциирован с 1.
\end{proof}

Теперь немного добавки про простые и неприводимые, на всякий случай.\\

\begin{lemma}
    (Неприводимые $\subset$ простые в ОГИ). Пусть $p$ - неприводимый в области главных идеалов. Тогда $p$ - простой.
\end{lemma}

\begin{proof}
    Пусть $p\vert ab$, хотим показать, что $p\vert a \vee p\vert b$. Воспользуемся тем, что мы в области главных идеалов: $(p,a)=(d)$, где $d:=\gcd (a, p)$, а тогда $px+ay=d$ для каких-то $x, y$. $d\vert p$, воспользуемся неприводимостью $p$: либо $d\sim p$, либо $d\sim 1$.\
    
    В первом случае $p\vert d$, тогда $p\vert d\vert a$.\

    Во втором случае можно после домножения на обратимые считать, что $px+ay=1$. Потом домножим на $b:pbx+aby=b$. $p$ явно делит первое слагаемое, ровно как и второе (по предположению). Значит, $p\vert b$.\

    В любом случае, приходим к желаемому.
\end{proof}

\resetall

\subsection{Идеалы, главные идеалы. Евклидово кольцо как кольцо главных идеалов}

\begin{defn}
    Подмножество 
    \[
        (a_1, \dots, a_n):=\{a_1x_1+\dots+a_nx_n\vert x_i\in R \text{ для всех } i\}
    \]
    коммутативного кольца $R$ называется \hypertarget{n9}{\textcolor{red}{\textit{идеалом}}}, порождённым $a_1, \dots, a_n$.
\end{defn}

\begin{defn}
    Подкольцо $I$ кольца $R$ называется \textit{левым идеалом}, если оно замкнуто относительно домнодения слева на элементы кольца: $RI=I$. Соответственно, также различают \textit{правые} и \textit{двусторонние идеалы}. 
\end{defn}

Также идеал можно задать следующими свойствами:
\begin{itemize}
    \item $\forall x, y\in I \Longrightarrow x+y\in I$;
    \item $\forall x\in I, \forall r\in R \Longrightarrow xr\in I$;
    \item $-x\in I$;
    \item $I$ - непустой.
\end{itemize}

\begin{defn}
    Идеал называется \textit{главным}, если он порождён одним элементом.
\end{defn}

\begin{defn}
    \hypertarget{n10}{\textcolor{red}{\textit{Область главных идеалов}}} - область целостности, в который каждый идеал главный.
\end{defn}

\begin{theorem}
    (Евклидовы кольца $\subset$ ОГИ). Пусть $R$ - евклидово кольцо, $I\unlhd R$ - идеал. Тогда $I$ - главный.
\end{theorem}

\begin{proof}
    Найдём элемент, который порождает идеал $I$.\
        
    Вырожденный случай: если $I=\{0\}$, тогда $I=(0)$.\

    Иначе возьмём $d\in I\backslash 0$ с минимальной нормой (по принципу индукции мы можем это сделать). Хотим показать, что $I=(d)$. Покажем это в обе стороны.\

    \textcolor{red}{$\Rightarrow$} Легко видеть, что $(d) \subset I$.\

    \textcolor{red}{$\Leftarrow$} Пусть $a \in I$, тогда поделим $a$ на $b$ с остатком: $a=bd+r$. Предположим, $r\neq 0$, $N(r)<N(d)$. Выразим $r$ линейной комбинацией $a\in I$ и $d \in I$: $r=a-bd\in I$ - противоречие с минимальностью нормы $d$. Значит, $r=0$, а тогда $a=bd\in (d)$.
\end{proof}

\resetall

\subsection{Основная теорема арифметики}

Сначала опять немного информации, которая к билету не относится, но к нему логично подводит.\

\begin{defn}
    Коммутативное кольцо с единицей $R$ удовлетворяет \hypertarget{n11}{\textcolor{red}{\textit{условию обрыва возрастающих цепей главных идеалов}}} или, что то же самое, является \textit{нетёровым кольцом}, если не существует бесконечной строго возрастающей цепочки главных идеалов $(d_1)\subsetneq (d_2)\subsetneq \dots$ Иначе говоря, бесконечной цепочки $\dots \vert d_2\vert d_1$, где все $d_i$ попарно не ассоциированы.
\end{defn}

\begin{theorem}
    (ОГИ $\subset$ нетёровы кольца). Область главных идеалов удовлетворяет условию обрыва возрастающих цепей главных идеалов (далее - УОВЦГИ).
\end{theorem}

\begin{proof}
    Предположим, что нашлась такая бесконечная цепочка $\{d_i\}$. Объединим $I:=\bigcup_{i=0}^\infty (d_i)$.\

    Покажем, что $I$ - идеал. $0\in I$. Пусть $u\in (d_i)$ и $v\in (d_j)$, где $i\leq j$, проверяем остальные условия. $u+v\in(d_j)$, потому что $u\in (d_j)$, с остальными аналогично, не очень сложно.\ 

    Вспомним, что мы находимся в ОГИ, то есть, каждый идеал главный. Пусть $d$ - \textit{генератор} $I$ ($I=(d)$). Любой $(d_i)$ строго содержится в $(d_{i+1})$, а этот содержится в $(d):(d_i)\subsetneq (d_{i+1})\subset (d)$, значит, любой из $\{(d_i)\}$ строго содержится в $(d)$. Но сам генератор $d$ тоже должен принадлежать какому-то из $\{(d_i)\}$, а значит, на каком-то моменте $(d)\subset (d_i)$. Противоречие.
\end{proof}

\begin{defn}
    Кольцо называется \hypertarget{n12}{\textcolor{red}{\textit{факториальным}}}, если одновременно выполнено:
    \begin{itemize}
        \item $R$ - область;
        \item любой неприводимый элемент $R$ - простой;
        \item $R$ - нетёрово.
    \end{itemize}
\end{defn}

\begin{exl}
    Как мы уже знаем, ОГИ $\subset$ факториальные кольца.
\end{exl}

А теперь, к основному.\\

\begin{theorem}
    (\hypertarget{n13}{\textcolor{red}{\textit{Основная теорема арифметики}}}). Пусть $R$ - факториальное кольцо.\ 

    Тогда любой элемент $x\in R$, если он не нуль и не обратимый, представляется в виде $r=p_1\dots p_n$, где $n\geq 1$, а $\{p_i\}$ - простые.\ 

    При этом, если $r=q_1\dots q_m$ - другое такое разложение, то $m=n$ и существует перестановка индексов $\pi: n\rightarrow n$, такая, что $p_i\sim q_{\pi_i}$ для всех $i$.
\end{theorem}

\begin{proof}
    Докажем существование. Зафиксируем $x$. Если он неприводимый, то он и простой по определению факториального кольца, поэтому сам будет своим подходящим разложением. Пусть $x=yz$, где $y, z \nsim 1$.Если $y$ необратим и приводим, разложим и его: $y=y_1 z_1$, где $y_1, z_1 \nsim 1$. Будем раскладывать так игреки, пока можем, и получим строго возрастающую цепочку идеалов $(y)\subsetneq(y_1)\subsetneq(y_2)\subsetneq\dots$ Вспомним нетёровость нашего кольца: бесконечно возрастать она не может, значит, на каком-то моменте заработаем для $x$ один не приводимый делитель $p: x=pw$ для какого-то $w$. Если $w$ необратим и приводим, разложим и его: $w=p_1 w_1$. Продолжим и получим ещё одну возрастающую цепочку идеалов: $(x)\subsetneq (w)\subsetneq (w_1)\subsetneq \dots$ К тому времени, когда она оборвётся, у нас будет разложение $x$ в конечное произведение неприводимых: $x=p_1\dots p_n$. Существование доказано.\ 

    Теперь перейдём к доказательству единственности. Разложим двумя способами: $r=p_1\dots p_n=q_1\dots q_m$. По индукции пожно вывести из определения простого, что\\

    \begin{lemma}
        Если $p$ - простой и $p\vert a_1\dots a_n$, то $p\vert a_i$ для какого-то $i$.
    \end{lemma}\

    Воспользуемся этим фактом:например, мы теперь знаем: что $q_m\vert p_i$ для какого-то $i$. Но $p_i$ неприводим, поэтому любой его делитель либо обратим, либо ассоциирован с ним. $q_m$ не боратим, так как он простой; значит, $q_m \sim p_i$. Переставим $p_i$ и $p_n$ и считаем, что $q_m$ теперь $\sim p_n$. Осталось вывести следующий факт:\\

    \begin{lemma}
        Пусть $a\sim b$, $ac\sim bd$, а $b\neq 0$. Тогда $c\sim d$.
    \end{lemma}

    \begin{proof}
        $a=\varepsilon b$и $ac=\varepsilon bc =\nu bd$ для каких-то обратимых $\varepsilon$ и $\nu$. Последнее равенство можем сократить на $b\neq 0$, потому что мы в области.
    \end{proof}

    Теперь $p_1\dots p_{n-1}\sim q_1\dots q_{m-1}$. Можем теперь сказать, что равенство $p_1\dots p_{n-1} = q_1\dots q_{m-1}$ верно по предположению индукции по $n$. Так же по индукции $n=m$, потому что получим противоречие, если какая-то из серий сомножителей $\{p_i\}$, $\{q_i\}$ закончится раньше.
\end{proof}

\begin{exl}
    Обыкновенное кольцо $\mathbb{Z}$ $\in$ евклидовы кольца $\subset$ ОГИ $\subset$ факториальные кольца.
\end{exl}

\resetall

\subsection{Кольцо вычетов $\mathbb{Z}/_{n\mathbb{Z}}$. Китайская теорема об остатках}

\begin{exl}
    Множество $\mathbb{Z}/_{n\mathbb{Z}} = \{[0], \dots, [n-1]\}$ остатков при делении на $n\in \mathbb{N}$ - коммутативное кольцо с единицей. \hypertarget{n14}{\textcolor{red}{\textit{Кольцо вычетов}}} (остатков) по модулю.
\end{exl}

\begin{defn}
    $m$, $n$ \textit{взаимно просты}, если $(m,n)=(1)=R$.
\end{defn}

\begin{lemma}
    Пусть $R$ - факториальное кольцо, $m, n\in R$ - взаимно простые элементы. Пусть, к тому же, $m$ и $n$ - делители $r:m,n\vert r$. Тогда их произведение тоже делит $r:mn\vert r$.
\end{lemma}

\begin{proof}
    Можно вывести из ОТА.
\end{proof}

\begin{theorem}
    (\hypertarget{n15}{\textcolor{red}{\textit{Китайская теорема об остатках}}}). Если $(m,n)=(1)$, то $\mathbb{Z}/(m)\times \mathbb{Z}/(n)\cong \mathbb{Z}/(mn)$.
\end{theorem}

\begin{proof}
    Пусть $x$ - классы, соответствующие числу $x$ в $\mathbb{Z}/(m)$ и $\mathbb{Z}/(n)$, соответственно. Рассмотрим гомеоморфизм $f=x\mapsto ([x]_m, [x]_n)$.\ 

    Его ядро - числа, которые делятся и на $m$, и на $n$, а поскольку они взаимно просты, то и на $mn$. Значит, $\text{Ker} f= (mn)$.\ 

    Проверим $f$ на сюръективность. Для этого просто хитро покажем, что $\text{Im} f \cong \mathbb{Z}/(mn)$. Тогда $mn = \vert \mathbb{Z}/(mn)\vert = \vert \text{Im} f\vert$. При этом $\text{Im} f \subset \mathbb{Z}/(m)\times \mathbb{Z}/(n)$ по определению (подкольцо) и $\vert \mathbb{Z}/(m)\times \mathbb{Z}/(n) \vert = mn$ простым подсчётом, откуда следует, что $\text{Im}=\mathbb{Z}/(m)\times \mathbb{Z}/(n)$.
\end{proof}

\begin{cons}
    $\mathbb{Z}/(n)$ - область целостности $\Longleftrightarrow$ $n$ - простое.
\end{cons}

\resetall

\subsection{Определение поля. $\mathbb{Z}/_{p\mathbb{Z}}$ как поле. Поле частных целостного кольца}

Напомним ещё раз определение поля.\

\begin{defn}
    \hypertarget{n16}{\textcolor{red}{\textit{Поле}}} - коммутативное кольцо с единицей, в котором также существует обратный элемент по умножению для ненулевых элементов.
\end{defn}

\begin{exl}
    $\mathbb{Z}/_{p\mathbb{Z}}$ - поле.
\end{exl}

\begin{proof}
    Мы уже много чего знаем про эту структуру (см. конец предыдущего билета). Для доказательства вышеприведённого факта нужно показать, что у каждого элемента есть обратный по умножению (кроме, конечно, нуля). Рассмотрим ненулевой элемент $a$, и умножим его на все остатки по модулю $p$, получим $\{0a, 1a, \dots, (p-1)a\}$. Заметим, что все полученные остатки различны. Предположим противное: $ka\equiv ma \Leftrightarrow (k-m)a\equiv 0$, но так как мы находимся в области, то либо $a=0$ (сразу нет), либо $k-m=0$, но так как они оба меньше $p$, то такого тоже, очевидно, не бывает. Тогда мы получили, что все остатки, полученные таким образом, различны. Но так как их ровно $p$, то найдётся и равный 1, элемент на который мы умножаем в том случае и будет обратным к $a$.
\end{proof}

В общем и целом, мы сейчас будем получать что-то вроде $\mathbb{Q}$, но над любым кольцом $R$. Введём отношение $\sim$ на множестве пар $R\times (R\backslash 0)$. Пусть $(a, b)\sim(a', b') \Leftrightarrow ab'=a'b$. Проверим, что мы получили отношение эквивалентности:

\begin{proof}
    Нужно показать рефлексивность, симметричность и транзитивность. Первые два утверждения очевидны, покажем последнее. Пусть $(a, b)\sim(a', b')$ и $(a', b')\sim(a'', b'')$, мы хотим показать, что $(a, b)\sim(a'', b'')$, то есть, $ab''=a''b$. Воспользуемся тем, что мы находимся в области целостности - домножим левую часть последнего равенства на ненулевой $b'$ и преобразуем, используя гипотезы:
    \[
        (ab')b''=b(a'b'')=bb'a''.
    \]
    Теперь сократим на $b'$.
\end{proof}

\begin{defn}
    Фактор $R/{\sim}$ называется \hypertarget{n18}{\textcolor{red}{\textit{полем частных}}} области целостности $R$ и обозначается за $\text{Frac}R$. Элементы будем обозначать дробями.
\end{defn}

Сложение и умножение определяется как в обычной жизни. Осталось проверить, что это действительно поле.

\begin{proof} Нужно выполнить совсем немного проверок: \
    \begin{itemize}
        \item $0\over 1$ - нуль;
        \item $1\over 1$ - единица;
        \item $\frac{-a}{b}$ - обратный к $\frac{a}{b}$ по сложению;
        \item $\frac{b}{a}$ - обратный к $\frac{a}{b}$ по умножению для ненулевых.
    \end{itemize}
\end{proof}

\resetall

\subsection{Определение гомоморфизма и изоморфизма колец. Фактор-кольцо}

\begin{defn}
    Пусть $R$ и $S$ - кольца. Функция $f:R\rightarrow S$ называется \hypertarget{n19}{\textcolor{red}{\textit{гомоморфизмом колец}}}, если для произвольных элементов выполняется
    \begin{itemize}
        \item $f(r_1+r_2)=f(r_1)+f(r_2)$;
        \item $f(r_1r_2)=f(r_1)f(r_2)$.
    \end{itemize}
\end{defn}

\begin{lemma}
    Если $f$ - гомоморфизм, то $f(0)=0$ и $f(-r)=-r$.
\end{lemma}

\begin{proof}
    В обоих пунктах - подсчёт двумя способами:\ 

    \begin{itemize}
        \item $f(0)+f(0)=f(0+0)=f(0)$;
        \item $f(r)+f(-r)=f(r+(-r))=f(0)=0$
    \end{itemize}
\end{proof}

Кстати говоря, не любой гомоморфизм сохраняет единицу.

\begin{exl}
    Пусть $f:r\rightarrow R\times S$ и $f=r\mapsto (r,0)$. Тогда $f(1)=(1,0)\neq 1$.
\end{exl}

\begin{defn}
    Если для гомоморфизма $f$ выполнено $f(1)=1$, то говорят, что он \textit{сохраняет единицу}.
\end{defn}

С гомоморфизмом связаны два важных понятия, которые мы рассмотрим далее.

\begin{defn}
    \hypertarget{n20}{\textcolor{red}{\textit{Ядро}}} $\text{Ker}f$ гомоморфизма $f:R\rightarrow S$ - полный прообраз нуля, $f^{-1}(0)$.
\end{defn}

\begin{lemma}
    Гомоморфизм $f$ инъективен тогда и только тогда, когда его ядро тривиально: $\text{Ker}f=\{0\}$.
\end{lemma}

\begin{proof}
    Потому что $f(x_1)=f(x_2)\Longleftrightarrow f(x_1-x_2)=0$
\end{proof}

\begin{lemma}
    $\text{Ker}f$ - двусторонний идеал в $R$.
\end{lemma}

\begin{proof}
    Пусть $k\in \text{Ker}f$, тогда для любого $r\in R$ $f(rk)=f(r)f(k)=f(r)\cdot 0 = 0 = 0\cdot f(r)=f(kr)$. Ещё, например, $f(k_1+k_2)=f(k_1)+f(k_2)=0$. Остальные пункты из определения так же очевидны.
\end{proof}

\begin{defn}
    \hypertarget{n21}{\textcolor{red}{\textit{Образ}}} области определения гомоморфизма $f$ обозначается как $\text{Im}f$.
\end{defn}

\begin{lemma}
    Если $f:R\rightarrow S$ - гомоморфизм, то $f(R)$ - кольцо.
\end{lemma}

\begin{proof}
    $f(a)+f(b)=f(a+b)$, $f(a)f(b)=f(ab)$ - как раз.
\end{proof}

\begin{defn}
    \hypertarget{n22}{\textcolor{red}{\textit{Изоморфизм}}} - биективный гомоморфизм. Пишут $R\cong S$, если между ними существует изоморфизм.
\end{defn}

А теперь про фактор-кольца.

\begin{defn}
    Пусть $R$ - кольцо (возможно, некоммутативное и без единицы), а $I$ - двусторонний идеал. Говорят, что $a$ \hypertarget{n23}{\textcolor{red}{\textit{сравнимо}}} с $b$ по модулю $I$ и пишут $a\equiv b \mod I$, если $a-b\in I$.
\end{defn}

\begin{lemma}
    Сравнимость по модулю - отношение эквивалентности.
\end{lemma}\

Так как мы получили отношение эквивалентности, по нему можно факторизовать. Тогда аналогами классов эквивалентности становятся множества вида $[a]:=\{b\in \mathbb{R}\vert b\equiv a \mod I\}$. Обозначим кмножество всех этих классов за $R/I$. Осталось ввести структуру кольца на этом множестве.\ 

Определим действия: $[a]+[b]=[a+b]$ и $[a][b]=[ab]$. Нетрудно понять, что действия над классами не зависят от выбора \textit{представителя}. Сложение вообще очевидно, а при умножении нужно ''прибавить и вычесть'',  чтобы собрать.\\

\begin{theorem}
    Пусть $R$ - произвольное кольцо, возможно, некоммутативное и без единицы; $I\trianglelefteq R$ - двустронний идеал.\ 

    Обозначим за $R/I$ фактор $R$ по отношению эквивалентности $\{a\equiv b\vert a-b\in I\}$, за $[a]$ - класс эквивалентности элемента $a\in R$.\ 

    Тогда:
    \begin{itemize}
        \item операции $[a]+[b]=[a+b]$ и $[a][b]=[ab]$ определены корректно и задают на $R/I$ структуру кольца;
        \item если $R$ коммутативно, то $R/I$ - тоже;
        \item если $R$ - кольцо с единицей, то $[1]$ - единица $R/I$.
    \end{itemize}
\end{theorem}

\begin{proof}
    В первом пункте мы уже проверили все неочевидные пункты в определении кольца, остальное - тривиально.
\end{proof}

\begin{defn}
    $R/I$ - \hypertarget{n24}{\textcolor{red}{\textit{фактор-кольцо}}} $R$ по $I$.
\end{defn}

\resetall

\subsection{Теорема о гомоморфизме}

\begin{theorem}
    (\hypertarget{n25}{\textcolor{red}{\textit{Теорема о гомоморфизме}}}). Пусть $f:R\rightarrow S$ - гомоморфизм колец. Тогда $f(R)\cong R/\text{Ker}f$.
\end{theorem}

\begin{proof}
    Что мы будем делать по сути: вместо того, чтобы сразу отправлять элемент из $R$ в $S$ посредством $f$, сначала спроецируем его в $R/\text{Ker} f$ и оттуда уже отобразим в $f(R)$. Проверяем следующее для формальности:\ 

    \begin{itemize}
        \item \textit{Корректность определения}. Пусть $[r]=[r']$. Тогда $r'-r\in \text{Ker} f$, что равносильно $f(r'-r)=0$, а тогда $f(r)=f(r')$.
        \item \textit{Сюръективность}. По определению $f(R)$ любой элемент оттуда - это $f(r)$ для какого-то элемента $r\in R$, а $f(r)$ - образ $[r]$ при нашем отображении.
        \item \textit{Инъективность}. Пусть $f(r_1)=f(r_2)$, тогда $f(r_1-r_2)=0$. Значит, $r_1-r_2\in \text{Ker} f$, что эквивалентно $r_1\equiv r_2 \mod \text{Ker} f$.
        \item \textit{Сохраняет операции}. $\varphi([a])+\varphi([b])=\varphi([a]+[b])=\varphi([a+b])=f(a+b)=f(a)+f(b)=\varphi([a])+\varphi([b])$. С умножением - агалогично.
    \end{itemize}
\end{proof}

Так как билет и так короткий - припишем сюда ещё одну теорему, которой почему-то нет в билетах.\\

\begin{theorem}
    (\hypertarget{n26}{\textcolor{red}{\textit{Универсальное свойство фактор-кольца}}}). Пусть $R$ - кольцо, $I\trianglelefteq R$ - двусторонний идеал, $\pi: R\rightarrow R/I$ - канонический гомоморфизм, $\varphi: R\rightarrow S$ - гомоморфизм колец, ядро которого содержит $I$: $\varphi(I)=\{0\}$. Тогда:\

    \begin{itemize}
        \item существует единственный гомоморфизм $\bar{\varphi}:R/I\rightarrow S$ такой, что $\varphi=\bar{\varphi}\circ \pi$;
        \item $\bar{\varphi}$ задаётся формулой $\bar{\varphi}=[x]\mapsto \varphi(x)$.
    \end{itemize}
\end{theorem}

\begin{proof}
    Раз уж теоремы в списке нет, то доказывать её не будем. Если вкратце, то сначала несложно проверяется единственность, затем - корректность, и, наконец, рутинная проверка на гомоморфизм.
\end{proof}

\resetall

\subsection{Кольцо многочленов. Целостность и евклидовость кольца многочленов над полем}

\begin{defn}
    \hypertarget{n27}{\textcolor{red}{\textit{Многочлен}}} - комбинация вида $\sum_{i=0}^\infty a_ix^i$, где почти все (кроме конечного числа) $\{a_i\}$ равны нулю. В кольце может и не быть единицы, но даже тогда мы определяем $a_0x^0:=a_0$ для удобства нотации.
\end{defn}

\begin{defn}
    $a$ \textit{коммутриует} с $b$, если $ab=ba$.
\end{defn}

\begin{defn}
    \hypertarget{n28}{\textcolor{red}{\textit{Кольцо многочленов}}} $R[x]$ - кольцо $R$ вместе с некоторыми $x\notin R$, для которых выполняются следующие свойства:\ 

    \begin{itemize}
        \item $\forall a\in R:ax=xa$;
        \item $\sum a_ix^i+\sum b_ix^i=\sum(a_i+b_i)x^i$;
        \item $-\sum a_ix^i=\sum-a_ix^i$;
        \item нуль есть $\sum 0x^i$;
        \item умножение по формуле свёртки: если 
        \[
            \biggl(\sum_i a_i x^i\biggr)\biggl(\sum_j b_j x^j\biggr)=\sum_k c_k x^k, 
        \]
        то
        \[
            c_k=a_kb_0+a_{k-1}b_1+\dots+a_0b_k=\sum_{i+j=k}a_ib_j.
        \]
    \end{itemize}
\end{defn}

\begin{defn}
    \hypertarget{n29}{\textcolor{red}{\textit{Степень многочлена}}} $\deg \sum a_i x^i$ - наибольшее $i$ такое, что $a_i\neq 0$. Если таких $i$ нет (многочлен нулевой), то его степень определять не будем.
\end{defn}

\begin{cons}
    Из определения степени сразу следует несколько свойств: \

    \begin{itemize}
        \item $\deg(f+g)\leq\deg f+\deg g$;
        \item $\deg(fg)\leq\deg f+\deg g$ для ненулевых $f, g$;
        \item $\deg(fg)=\deg f+\deg g$ для ненулевых $f, g$, если мы находимся в области целостности.
    \end{itemize}
\end{cons}

Последнее получается постольку поскольку старшие коэффициенты просто перемножеются, поэтому можно сформулировать такую лемму:\\

\begin{lemma}
    \hypertarget{t3}{Если} $R$ - область, то и $R[x]$ - область.
\end{lemma}\

А сейчас будем учиться делить многочлены с остатком, тем самым, покажем, что полученное кольцо евклидово (не всегда, конечно).\\

\begin{lemma}
    Пусть $R$ - кольцо, $f=a_nx^n+\dots \in R[x]$, $g=b_mx^m+\dots\in R[x]$, $\forall i: b_m^{n-m+1}\vert a_i$. Тогда существуют многочлены $q,r\in R[x]$ такие, что $f=gq+r$ и $r=0\vee \deg r<\deg g$.
\end{lemma}

\begin{proof}
    Докажем индукцией по $n$. База: если  $n<m$, то положим $q:=0$ и $r:=f$.\ 

    Пусть теперь $n\geq m$. По условию делимости $b_m^{n-m-1}c:=a_n$ для некоторог $c$. Посмотрим на $f_1:=f-tg$, где $t:=cb_m^{n-m}x^{n-m}$ - страший член неполного частного. $\deg tg = n$, потому старший член сократился при делении. Предположение индукции верно для пары $f_1,g$, так как единственный аспект под вопросом - делимость коэффициентов, но он тоже верен, что видно из определения $f_1$. Тогда применим индукцию: $f_1=q_1g+r$. Подставим $f=(t+q_1)g+r$. $r$ найден.
\end{proof}

\begin{cons}
    Пусть $F$ - поле. Тогда $F[x]$ - евклидово кольцо.
\end{cons}

\begin{proof}
    По доказанному выше, $\deg: F[x]\backslash 0\rightarrow \mathbb{N}_0$ - евклидова норма, потому что старший коэффициент ненулевого многочлена всегда обратим.
\end{proof}

\begin{remark}
    А вот для евклидова $R$, $R[x]$ не обязательно будет евклидовым кольцом.
\end{remark}

\resetall

\subsection{Лемма Гаусса}

\begin{defn}
    \hypertarget{n30}{\textcolor{red}{\textit{Содержание}}} $\cont f$ для $f\in R[x]$ - это $\gcd$ всех коэффициентов $f$.
\end{defn}

Видно, что для любого многочлена $f$ существует $f_1:f_1\cont f$ для некоторого $f_1$ такого, что $\cont f\sim 1$.\\

\begin{lemma}
    (\hypertarget{n31}{\textcolor{red}{\textit{Лемма Гаусса}}}). Если $\cont f\sim 1$ и $\cont g\sim 1$, то $\cont fg \sim 1$.
\end{lemma}

\begin{proof}
    Пусть $p\in R$ - простой и $p\vert \cont fg$, ради противоречия. Раз у $fg$ все коэффициенты делятся на $p$, то по модулю $(p)$ он нулевой: $fg=0$ в $R/(p)[x]$. (Конечно, здесь мы имеем в виду его образ при проекции $R[x]\rightarrow R/(p)[x]$, которую естественным образом индуцирует каноническая проекция $\pi: R\rightarrow R/(p):\sum a_i x^i\mapsto \sum \pi (a_i)x^i$). Но поскольку $p$ простой, $R/(p)$ - это область (посмотрим, что такое нуль фактор-кольца и соотнесём с определением простого), а тогда $f=0 \vee g=0$ в $R/(p)[x]$ \hyperlink{t3}{(см)}, что противоречит определению $f$ и $g$.
\end{proof}

\begin{cons}
    $(\cont f)(\cont g)\sim cont fg$.
\end{cons}

\begin{proof}
    Раз $f_1:f_1\cont f$ b $g_1:g_1\cont g$ для некоторых $f_1, g_1 :\cont f_1, \cont g_1\sim 1$, то $\cont f_1g_1\sim 1$, а тогда $\cont fg=\cont (f_1g_1\cont f \cont g)\sim \cont f_1 g_1 \cont f \cont g\sim \cont f\cont g$.
\end{proof}

\resetall

\subsection{Факториальность кольца многочленов}

Щас дикий пиздец будет. Пристегнитесь, мы взлетаем. Начнём с леммы, которая встречается в теореме, но доказывалась раньше.\\

\begin{lemma}
    Если $R$ - нетёрова область. то $R[x]$ тоже нетёрова область.
\end{lemma}

\begin{proof}
    Что область, мы уже знаем (\hyperlink{t3}{см}). Поймём нетёровость.\ 

    Предположим противное: пусть $\dots \vert f_2\vert f_1$ - бесконечная цепочка попарно не ассоциированных многочленов. С ростом индексов степень невозрастает, значит, с какого-то момента она стабилизируется. Тогда отбросим начальный отрезок цепочки (конечно, конечный) и будем считать: что все степени равны $n$.\ 

    Однако теперь посмотрим на $i$-ый коэффициент в каждом многочлене и поймём, что для любых двух последовательных $a_i c=b_i$. Однако опять получается бесконечная цепочка, противоречие.
\end{proof}

Теперь будем плавно переходить к $(\Frac R)[x]$. Пусть $f$ там и лежит. Но тогда заметим, что существует $\tilde{f}\in R[x]$ и $c\in R$ такие, что $f=\frac{\tilde{f}}{c}$. Определим тогда 
\[
    \cont f:=\frac{\cont \tilde{f}}{c}.
\]
При таком определении всё окей с предыдущими леммами, которые нам понядобятся.\\

\begin{lemma}
    Пусть $f, g\in R[x]$. Тогда следующие условия эквивалентны:\
    
    \begin{itemize}
        \item $f\vert g$ внутри $R[x]$;
        \item $f\vert g$ внутри $(\Frac R)[x]$, и $\cont f\vert \cont g$ внутри $R$.
    \end{itemize}
\end{lemma}

\begin{proof}
    В разные стороны поочерёдно:\ 

    \textcolor{red}{$\Rightarrow$} Пусть $g=fh$ для какого-то $h\in R[x]$. Тогда сразу имеем первое условие, а второе вытекает из мультипликативности: $\cont g\sim \cont f\cont h$.\ 

    \textcolor{red}{$\Leftarrow$} Пусть $g:=f\frac{\tilde{h}}{c}$ для некоторых $\tilde{h}\in R[x]$ и $c\in R$. То же самое: $gc=f\tilde{h}$. Применим $\cont$: $c \cont g = (\cont f)(\cont \tilde{h})$. По гипотезе, $\cont g\sim d\cont f$ для некоторого $d\in R$. В итоге $cd\cont f\sim \cont f\cont \tilde{h}$, а мы умеем сокращать в обрастях целостности, тогда $cd\sim \cont \tilde{h}$. Значит, $\cont \tilde{h}$ делитная на $c$, и начальное равенство можно сократить: $g\sim f\hat{h}$, где $\hat{h}$ - какой-то многочлен из $R[x]$. Тогда точно $f\vert g$ внутри $R[x]$.
\end{proof}

А теперь главное блюдо этого билета.\\

\begin{theorem}
    (\hypertarget{n32}{\textcolor{red}{\textit{Теорема Гаусса}}}). Если $R$ факториально, то $R[x]$ факториально.
\end{theorem}

\begin{proof}
    Про область мы уже знаем, про нетёровость тоже (из всех этих прошлых лемм). Тогда осталость только понять, что если $p\in R[x]$ неприводим, то он прост.\\

    \textit{Первый случай}: $\deg p>0$. \\

    \begin{lemma}
        Если $p\in R[x]$ - неприводимый многочлен степени хотя бы 1, то $\cont p\sim 1$.
    \end{lemma}

    \begin{proof}
        Предположим противное, пусть $\exists c\nsim 1: c\vert \cont p$. Запишем тривиальное разложение $p=c\cdot \frac{p}{c}$. $p$ неприводим и, по определению, неообратим, значит, по крайней мере, один из этих сомножителей ассоциирован с $p$. Если оба с ним ассоциированы, то $p\sim p^2$, откуда, поскольку мы в области, $p\sim 1$, что , опять же, невозможно. Если $c\sim p$ и $\frac{p}{c}\sim 1$, то $\deg p = 0$ - это не случай, который мы рассматриваем. Иначе $c\sim 1$ и $\frac{p}{c}\sim p$ - противоречие с определением уже $c$.
    \end{proof}

    \begin{lemma}
        Если $p\in R[x]$ - такой же, как и в предыдущей лемме, то $p$ неприводим в $(\Frac R)[x]$.
    \end{lemma}

    \begin{proof}
        Предположим противное: пусть $p:=\frac{\tilde{g}}{c}\frac{\tilde{h}}{d}$, где $\tilde{g}, \tilde{h}\in R[x]$ и $\frac{\tilde{g}}{c}, \frac{\tilde{h}}{d}\nsim 1$ в $(\Frac R)[x]$. Пепепишем: $cdp=\tilde{g}\tilde{h}$. По первой лемме, $\cont p\sim 1$, значит, беря содержание обеих частей, получаем 
        \[
            cd\sim (\cont \tilde{g})(\cont \tilde{h}).
        \]
        Вернёмся к изначальному $p=\frac{\tilde{g}}{c}\frac{\tilde{h}}{d}$. Здесь $\tilde{g}=(\cont \tilde{g})\hat{g}$ и $\tilde{h}(\cont \tilde{h})\hat{h}$ для некоторых $\hat{g}, \hat{h}\in R[x]$, поэтому
        \[
            p=\frac{(\cont \tilde{g})(\cont \tilde{h})}{cd}\hat{g}\hat{h}\sim \hat{g}\hat{h}.
        \]
        Воспользуемся неприводимостью $p$ в $R$: скажем, $\hat{g}\sim 1$. Тогда, раз мы в поле, 
        \[
            \frac{\tilde{g}}{c}\sim\frac{\cont \tilde{g}}{c}\sim 1, 
        \]
        что и требовалось.
    \end{proof}

    \begin{cons}
        Более того, в последней лемме $(\Frac R)[x]$ - евклидово, как мы знаем, так что $p$ ещё и простой.
    \end{cons}

    Осталось показать, что от также простой в $R[x]$.\\

    \begin{lemma}
        Если $p$ - всё тот же, то он простой в $R[x]$.
    \end{lemma}

    \begin{proof}
        Пусть $p\vert ab$ для каких-то $a,b\in R[x]$. По следствию из второй леммы, $p$ простой в $(\Frac R)[x]$, тогда, без ограничения общности, $p\vert a$ в $(\Frac R)[x]$. По первой лемме, $\cont p\sim 1$, а тогда $\cont p\vert \cont a$. Вывели, что $p\vert a$ в  $R[x]$ по лемме, которая была перед теоремой.
    \end{proof}

    \textit{Второй случай}: $p\in R$ неприводимый в $R[x]$ и $\deg p=0$.\\

    \begin{lemma}
        $p$ неприводим и в $R$.
    \end{lemma}

    \begin{proof}
        Пусть $p=ab$. $a$ и $b$ - тоже какие-то константы, потому что мы в области целостности, и при умножении степени сохраняются. При этом, без ограничения общности, $a\sim 1$ в $R[x]$. Но тогда $a\sim 1$ и в $R$, потому что $R^{\times}=R[x]^{\times}$ (это, похоже, обратимые элементы).
    \end{proof}

    $R$ факториально, поэтому $p$ простой и в $R$ по определению факториальности. Значит, $R/(p)$ - область, а тогда $R/(p)[x]$ - тоже область. И тут наша последняя лемма:\\

    \begin{lemma}
        $R/(p)[x]\cong R[x/(p)]$.
    \end{lemma}

    \begin{proof}
        Посмотрим на $f=\sum a_i x^i \mapsto \sum [a_i]x^i$. Видно, что он сюръективен, а его ядро - это ровно $(p)\subseteq R[x]$. (Тут типа применяется \hyperlink{n25}{теорема о гомоморфизме}).
    \end{proof}

    Значит, $R[x]/(p)$ - тоже область. Значит, $p$ - простой в $R[x]$.

\end{proof}

\resetall

\subsection{Теорема Безу. Производная многочлена и кратные корни}

\begin{theorem}
    (\hypertarget{n33}{\textcolor{red}{\textit{Универсальное свойство кольца многочленов}}}). Пусть $i:R\rightarrow R[x]$ - стандартное вложение (отправляет каждый элемент в себя), $f:R\rightarrow S$ - гомоморфизм колец, $r\in R$ - произвольный элемент. Тогда существует единственный гомоморфизм $ё\bar{f}: R[x]\rightarrow S$ такой, что $f=\bar{f}\circ i$ и $\bar{f}(x)=f(r)$.
\end{theorem}

\begin{proof}
    Раз уж в билетах нет, то и доказательства не будет. Определить его не сложно, показать, что гомоморфизм - тоже.
\end{proof}

\begin{defn}
    По универсальному свойству кольца многочленов дл $f:=\id _R$ и произвольного $r\in R$ существует единственный $\bar{f}:R[x]\rightarrow R$ такой, что $\bar{f}(x)=f(r)=r$. В этом случае $\bar{f}$ называется \textit{гомоморфизмом вычисления} и обозначается за $\ev_r$.
\end{defn}

\begin{theorem}
    (\hypertarget{n34}{\textcolor{red}{\textit{Теорема Безу}}}). $R[x]/(x-a)\simeq R$ посредством $[f(x)]\mapsto f(a)$, где $R$ - коммутативное кольцо с единицей.
\end{theorem}

\begin{proof}
    Понятно, что $x-a\in\Ker \ev_a$, тогда и идеал $(x-a)\subset \Ker \ev_a$.\ 

    Теперь нужно показать в обратную сторону. Здесь делим с остатком: пусть $f:=(x-a)g+c$, где $c=f(a)$. Тогда $[f]=[c]$ в $R[x]/(x-a)$. Значит, $f\in (x-a)R[x]\Leftrightarrow f(a=0)$.
\end{proof}

В общем, опять очередное применение теоремы о гомоморфизме, про образ мы уже понимаем, сам он задаётся корректно, а мы лишь проверяем, что идеал $(x-a)$ есть ядро.\

\begin{proof}
    (Второе доказательство). Сначала заметим, что для $R[x]/(x)\cong R$ теорема очевидна, а затем при помощи универсального свойства кольца многочленов, выполним замену переменной на $(x-a)$. Условно, у нас есть гомоморфизмы в обе стороны $x\rightarrow (x-a)$ и наоборот $(x+a)\leftarrow x$. Тогда $R[x]/(x-a)\cong R[x]/(x)\cong R$.
\end{proof}

\begin{defn}
    Для $f=\sum a_i x^i$ определим $f':=\sum ia_i x^{i-1}$.
\end{defn}

Тогда видно, что $(f+g)'=f'+g'$, а также $(fg)'=f'g+fg'$, что уже проверить сложнее.\\

\begin{lemma}
    $a$ - кратный (кратности больше 1) корень $f$ тогда и только тогда, когда $a$ - корень многочлена и его производной.
\end{lemma}

\begin{proof}
    Поделим $f$ на $(x-a):f=(x-a)g+f(a)=(x-a)g$. $a$ - кратный корень $f$ тогда и только тогда, когда $g(a)=0$. Теперь давайте продифференцируем это выражение: $f'=g+(x-a)g'$. Отсюда $f'(a)=g(a)$. Что и требовалось.
\end{proof}

\resetall

\subsection{Интерполяция Лагранжа}

\begin{theorem}
    (Частично Лагранж). Пусть $F$ - поле, $\{a_0, \dots, a_n\}$ - набор его различных элементов. Тогда 
    \[
        \frac{F[x]}{(\Pi(x-a_i))}\cong F^n
    \]
    посредством
    \[
        f\mapsto (f(a_0), \dots, f(a_{n-1})).
    \]
\end{theorem}

\begin{proof}
    Докажем, что его ядро есть $((x-a_0)\dots (x-a_{n-1}))$. Спроецируем $F^n$ на $F$ и применим \hyperlink{n34}{теорему Безу}. Условно говоря, $f$ лежит в ядре $\ev_{a_0, \dots, a_{n-1}}$, но тогда и в ядре $\ev_{a_i}$, где $ev_{a_i}$ - композиция $F[x]\rightarrow F^n\rightarrow F$, то есть, вычисление значения в конкретной точке. Тогда по теореме Безу мы получили, что $x-a_i\vert f$, но над полем они неприводимы и взаимно просты. Тогда по основной теореме арифмитики получим, что их $\lcm$ есть их произведение, тогда $f$ делится на это произведение. И наоборот, если $\Pi(x-a_i)\vert f$, то $\forall i:f(a_i)=0$. Получилось.\ 

    Докажем теперь сюръективность. найдём прообраз $(b_0, \dots, b_{n-1})$.
    \[
        \sum_i b_i\cdot \frac{\Pi_{j\neq i}(x-a_j)}{\Pi_{j\neq i}(a_i-a_j)}
    \]
    подходит, нетрудно убедиться. Это и есть \textcolor{red}{\hypertarget{35}{\textit{интерполяция по Лагранжу}}} (возможно, надо рассказать только про неё, но тогда совсем пустой билет выходит).
\end{proof}

\resetall

\subsection{Интерполяция Эрмита}

Концептуально, мы хотим научиться ещё как-нибудь интерполировать. Например, по точкам и значениям производных в них (до каких-то определённых порядков).\\

\begin{theorem}
    \[
        \frac{F[x]}{(\Pi(x-a_i)^{m_i})}\cong \frac{F[x]}{((x-a_0)^{m_0})}\times \dots \times \frac{F[x]}{((x-a_n)^{m_n})}
    \]
\end{theorem}

\begin{proof}
    Опять-таки рассмотрим гомоморфизм $f\mapsto (f+(x-a_0)^{m_0}F[x]), \dots , f+(x-a_n)^{m_n}F[x])$. Аналогично предыдущему билету, показываем, что ядро - произведение этих разностей в нужных степенях. Так же раскладываем в проекции по каждому элементу произведения (а была на лекциях такая лемма, что $f[x]$ в $R[x]/((x-a)^m)\longleftrightarrow (f(a), f'(a), \dots, f^{(m-1)})$ (биекция)). $f$ принадлежит ядру, а это равносильно тому, что $(x-a_i)^{m_i}\vert f(x)$, тогда в силу взаимной простоты, произведение $\Pi(x-a_i)^{m_i}$ делит $f(x)$. Тогда ядро и равно этому произведению.\ 

    Теперь нам нужно доказать сюръективность. Зафиксируем $l_i<m_i$. Хотим найти такую $f$, что\\

    \begin{itemize}
        \item $f^{(l)}(a_j)=0$ для $l<m_j, j\neq i$;
        \item $f^{(l)}(a_i)=0$ для $l<m_i, l\neq l_i$;
        \item $f^{(l_i)}(a_i)=1$.
    \end{itemize}

    Будем считать, что для больших значений $l_i$ (то есть, $\{l_i+1, \dots, m_i-1\}$) мы уже всё проделали.\ 

    Первое условие гласит, что $a_j$ должен быть корнем кратонсти хотя бы $m_j$. Значит, $\Pi_{j\neq i}(x-a_j)^{m_j}\vert f$. Второе и третье же влекут, что $\frac{1}{l_i!}(x-a_i)^{l_i}\vert f$. Рассмотрим тогда 
    \[
        f:=C(a_i)\cdot \frac{1}{l_i!}(x-a_i)^{l_i}\Pi_{j\neq i}(x-a_j)^{m_j}, 
    \]
    где 
    \[
        C(x):=\Biggl(\Bigl((x-a_i)^{l_i}\Pi_{j\neq i}(x-a_j)^{m_j}\bigr)^{(l_i)}\Biggr)^{-1}.
    \]
    Он удовлетворяет первому и третьему условию. Единственная проблема состоит в том, чтопроизводные $f^{(l)}$ порядком $l$ выше $l_i$ могут принимать какие-то лишние значения $f^{l}(a_i)$ в точке $a_i$, но эту проблему можно решить, ведь из предположения индукции мы можем вычесть какие-то базисные многочлены с соответствующими коэффициентами и получить новый многочлен, который будет удовлетворять уже всем условиям.\

    Это была \hypertarget{n36}{\textcolor{red}{\textit{интерполяция Эрмита}}}. Простой формулы нет.
\end{proof}

\resetall

\subsection{Поле разложение многочлена}

Пусть $F$ - поле и $f\in F[x]$. Поле многочленов евклидово и факториально, значит, у $f$ есть неприводимый делитель $g$ ($f=gh$) для некоторого $h$. Рассмотрим теперь  $F[x]/(g)$ - область, поскольку $g$ - простой из-за факториальности $F[x]$. Более того, выполнена лемма:\\

\begin{lemma}
    Пусть $R$ - коммутативное кольцо главных идеалов с единице, а $p\in R$ - простой. Тогда $R/(p)$ - поле. 
\end{lemma}

\begin{proof}
    Пусть $[t]\in R/(p)$ таков, что $[t]\neq [0]$. Тогда $(t, p)=(1)$ по определению $p$, и существует линейной представление НОД: $tu+pv=1$ для некоторых $u, v$, откуда сразу $[tu]=[t][u]=1-[pv]=[1]-[0]$. То есть, вот мы и нашли для каждого обратный по умножению.
\end{proof}

То есть, что мы тут делаем. Если $g$ - многочлен над $F[x]/(g)$, то у него появляются корни - $[x]$, например, $g([x])=[g(x)]=[0]$. Тем более, $[x]$ - корень $f$. То есть, его, допустим, у нас есть какой-то многочлен, то по нему можно отфакторизовать и получить $R$ расширенное с дополнительными корнями. Пусть тогда $F[x]/(g)$ - это $F_1$, тогда если $a\in F_1$ - найденный нами в $F_1$ корень $f$, то многочлен по теореме Безу представляется как $f=(x-a)f_1$ для некоторого $f_1$ на единицу меньшей степени. Тогда будем так по индукции присоединять корни $f$, пока не придём к полю $F_n$, где $n:=\deg f$.

\begin{defn}
    $F_n$ из рассуждений выше называется \hypertarget{n37}{\textcolor{red}{\textit{полем разложения}}} $f$.
\end{defn}

\begin{theorem}
    Пусть: $F$ - поле, $f\in F[x]$ - многочлен, $n:=deg f$, $F_n$ - поле разложения, $\varphi: F\rightarrow F_n, \psi : F\rightarrow K$ - какие-то вложения, в $K[x]$ $f$ раскладывается на линейные множители.\ 

    Тогда существует (не обязательно единственное) вложение $\bar{\psi}:F_n\rightarrow K$ такое, что $\psi=\bar{\psi}\circ \varphi$. Вэтом смысле, $F_n$ - наименьшее поле, которое содержит все корни $f$.
\end{theorem}

\begin{proof}
    Инъективность $\bar{\psi}$, как и $\varphi с \psi$, следует попросту из того, что любой гомоморфизм полей инъективен или тривиален (это обсуждалось на лекциях), а $\bar{\psi}$ должен сохранять единицу, так как $\psi$ и $\varphi$ её сохраняют.\ 

    Будем одказывать факт по индукции. Изначально, мы можем взять в качестве $\bar{\psi}$ просто $\psi$, когда мы ещё не присоединили никаких корней. Теперь доказываем переход. Предполагаем, что $F_i\rightarrow K$ имеется. Тогда рассмотрим новый для какого-то свежеприсоединённого $Y: F[Y]\rightarrow F[Y]/(g(Y))\dashrightarrow K$. Нам нужно: придумать куда отправить $Y$, а также, чтобы было выполнено $g(a)=0$ (для всех остальных элементов всё и так уже прекрасно). То есть, нам нужно выбрать корень $g(x)$ внутри $K$, но такой есть в силе того, что $f$ раскладывается внтури этого поля на множители, тогда и $g$ раскладывается (как его делитель по основной теореме арифметики), оттуда и выберем (любой, отсюда и пропадает единственность). Переход доказан.
\end{proof}

Некоторые свойства поля разложение, на всякий случай:\\

\begin{lemma}
    Пусть $R$ - коммутативное кольцо с 1 и задан гомоморфизм $\varphi: \mathbb{Z}\rightarrow R$, $\varphi(1)=1_R$, $\Ker(\varphi)=(p)$.  Если $R$ - область, то $p$ - простое или нуль.
\end{lemma}\

\begin{lemma}
    Пусть $R$ - коммутативное кольцо с единицей. Если $\chard (R)=p$ - простое, то отображение $\varphi: R\rightarrow R, \varphi(x)=x^p$ - гомоморфизм колец.
\end{lemma}\

\begin{lemma}
    Пусть $f(x), g(x)\in F[x]$, $F$ вкладывается в $E$. Тогда $\gcd_{F[x]}(f(x), g(x))\sim \gcd_{E[x]}(f(x), g(x))$
\end{lemma}\

\begin{lemma}
    У многочлена $f(x)\in F[x]$ есть кратный корень тогда и только тогда, когда $f(x)$ и $f'(x)$ не взаимно просты.
\end{lemma}

\resetall

\subsection{Комплексные числа. Решение квадратных уравнений в $\mathbb{С}$}

Пусть есть поле $\mathbb{R}$. Рассмотрим поле разложения $x^2+1\Rightarrow \mathbb{R}[y]/(y^2+1)$. Будем обозначать это поле за $\mathbb{C}$ и называть \hypertarget{n38}{\textcolor{red}{\textit{полем комплексных чисел}}}. При этом вместо $y$ пишут $i$. Это поле хорошо тем, что всякий многочлен в нём имеет корень. Покажем это для квадратных многочленов. Пусть дан многочлен $ax^2+bx+c$, поделим его на $a$ и сделаем замену переменной, полуичим многочлен вида $z^2$=d. То есть, нам нужно научить ся извлекать корень из комплексного числа. Это делается либо ''тупо в лоб'', либо через формулу Муавра:\
$z^n=e^{i\varphi n}=\cos(\varphi n)+i\sin(\varphi n)$. Распишем $c=s(\cos\psi+i\sin\psi)$ для параметров $s, \psi\in\mathbb{R}, s\geq 0$, тогда подойдёт
\[
    z=s^{\frac1n}(\cos(\psi / n)+i\sin(\psi/n)).
\]

Во время рассказа можно упомянуть, что такое модуль и сопряжённое, а также несколько их свойств (которые слишком уж очевидны).

\begin{defn}
    Поле называется \hypertarget{n39}{\textcolor{red}{\textit{алгебраически замкнутым}}}, если любой многочлен $f(x)\in F[x]$ степени хотя бы 1 имеет хот ябы один корень. То есть все многочлены в $F[x]$ раскладываются на линейные множители.
\end{defn}

\resetall

\subsection{Основная теорема алгебры}

\begin{theorem}
    (\hypertarget{n40}{\textcolor{red}{\textit{Основная теорема алгебры}}}). $\mathbb{c}$ - алгебраисески замунутое поле (см конец предыдущего билета). 
\end{theorem}

\begin{proof}
    Будем доказывать, через модули чисел (для комплексных и вещественных это одно и то же). Будем говорить, что последовательность $\{z_n\}_{n\in\mathbb{N}}\subset \mathbb{C}$ сходится к $z_0$, если 
    \[
        \forall \varepsilon >0 \exists N\in \mathbb{N} \forall n>N:\vert z_n-z_0\vert <\varepsilon.
    \]

    \begin{lemma}
        Пусть $z_n$ - последовательность комплексных чисел, $x_n$ и $y_n$ - её вещественная и мнимая части. $\{z_n\}$ сходится к $z_0$ тогда и только тогда, когда $\{x_n\}\rightarrow x_0:=\text{Re}z_0$ и $\{y_n\}\rightarrow y_0:=\text{Im}z_0$.
    \end{lemma}

    \begin{proof}
        \textcolor{red}{$\Leftarrow$} Возьмём $\varepsilon/2$ и такие моменты начиная с $N_1, N_2$ для $x_1, x_2$ соответственно, начиная с которых мнимые и вещественные части попадают в выбранную окрестность. Теперь выберем максимальный из этих моментов и получим требуемое.\ 
        \textcolor{red}{$\Rightarrow$} Возьмём $\varepsilon$, выберем натуральное $N$ так, чтобы $\forall n >N \vert z_n-z_0\vert <\varepsilon$. То есть, $\sqrt{(x_n-x_0)^2+(y_n-y_0)^2}<\varepsilon$, но тогжа и расстояние от мнимой, и расстояние от вещественной части до предела меньше $\varepsilon$.
    \end{proof}

    \begin{lemma}
        (Непрерывность арифметики). Пусть $\{z_n\}\rightarrow z_0, \{w_n\}\rightarrow w_0$. Тогда выполнены правила предела суммы и произведения пределов.
    \end{lemma}

    \begin{cons}
        (Непрерывность многочленов). Пусть $f(z)$ - многочлен из $\mathbb{C}[z], \{z_n\}\rightarrow z_0$. Тогда $\{f(z_n)\}\rightarrow f(z_0)$.
    \end{cons}

    \begin{lemma}
        (Секвенциальная компактность диска). Пусть последовательность $\{z_n\}$ такова, что последовательность из её модулей $\{\vert z_n\vert\}$ ограничена. Тогда из $\{z_n\}$ можно выбрать сходящуюся подпоследовательность $\{z_{n_k}\}\rightarrow z_0$, где $\vert z_n\vert$ - конечное число.
    \end{lemma}

    \begin{proof}
        Пусть $z_n=x_n+y_n i$. Тогда сначала выберем подпоследовательность по $x_i$ (ну она ограничена, по матану мы так умеем), а потом из соответствующих им $y_j$ также выберем сходящуюся подпоследовательность.
    \end{proof}

    \begin{lemma}
        Пусть $f(x)\in\mathbb{C}[z]$ - многочлен ненулевой степени, а $\{z_n\}$ расходится. Тогда $\{f(z_n)\}$ также расходится.
    \end{lemma}

    \begin{proof}
        Пусть 
        \[
            f(z)=a_nz^n+a_{n-1}z^{n-1}+\dots+a_0.
        \]
        Тогда
        \[
            \frac{f(z)}{z^n}=a_n+\frac{a_{n-1}}{z}+\dots+\frac{a_0}{z^n}.
        \]
        При $z\rightarrow \infty$ все члены, кроме $a_n$, стремятся к 0. По непрерывности арифметики получим, что при $z\rightarrow \infty$ $\frac{f(z)}{z^n}\rightarrow a_n$. А значит, $f(z)$ тем более стремится к бесконечности.
    \end{proof}

    \begin{lemma}
        Пусть $f\in\mathbb{C}[z]$ - многочлен ненулевой степени. Пусть $z_0$ - не корень $f$. Тогда существует $z_1\in \mathbb{C}$ такое, что $\vert f(z_1)\vert <\vert f(z_0)\vert $.
    \end{lemma}

    \begin{proof}
        Можно считать, что $z_0=0$ (через замену переменной). Без ограничения общности, $f(0)=1$, так как можем сначала поделить на $f(0)$, а затем обратно домножить. Значит, мы получили, что свободный член равен единице.  Пусть $k$ - следующий посне 0 номер ненулевого коэффициента, тогда
        \[
            f(z)=1+c_kz^k+\dots+c_nz^n, c_k\neq 0.
        \]
        Возьмём $z=wt$, где $t\in (0,1)$ - небольшое вещественное число, а $w$ таково, что $с_kw^k=-1$. Подставим это в $f$ и вынесем $t^k$:
        \[
            f(z)=1-t^k+c_{k+1}w^{k+1}t^{k+1}+\dots+c_nw^nt^n=1-t^k+t^k(c_{k+1}w^{k+1}t+\dots+c_nw^nt^{n-k}).
        \]
    
        Здесь последний множительно в скобках не превосходит $A\cdot(n-k)\cdot t$ для какой-то положительной константы $A$, которая зависит от $\{c_i\}$ и $w$. Таким образом, при
        \[
            t<\frac{1}{(n-k)A}
        \]
        этот множитель меньше единицы: $\vert c_{k+1}w^{k+1}t+\dots+c_nw^{n}t^{n-k}\vert<1$, а тогда $\vert f(z\vert <(1-t^k)+t^k=1)$, что и требовалось.
    \end{proof}

    Осталось доказать основную теорему алгебры. Рассмотрим
    \[
        \inf_{z\in\mathbb{C}}\vert f(z)\vert = m.
    \]
    Если он достигается, то по лемме не может быть положительным, так что $ ь=0$. Пусть не достигается, тогда выберем подпоследовательность $\{z_n\}$ такую, что $\{\vert f(z_n)\vert\}\rightarrow m$. Тогда $\{\vert z_n\vert\}$ ограничена, иначе бы из неё можно было выбрать подпоследовательность, которая стремится к бесконечности, а тогда и многочлен на этой подпоследовательности стремился бы к бесконечности. По секвенциальной компактности диска, выбираем сходящуюся подпоследовательность, стремящуюся к $z_0$. Тогда по непрерывности $f$, значение в $z_0$ равно пределу значений сходящейся подпоследовательности, инфимум всё же достигается.\

    Мы поняли, что $\inf \vert f(z_n)\vert=0$, при этом этот инфинум всё же достигается. Значит, существует $z_0$ такое, что $\vert f(z_0)\vert =0$. Таким образом, у $f$ есть хотя бы один корень.  
\end{proof}

\resetall

\subsection{Разложение рациональной функции в простейшие дроби над $\mathbb{C}$ и над $\mathbb{R}$}

Учимся раскладывать $\frac{f(x)}{g(x)}$, где $g(x)=(x-\alpha_1)^{m_1}\dots(x-\alpha_k)^{m_k}$,   в сумму дробей вида $\frac{c}{(x-\alpha_i)^{l_i}}, c\in\mathbb{C}$, то есть, хотим найти разложение
\[
    \frac{f(x)}{\Pi(x-\alpha_i)^{m_i}}=\sum_{i=1}^k\sum_{l=1}^{m_i} \frac{c_{i, l}}{(x-\alpha_i)^{l_i}}
\]

\begin{theorem}
    \hypertarget{n41}{\textcolor{red}{\textit{Разложение}}} правильной дроби в $\CC(x)$ в сумму простейших существует и притом единственно.
\end{theorem}

\begin{proof}
    Домножим разложение, которое мы хотим найти, на $g(x)$. Теперь хотим найти разложение
    \[
        f(x)=\sum_{i=1}^k\sum_{l=1}^{m_i}c_{i, l}(x-\alpha_i)^{m_i-l}\Pi_{j\neq i}(x-\alpha_j)^{m_j}.
    \]
    Заметим, что в $f(\alpha_i)$ обнуляется всё, кроме одного слагаемого вида $c_{i, l}(a_i-a_i)^0\Pi_{j\neq i}(\alpha_i-a_j)^{m_j}$. Тогда 
    \[
        c_{i, m_i}=\frac{f(\alpha_i)}{\Pi_{j\neq i}(a_i-a_j)^{m_j}}.
    \]
    Теперь из $f(x)$ вычтем те слагаемые из этой суммы, из которых мы нашли коэффициенты. Пусть 
    \[
        f_i(x):=f(x)-\sum_i c_{i, m_i}\Pi_{j\neq i}(x-\alpha_j)^{m_j}.
    \]
    Этот многочлен делится на $(x-\alpha_i)$ для любого $i$. Вычтенное сокращается:
    \[
        f_1=\sum_{i=1}^k\sum_{l=1}^{m_i-1}c_{i, l}(x-\alpha_i)^{m_i-l}\Pi_{j\neq l}(x-\alpha_j)^{m_j}.
    \]
    Большинство слагаемых при оставшихся коэффициентах делятся на $(x-\alpha_i)^2$, а при взятии производной и вычислении на $\alpha_i$ они будут обнуляться. Одно слагаемое останется: то, которое привязано к $c_{i, m_i-1}$. Из значения $f'(\alpha_i)$ так же полкчаем формулу для следующей партии коэффициентов $c_{i, m_i-1}$, по индукции найдём для всех остальных. Из этого сразу будет следовать единственность.\ 

    Будем теперь считать, что мы знаем $c_{i, l}$. Возьмём $h$ равным тому исходному выражению для $f$:
    \[
        h(x):=\sum_{i=1}^k\sum_{l=1}^{m_i}c_{i, l}(x-\alpha_i)^{m_i-l}\Pi_{j\neq i}(x-\alpha)^{m_j}.
    \]
    Хотим показать, что $f(x)=h(x)$. Что мы знаем об этих функциях? По определению коэффициентов $c_{i,j}$ производные $f$ и $g$ в точках $\{\alpha_i\}$ равны, а именно, для каждого $i$
    \[
        f(\alpha_i)=h(\alpha_i), \dots, f^{(m_i-1)(\alpha_i)=h^{(m_i-1)(\alpha_i)}}.
    \]
    А это задача интерполяции Эрмита, у неё единственное решение нужной степени, из этого $h(x)=f(x)$.
\end{proof}

\begin{theorem}
    Разложение правильной дроби в $\RR(x)$ в сумму простейших существует и притом единственно.
\end{theorem}

\begin{proof}
    Есть дробь $\frac{f(x)}{g(x)}$, представим знаменатель $g(x)=\Pi p_i^{m_i}$, где $p_i(x)$ - неприводимые, попарно не ассоциированные. Для начала давайте найдём разложение
    \[
        \frac{f(x)}{g(x)}=\sum_i\frac{a_i(x)}{p_i^{m_i}}, 
    \]
    где все дроби в сумме правильные. Домножим на знаменатель:
    \[
        f(x)=\sum_i a_i(x)\Pi_{j\neq i}p_j^{m_j}.
    \]
    Здесь все слагаемые, кроме одного, делятся на $p_i^{m_i}$.\ 

    Теперь нам надо перейти к фактор-кольцу $F[x]/(p_i^{m_i})$, тогда нам необходимо будет равенство
    \[
        [f(x)]=[a_i(x)]\Bigl[\Pi_{j\neq i}p_j^{m_j}\Bigr], 
    \]
    при этом элементы $[f(x)]$ и $\Bigl[\Pi_{j\neq i}p_j^{m_j}\Bigr]$ мы также знаем, а значит, можем восстановить и $[a_i(x)]$, нам нужно лишь обратимость $\Bigl[\Pi_{j\neq i}p_j^{m_j}\Bigr]$ в $F[x]/(p_i^{m_i})$. В $F[x]/(p_i^{m_i})$ обратимыми будут те элементы, которые взаимно просты с $p_i^{m_i}$, так как $F[x]$ - евклидово кольцо и работает линейное представление НОД. Наше произведение $\Bigl[\Pi_{j\neq i}p_j^{m_j}\Bigr]$ взаимно просто с $p_i^{m_i}$, поэтому обратимо. Можем найти $[a_i(x)]$. При этом мы можем выбрать такой представитель этого класса эквивалентности, что $\deg a_i<\deg p_i^{m_i}$. Получилось, что если разложение существует, то оно единственно.\ 

    Докажем существование. Давайте этим методом составим некоторый многочлен $h(x)$. Тогда мы знаем, что для любого $i$ $h(x)\equiv f(x \mod p_i^{m_i})$. По КТО $h$ и $f$ сравнимы и по модулю произведения $\{p_i^{m_i}\}$, то есть, $g$. А из-за того, что $\deg f< \deg g, \deg h<\deg g$, получаем, что $f(x)=h(x)$. \

    Осталось каждую дробь суммы $\frac{a_i(x)}{p_i^{m_i}}$ разложить в сумму простейших, то есть
    \[
        \frac{a_i}{p_i^{m_i}}=\sum_{l=1}^{m_i}\frac{a_{i,l}(x)}{p_i^l(x)}, 
    \]
    где $\deg a_{i,l}<\deg p_i$. Опять же, нам нужно разложение
    \[
        a_i=\sum_l a_{i, l}p_i^{m_i-l}.
    \]
    В это можно и поверить наслово, но ниже приведём краткое доказательство.
\end{proof}

\begin{lemma}
    (Та самая). Пусть $F$ - поле, $p$ - многочлен ненулевой степени из $F[x]$. Тогда любой многочлен $a\in F[x]$ может быть записан единственным образом в виде 
    \[
        a=a_0+a_1 p+\dots+a_n p^n,
    \]
    где дл] всякого $i$ либо $\deg a_i<\deg p$, либо $a_i=0$.
\end{lemma}

\begin{proof}
    Существование по индукции по степени $a(x)$. База: если $\deg a<\deg p$, то просто возьмэм $a_0(x)=a(x)$. Если же $\deg a \geq \deg p$, то поделим $a(x)$ на $p(x)$ с остатком, получим $a(x)=p(x)q(x)+r(x)$, при этом $\deg r(x)<\deg p(x)$. Посмотрим на степень $q(x)$, 
    \[
        \deg q(x)=\deg(a(x)-r(x))-\deg p(x)=\deg a(x ) -deg p(x) < \deg a(x), 
    \]
    так как $\deg a(x)>\deg r(x)$, а значит, можно применить предположение индукции для $q(x)$:
    \[
        q(x)=a_1(x)+a_2(x)p(x)+\dots+a_n(x)(p(x))^{n-1}, 
    \]
    тогда
    \[
        a(x)=r(x)+q(x)p(x).
    \]
    Единственность тоже по индукции, теперь по $n$. Заметим: что $a_0(x)\equiv b_0(x)\mod p(x)$, так как всё остальное на $p(x)$ делится, при этом их степени меньше степени $p(x)$, а значит, $a_0(x)=b_0(x)$. Тогда сократим эти члены, поделим на $p(x)$ и применим индекционное предположение. 
\end{proof}

\resetall

\subsection{Определение векторного пространства. Линейная зависимость. Существование базиса}

Для начала введём понятия, которых нет в билетах, а также некоторые их свойства, которые пригодятся для определения объектив в этом билете.

\begin{defn}
    Пусть $R$ - кольцо. Тогда \hypertarget{n42}{\textcolor{red}{\textit{левый $R$-модуль}}} - это абелева группа $M$ вместе с операцией $R\times M\rightarrow M, (r, m)\mapsto r\cdot m$, удовлетворяющая следующим свойствам:\
    
    \begin{itemize}
        \item $r\cdot(a+b)=r\cdot a+r\cdot b$;
        \item $(r+s)\cdot a=r\cdot a+s\cdot b$;
        \item $(rs)\cdot a=r\cdot(s\cdot a)$;
        \item (если $R$ - кольцо с 1) $1\cdot a=a$.
    \end{itemize}

    Аналогично определяется \textit{правый $R$-модуль}, только там всё с другой стороны.
\end{defn}

\begin{defn}
    Пусть $M, N$ - $R$-модули. \textit{Прямая сумма $R$-модулей} - $R$-модуль $M\oplus$ с носителем $M\times N$ и покомпонентными операциями:
    \[
        r\cdot (m,n):=(r\cdot m, r\cdot n), (m_1, n_1)+(m_2, n_2):=(m_1+n_1, m_2+n_2).
    \]
    Все свойства $R$-модуля выполняются покоординатно, значит, выполняются.
\end{defn}

\begin{defn}
    Пусть $M$ и $N$ - левые $R$-модули. Отображение $\varphi: M\rightarrow N$ между ними, которое сохраняет операции:
    \[
        \varphi(rm)=r\varphi(m) \text{ и } \varphi(m_1+m_1)=\varphi(m_1)+\varphi(m_2), 
    \]
    называется \textit{гомоморфизмом модулей} или \hypertarget{n48}{\textcolor{red}{\textit{линейным отображением}}}.
\end{defn}

\begin{defn}
    Обозначим за $\{m_i\}$ произвольные элементы модуля, за $\{r_i\}$ - элементы соответствующего кольца.\ 

    \begin{itemize}
        \item Уравнение 
        \[
            \sum_{i=1}^n m_i r_i=0
        \]
        называется \hypertarget{n43}{\textcolor{red}{\textit{линейной зависимостью}}}.
        \item Линейная зависимость \textit{тривиальна}, если все $\{r_i\}$ равны нулю.
        \item Конечный набор $m_1, \dots, m_n$ \textit{линейно независим}, если любая его линейная зависимость тривиальна.
        \item Бесконечный набор $\{m_i\}$ линейно независим, если любой его конечный поднабор линейно независим.
    \end{itemize}
\end{defn}

\begin{lemma}
    Пусть $\{m_i\}_{i\in I}$ - набор элементов свободного модуля $R^{(I)}$. Любой конечный поднабор $\{m_i\}$ линейно независим тогда и только тогда, когда линейное отображение $f\mapsto \sum f(i)m_i$ инъективно.
\end{lemma}

\begin{proof}
    Вспомним, что гомоморфизм групп инъективен тогда и только тогда, когда его ядро тривиально. Тогжа обратим внимание на определение линейной независимости и линейного отображения и поймём, что здесь сказано одно и то же.
\end{proof}

\begin{defn}
    Набор $\{m_i\}$ называется \hypertarget{n44}{\textcolor{red}{\textit{системой образующих}}}, если линейное отображение $f\mapsto \sum f(i)m_i$ сюръективно.
\end{defn}

\begin{cons}
    Эквивалентно, любой элемент модуля можно представить линейной комбинацией конечного поднабора $\{m_i\}$ (в нормальной жизни это называлось порождающей системой).
\end{cons}

\begin{lemma}
    Подмножество $\{m_i\}$ модуля - линейно независимая система образующих тогда и только тогда, когда линейное отображение $f\mapsto \sum f(i)m_i$ - изоморфизм.
\end{lemma}

\begin{defn}
    \hypertarget{n45}{\textcolor{red}{\textit{Базис}}} - линейно независимая система образующих.
\end{defn}

\begin{defn}
    Пусть $F$ - поле. $F$-модуль называется \hypertarget{n46}{\textcolor{red}{\textit{векторным пространством}}} над $F$.
\end{defn}

\begin{theorem}
    (О существовании базиса). Любую линейно независимую систему в векторном пространстве можно дополнить до базиса. В частности, в любом векторном пространстве есть базис.
\end{theorem}

\begin{proof}
    Будем доказывать через лемму Цорна, рассмотрим частично упорядоченное множество $S$, носитель которого - множество линейно независимых систем, содержащих данную, а отношение порядка - включение.\ 

    Рассмотрим супремум (обхединение) некоторой цепи и линейную зависимость какого-то конечного подмножества этого объединения. Каждый вектор в этой линейной зависимости лежит в некотором из элементов цепи. Всего таких векторов конечное число, а цепь по определению линейно упорядочена, поэтому мы можем выбрать наибольший элемент цепи, который содержит каждый их векторов. Этот наибольший набор по определению линейно независим, значит, вся рассматриваемая линейная комбинация тривиальна, Получается, что объединение тоже линейно независимо и содержит исходную систему.\ 

    Теперь применяем лемму Цорна: в нашем линейно упорядоченном множестве $S$ существует максимальный элемент, то есть, максимальная линейно независимая система среди линейно независимых систем, содержащих данную. Но так как любая линейно назависимая система, содержащая максимальную систему, содержит и данную, то максимальная система будет максимальной линейно независимой среди всех линейно независимых систем. По одной из переформулировок это и есть базис (мы их, конечно, не привели, но нетрудно догадаться, что базис это минимальная по включению система образующих или максимальная по включению линейно независимая система).
\end{proof}

\resetall

\subsection{Размерность векторного пространства}

\begin{defn}
    \hypertarget{n47}{\textcolor{red}{\textit{Размерность}}} $\dim V$ векторного пространства $V$ - мощность его базиса.
\end{defn}

Далее мы докажем, что все бащисы векторного пространства равномощны, тем самым определение будет корректно.\\

\begin{theorem}
    Все базисы одного векторного пространства равномощны.
\end{theorem}

\begin{proof}
    1 случай (конечный). Рассмотрим два конечный базиса $V$: $\{v_1, \dots, v_n\}$ и $\{u_1, \dots , u_m\}$. Представим $v_n$ в виде $v_n=\sum_{i=1}^mu_i\alpha_i, \alpha_i\in F$, причём не все $\alpha_i$ равны нулю, так как иначе $v_n=0$ - нетривиальная линейная зависимость первого набора. Не умаляя общности, пусть $\alpha_m\neq 0$. Тогда $v_n\alpha_m^{-1}=u_m+\sum_{i=1}^{m-1}u_i\alpha_i\alpha_m^{-1}$. Заменим в первом базиче $v_n$ на $v_n\alpha_m^{-1}$, и первый базис останется базисом (понятно, что они опять-таки всё порождают, поделим изначальный коэффициент того или иного разложения на $\alpha_m^{-1}$. аналогично, они линейно независимы, так как можно поделить последний коэффициент). Во втором базисе заменим $u_m$ на $u_m+\sum_{i=1}^{m-1}u_i\alpha_i\alpha_m^{-1}$, и второй базис также останется базисом, покажем это. \ 

    Предположим, что полученный набор линейно зависим, то есть, $\sum_i=1^{m-1}u_i\beta_i+(u_m+\sum_{i=1}^{m-1})\beta_m=0$ - некоторая нетривиальная линейная зависимость нового набора. $\beta_m\neq 0$, так как иначе существовала бы нетривиальная зависимость набора $\{u_1, \dots, u_{m-1}\}$. Раскроем скобки, приведём подобные, получим некоторую линейную линейную зависимость изначального второго набора, причём нетривиальную, так как коэффициент при $u_m$ будет равен $\beta_m$, то есть, будет ненулевым. Противоречие с тем, что изначальный второй набор - базис. Значит, полученный набор линейно независим. Аналогично, полученный набор - система образующих, а значит, базис. Таким образом, мы получили два базиса, у которых хотя бы один элемент общий. Так будем преобразовывать бащисы, увеличивая количество общих элементов. В конце концов окажется, что один базис содержится в другом, а тогда они совпадают, так как базис - минимальная по включению система образующих. При этом на каждом шаге мы меняли только по вектору из каждого бащиса, а количество векторов в базичах не меняли. Значит, изначальные базисы равномощны.\ 

    2 случай (бесконечный). Пусть $u=\{u_i\}_{i\in I}, v=\{v_j\}_{j\in J}$ - базисы, причём $\vert I\vert >\vert J\vert $ и $\vert I\vert$ бесконечна. Выразим элементы $u$ через $v:u_i=\sum_{j\in J_i}v_j\alpha_j, \vert J_i\vert<\infty$. Рассмотрим отображение из $I$ в множество конечных подмножеств $J$, которое мы обозначаим как $J'$, действующее по правилу $i\mapsto J_i$. Воспользуемся утверждением, доказанным в курсе теории множеств: если $J$ конечно, то $\vert J'\vert=2^{\vert J\vert}$, то есть, $J'$ конечно, иначе $\vert J'\vert=\vert J\vert$. С исходным неравенством это становится $\vert I\vert>\vert J'\vert$ (если $J$ конечно: то неравенство верно, так как $I$ бесконечно, а $J'$ конечно: иначе $\vert I\vert > \vert J\vert = \vert J'\vert$). \ 

    Рассмотрим конечное подмножество $K$ индексов второго базиса и посчитаем число его прообразов в первом базисе при отображении $i\mapsto J_i$. Векторов, которые переходят в $K$, не может быть больше $\vert K\vert $, поскольку в ином случае, в пространстве, которо порождено векторами $K$, нашлась бы ли линейно независимая система из $\vert K\vert +1$ векторов (любое подмножество базиса $u$ линейно независимо), а это невозможно по рассуждению для конечного случая. Значит, у $K$ конечное число прообразов в $I$. \ 

    Таким образом, опять же по теоретико-множественным соображениям, либо $J'$ конечно, и, следовательно, $I$ тоже конечно, как объединение конечного числа конечных множеств, либо $J'$ бесконечно, и, следовательно, $\vert I\vert = \vert J\vert = \vert J'\vert$, так как $I$ - объединение $\vert J\vert$  множеств конечной мощности каждое.\ 

    Так или иначе, мы получили противоречие с изначальными условиями на $I$ и $J$.
\end{proof}

\begin{cons}
    Любое векторное пространство $V$ над $F$ изоморфно свободнуму модулю $ F^{(I)}$, причём $\vert I\vert$ определена однозначно.
\end{cons}

Теперь изначальное определение корректно.

\resetall

\subsection{Линейные отображения векторных пространств. Подпространство, фактор-пространство. Ранг линейного отображения}

Определение \hyperlink{n48}{линейного отображения} мы уже упоминали в прошлом билете. И тем не менее, сначала опять несколько определений, которые нужны для понимания того, что происходит в самом билете.

\begin{defn}
    Подмножество модуля, которое само является модулем над тем же кольцом и замкнуто относительно всех операций, называется \hypertarget{n49}{\textcolor{red}{\textit{подмодулем}}}.
\end{defn}

\begin{defn}
    Пусть $M$ - левый $R$-модуль, $N$ - подмодуль. \hypertarget{n50}{\textcolor{red}{\textit{Фактормодулем}}} $R/N$ называется фактор $R$ по отношению эквивалентности 
    \[
        a\equiv b \Longleftrightarrow a-b\in N
    \]
    с определёнными на нём операциями
    \[
        [a]+[b]:=[a+b], r\cdot[a]:=[ra].
    \]
\end{defn}

Нетрудно теперь заменить несколько слов и понять, что такое подпространство и фактор-пространство. 

\begin{defn}
    \hypertarget{n51}{\textcolor{red}{\textit{Коразмерность}}} $\codim_V U$ подпространства $U\leq V$ равна $\dim V/U$.
\end{defn}

\begin{defn}
    Пусть $U\leq V$ - векторные пространства. Тогда \hypertarget{n52}{\textcolor{red}{\textit{относительный базис}}} - дополнение какого-то базиса $U$ до базиса $V$.
\end{defn}

Теперь можно немного подробнее поговорить о линейных отображениях именно векторных пространств.\\ 

Пусть $f:V\rightarrow W$ - линейное. Скажем, что ядро есть $U\leq V$, тогда найдём относительный базис $\{v_i\}_{i\in I}$. Тогда $f$ определяется своими значениями именно на этом относительном базисе. Значения $\{f(v_i)\}_{i\in I}$ внутри $W$ должны быть линейно независимы, иначе:
\[
    \sum_i f(v_i)\alpha_i=0 \Leftrightarrow f(\sum_i v_i\alpha_i)=0, \text{ но тогда }\sum_i v_i\alpha_i\in \Ker f=U,
\]
а затем $\{f(v_i)\}_{i\in I}$ дополняется до базиса в $W$. То есть, базис $V$ разбивается на две части: одна зануляется, другая во что-то переходит и как-то дополняется в $W$ до базиса там.

\begin{defn}
    Пусть $f$ - линейное отображение векторных пространств. Тогда размерность образа $f$ называется \hypertarget{n53}{\textcolor{red}{\textit{рангом}}} $f$ и обозначается 
    \[
        \rank f:=\dim \Imf f=\codim \Ker f.
    \]
    (на более простом языке - это количество единичных столбцов в матрице преобразования, написанной для правильных базисов).
\end{defn}

\resetall

\subsection{Матрица линейного отображения. Композиция линейных отображений и произведение матриц. Кольцо матриц}

Матрица линейного отображения $F^n\rightarrow F^m$ будет выглядеть так: $n$ столбцов высоты $m$, где каждый столбец - образ какого-то элемента базиса первого пространства (то есть, его разложение в базис второго пространства по строчкам).\\

Пусть у нас имеется композиция линейных отображений $b\circ a, a:F^n\rightarrow F^m, b: F^m \rightarrow F^p$. Пусть $\{e_j\}$ - базис первого, $\{f_i\}$ - базис второго, $\{g_k\}$ - базис третьего. Тогда
\[
    a(e_j)=\sum_{i=1}^m f_i a_{ij},
\]
где $a_{ij}$ - матричные коэффициенты на пересечении $i$-ой строки и $j$-го столбца. Аналогично задаём образ второго базиса:
\[
    b(f_i)=\sum_{k=1}^p g_k b_{ki}.
\]
Тогда мы можем найти $b\circ a$:
\[
    (b\circ a)(e_j)=b(\sum_{i=1}^m f_i a_{ij})=\sum_{i=1}^m b(f_i)a_{ij}=\sum_{i=1}^m\sum_{k=1}^p g_k b_{ki}a_{ij}.
\]
Конечно, знаки суммирования можно переставить. И что всё это означает? Это означает, что если мы хотим найти матричный коэффициент композиции, то нужно рассмотреть сумму:
\[
    (b\circ a)_{kj}=\sum_{i=1}^m b_{ki}a_{ij}.
\]
То есть, количество строк матрицы преобразования $a$ должно совпасть с количеством столбцов матрицы преобразования $b$.\\

Отсюда видно, что квадратные матрицы образуют кольцо (причём некоммутативное при размерности большей 1). Сложение в этом кольце определено покоординатно, умножение мы задали, остаётся только проверить свойства, проверяется это покоординатно.

\resetall

\subsection{Элементарные преобразования. Метод Гаусса. Системы линейных уравнений}

Предупреждение: в билете происходит словесный понос, так как в начале, по традиции, немного ''предыстории'', а затем идёт описание сюжета, зачем и почему мы получаем элементарные преобразования и другие объекты, а также, самый сок: метод Гаусса и решение СЛУ. Так что, при ответе на билет можете ''выжать'' важную информацию отсюда, которая обозначена красными определениями, а также, небольшую часть общего рассуждения.\\

Для начала рассмотрим \hypertarget{n54}{\textcolor{red}{\textit{замену базиса}}}, которую вряд ли можно назвать элементарным преобразованием, но тем не менее. Когда мы совершаем линейное отображение, важно, какой базис мы выбрали, потому что от этого зависит, как будет выглядеть матрица отображения. Однако давайте заметим, что базис можно поменять посредством тождественного преобразования из векторного пространства в себя, которое записывается \textit{квадратной матрицей}, в которой каждый ноэлемент одного базиса расписан через элементы второго. Тогда чтобы найти, как выглядит матрица преобразования в другом базисе, нужно умножить матрицу преобразования базиса на матрицу отображения. Также можно заменить базис получившегося пространства, но по сути, это то же самое, только умножаем мы с другой стороны.\

\begin{defn}
    Такая матрица смены базиса называется \hypertarget{n55}{\textcolor{red}{\textit{матрицей перехода}}}.
\end{defn}

\begin{defn}
    \textit{Ранг} матрицы отображения, конечно, не превосходит длин сторон матрицы (лучше рассматривать всё это в правильных базисах (с единичками)), то есть, $r\leq \min(n, m)$, а если $r=\min(m,n)$, то отображение будет называться отображением \hypertarget{n57}{\textcolor{red}{\textit{полного ранга}}}.
\end{defn}

Для того, чтобы определить метод Гаусса, обратимся к такой задаче:

\begin{exl}
    Дана матрица $m\times n$ ($m$ строк, $n$ столбцов), необходимо найти все обратимые матрицы $m\times m$ и $n\times n$ (то есть, такие, что в произведении они дадут следующую матрицу).
\end{exl}

\begin{equation*}
    \begin{pmatrix}
        1 & 0 & \cdots & 0 & 0 & \cdots & 0 \\
        0 & 1 & \cdots & 0 & 0 & \cdots & 0 \\
        \vdots & \vdots & \ddots & \vdots & \vdots & \cdots & \vdots \\
        0 & 0 & \cdots & 1 & 0 & \cdots & 0 \\
        0 & 0 & \cdots & 0 & 0 & \cdots & 0 \\
        \vdots & \vdots & \vdots & \vdots & \vdots & \ddots & \vdots \\
        0 & 0 & \cdots & 0 & 0 & \cdots & 0
    \end{pmatrix}
\end{equation*}\

То есть, она диагональная до какого-то момента, а потом всё - нули (и, конечно, она не обязательно квадратная). В общем и целом, мы сейчас будем искать набор каких-то заведомо обратимых матриц. \

\begin{defn}
    \hypertarget{n58}{\textcolor{red}{\textit{Трансвекция}}} (или элементарное преобразование первого рода) - квадратная матрица, полученная из единичной путём добавления на $ij$-ую позицию какого-то элемента $\xi$, записывается это так: $t_{ij}(\xi)$ - трансвекция.
\end{defn}

\begin{cons}
    Трансвекции обратимы: $t_{ij}(\xi)\times t_{ig}(-\xi)=I$ (кстати, вообще, $t_{ij}(\alpha)\times t_{ig}(\beta)=t_{ig}(\alpha+\beta))$, единичная матрица (кстати, тоже трансвекция, только от нуля).
\end{cons}

\begin{cons} Что же вообще они дают? \
    \begin{itemize}
    \item Умножение матрицы \textit{слева} на $t_{ij}(\xi)$, прибавляет к $i$-ой строчке $j$-ую строчку, домноженную на $\xi$ (слева);
    \item Умножение матрицы \textit{справа} на $t_{ij}(\xi)$, прибавляет к $j$-му столбцу $i$-ый столбец, домноженный на $\xi$ (справа).
    \end{itemize}
\end{cons}

Далее для удобства будем рассматривать только умножение слева на произведение трансвекций. И давайте поймём, как можно переставлять строчки в матрице. Если мы хотим переставить строчки $i, j$, то нужно выполнить умножение на таукю матрицу:

\begin{equation*}
    \begin{pmatrix}
        1 & 0 & \cdots & 0 & \cdots & 0 & \cdots & 0 \\
        0 & 1 & \cdots & 0 & \cdots & 0 & \cdots & 0 \\
        \vdots & \vdots & \ddots & \vdots & \vdots & \vdots & \cdots & \vdots \\
        0 & 0 & \cdots & 0 & \cdots & 1 & \cdots & 0 \\
        \vdots & \vdots & \vdots & \vdots & \ddots & \vdots & \cdots & \vdots \\
        0 & 0 & \cdots & 1 & \cdots & 0 & \cdots & 0 \\
        \vdots & \vdots & \vdots & \vdots & \vdots & \vdots & \ddots & \vdots \\
        0 & 0 & \cdots & 0 & \cdots & 0 & \cdots & 1
    \end{pmatrix}
\end{equation*}\

Где просто в прямоугольничке, образованном $i, j$ строками и столбцами мы переставили 0 и 1 в углах. Если умножит матрицу на такую слева, то мы как раз переставим $i, j$ - строки (если справа, то, конечно, столбцы). Но такую матрицу получить невозможно. А возможно такую: 

\begin{equation*}
    \begin{pmatrix}
        1 & 0 & \cdots & 0 & \cdots & 0 & \cdots & 0 \\
        0 & 1 & \cdots & 0 & \cdots & 0 & \cdots & 0 \\
        \vdots & \vdots & \ddots & \vdots & \vdots & \vdots & \cdots & \vdots \\
        0 & 0 & \cdots & 0 & \cdots & 1 & \cdots & 0 \\
        \vdots & \vdots & \vdots & \vdots & \ddots & \vdots & \cdots & \vdots \\
        0 & 0 & \cdots & -1 & \cdots & 0 & \cdots & 0 \\
        \vdots & \vdots & \vdots & \vdots & \vdots & \vdots & \ddots & \vdots \\
        0 & 0 & \cdots & 0 & \cdots & 0 & \cdots & 1
    \end{pmatrix}
\end{equation*}\

Но тогду у новой $j$-ой строчки мы поменяем знак. Делаем мы это при помощи трансвекций таким образом: $(u,v)\rightarrow (u+v, v)\rightarrow (u+v, -u)\rightarrow (v, -u)$, где $u, v$ мы обозначили строчки. Тогда даже можно выразить явную формулу: $t_{ij}(1)\times t_{ij}(-1)\times t_{ij}(1)$.\ 

\begin{defn}
    \hypertarget{n59}{\textcolor{red}{\textit{Псевдоотражение}}} ($\text{di}_i(\varepsilon)$) - преобразование (элементарное второго рода), которое представляется в виде матрицы, у которой на пересечении $i$-ой строки и столбца стоит не $1$, а $\varepsilon$. Описывает она домножение строки (или столбца, в зависимости от того, с какой стороны умножаем), на $\varepsilon$. Однако получить её путём умножения трансвекций мы не можем. Также $\varepsilon \in R^\times$, то есть, в поле не равен нулю.
\end{defn}

Однако, мы можем при помощи трансвекций получить достаточно похожую матрицу:

\begin{equation*}
    \begin{pmatrix}
        1 & 0 & \cdots & 0 & \cdots & 0 & \cdots & 0 \\
        0 & 1 & \cdots & 0 & \cdots & 0 & \cdots & 0 \\
        \vdots & \vdots & \ddots & \vdots & \vdots & \vdots & \cdots & \vdots \\
        0 & 0 & \cdots & \varepsilon & \cdots & 0 & \cdots & 0 \\
        \vdots & \vdots & \vdots & \vdots & \ddots & \vdots & \cdots & \vdots \\
        0 & 0 & \cdots & 0 & \cdots & \varepsilon^{-1} & \cdots & 0 \\
        \vdots & \vdots & \vdots & \vdots & \vdots & \vdots & \ddots & \vdots \\
        0 & 0 & \cdots & 0 & \cdots & 0 & \cdots & 1
    \end{pmatrix}
\end{equation*}\

Которая на диагонали содержит взаимно обратную ''пару'' к элементу, который мы размещаем. Такую матрицу можно получить произведением шести трансвекций, всё это можно проделать на матрицах размерности $2$. Тогда при умножении матрицы на данную, мы $i$-ую строчку умножим на $\varepsilon$, а $j$-ую на него поделим.\ 

Итого, у нас есть определённый набор операций (которые мы можем получить из произведения трансвекций). Или же, \hypertarget{n60}{\textcolor{red}{\textit{элементарные преобразования}}}: \

\begin{itemize}
    \item прибавить к строчке какую-то другую с коэффициентом;
    \item переставить две строчки (у одной из них изменится знак);
    \item умножить одну строчку на элемент, но другую на него поделить.
\end{itemize}

Теперь мы этими операциями будем приводить матрицу \hypertarget{n61}{\textcolor{red}{\textit{методом Гаусса}}} (сведение к ступенчатому виду).\ 

Будем приводить матрицу к \textit{эшелонированному виду}. Рекомендуется заглянуть в лекцию №19, потому что всё же, я не нарисую все картинки с доски, а ограничусь только краткими словесными рассуждениями. \ 

Для начала умножаем только слева на трансвекции (работаем только со строками). Рассмотрим первый (слева направо) ненулевой столбец, поменяем строчки местами так, чтобы в первой строке в этом столбце стоял не нуль. Теперь успешно повычитаем первую строку с коэффициентом из остальных так, чтобы уничтожить все остальные ненулевые элементы в первом столбце. Теперь рассмотрим матрицу без первых нулевых и этого столбца, а также верхней строки. Проделаем для неё ту же самую операцию, и так далее много раз. В итоге либо ''упрёмся в стенку'', либо снизу останутся нулевые строки. Теперь все \hypertarget{n62}{\textcolor{red}{\textit{ведущие}}} элементы (которые мы оставляли сверху в столцах, и количество которых, кстати, равно рангу матрицы), поочерёдно будем превращать в единицы, путём деления одного на себя, и умножая на него следующий. Если снизу остались нулевые строки, то в единичку мы можем превратить и последний ведущий элемент, иначе это невозможно.\ 

Теперь вспоминаем, что мы ещё умеем умножать на трансвекции справа, и уничтожим всё в строках, кроме ведущих. И, наконец, переставим их последовательно (у последнего ведущего опять же может остаться знак ''-'', но с этим ничего не поделаешь, это инвариант матрицы, который называется \textit{определителем}, однако сейчас не об этом). То есть, если матрица не квадратная матрица, то мы всегда можем выстроить по диагонали единицы, а в случае квадратной матрицы, в последней позиции диагонали мы можем получить не $1$, а некоторый другой элемент.\\

Наконец, \hypertarget{n63}{\textcolor{red}{\textit{системы линейных уравнений}}} - система вида:

\begin{equation*}
    \begin{cases}
        a_{11}x_1+\dots+a_{1n}x_n=b_1\\
        \cdots \\
        a_{m1}x_1+\dots+a_{mn}x_n=b_m
    \end{cases}
\end{equation*}\

Где $a_{ij}$ и $b_i$ - какие-то элементы поля, я $x_j$ - переменные, значения которых мы хотим выяснить.\ 

Скажем, что у нас есть несколько матриц:

\begin{equation*}
    A = 
    \begin{pmatrix}
        a_{11} & \cdots & a_{1n} \\
        \vdots & \ddots & \vdots \\
        a_{m1} & \cdots & a_{mn}
    \end{pmatrix},
    x=
    \begin{pmatrix}
        x_1 \\
        \vdots \\
        x_n
    \end{pmatrix},
    b=
    \begin{pmatrix}
        b_1 \\
        \vdots \\
        b_m
    \end{pmatrix}
\end{equation*}\

Они есть $A: F^n\rightarrow F^m, x\in F^n, b\in F^m$.\ 

Бывает, конечно, два случая: либо решений нет, либо решение есть. Если решение есть, то мы можем из него получить и все остальные, путём прибавления элементов из ядра $A$. Но тогда нужно как-то решить уравнение $Ax=0$, что не сильно-то легче, чем решить изначальное уравнение $Ax=b$. Кстати, здесь можно также заметить, что решение системы существует тогда и только тогда, когда $b\in \Imf A$. Теперь в ходе рассуждений можно упомянуть следующую лемму, которой, конечно, в билетах нет:\\

\begin{lemma}
    (Размеронсти ядра и образа). Пусть $f: V\rightarrow V'$ - линейное отображение векторных пространств. Тогда
    \[
        \dim V = \dim \Imf f +\dim \Ker f
    \]
\end{lemma}

А так как ранг есть размерность образа, то $\dim \Ker A = n - \rank A$. Но вообще, это не очень-то и важно, а просто как вспомогательный факт. Нам же нужно научиться понимать, есть ли решения у системы, и, если есть, то как их находить.\ 

Если у нас имеется $Ax=b$, то можно домножить это уравнение на некую квадратную матрицу $P \Rightarrow PAx=Pb$, где $P$ - обратимая матрица, чтобы не потерять никакой информации, то есть, она выражается через трансвекции. Так как мы умножаем слева, то по сути, мы решаем систему ''школьным методом'' - комбинирование строк. Приводим матрицу к ступенчатому виду, причём лучше сразу привести матрицу к \hypertarget{n64}{\textcolor{red}{\textit{расширенному виду}}}, то есть, к матрице $A$ приписать справа столбец $b$ и производить все операции сразу на такой расширенной матрице. Тогда если мы получили, что в части только с $А$ есть нулевые строчки, которые равны каким-то ненулевым значениям в столбце $b$, то мы получили противоречие, решений нет. Иначе рассмотрим последнюю ненулевыю строку, для всех переменных кроме одной выберем произвольные значения (параметры), а последнее будет задано тогда однозначно. И так дальше будем подниматься по строчкам, получим решение всей системы.\\

Перед тем, как дорешать систему, мы пришли как раз к теореме Кронекера-Капелли, так как как раз количество ведущих элементов равно рангу, и если в матрице $A$ и в матрице расширенной количество ведущих совпадает, то и противоречия с последними ненулевыми строками не будет. Иначе, оно появляется, и только в таком случае система решений не имеет.

\resetall

\subsection{Теорема Кронекера-Капелли}

\begin{theorem}
    (\hypertarget{n65}{\textcolor{red}{\textit{Теорема Кронекера-Капелли}}}). Система линейных уравнений имеет решение тогда и только тогда, когда ранг исходной матрицы равен рангу расширенной матрицы.
\end{theorem}\

Все рассуждения см. начиная с \hyperlink{n54}{системы линейных уравнений} в предыдущем билете.

\resetall

\subsection{Определение группы. Циклическая группа. Порядок элемента}

\subsection{Группа перестановок. Циклы, транспозиции. Знак перестановки}

\subsection{Действие группы на множестве. Орбиты. Классы сопряженности}

\subsection{Группа обратимых элементов кольца. Вычисление обратимых элементов $\mathbb{Z}/_{n\mathbb{Z}}$. Функция Эйлера}

\subsection{Гомоморфизмы и изоморфизмы групп. Смежные классы, теорема Лагранжа. Теорема Эйлера}

\subsection{Многочлены деления круга}

\subsection{Конечные поля (существование, единственность, цикличность мультипликативной группы)}

\subsection{Фактор-группа, теорема о гомоморфизме}

\subsection{Определитель матрицы. Инвариантность при элементарных преобразованиях, разложение по строчке и столбцу}

\subsection{Присоединенная матрица. Формула Крамера. Определитель транспонированной матрицы}

\subsection{Вычисление определителя методом Гаусса}

\subsection{Принцип продолжения алгебраических тождеств. Определитель произведения матриц}



\newpage

\hypertarget{t2}{И в заключение...}



\section{Пофамильный указатель всех мразей}

\begin{multicols}{2}
    [
    Быстрый список для особо заебавшегося поиска.
    ]

    \hyperlink{n42}{$R$-модуль}\
    
    \hyperlink{n39}{алгебраическая замкнутость}\
    
    \hyperlink{n4}{ассоциированность}\

    \hyperlink{n45}{базис}

    \hyperlink{n61}{ведущие элементы}\

    \hyperlink{n46}{векторное пространство}\

    \hyperlink{n19}{гомоморфизм}\
    
    \hyperlink{n3}{делитель нуля}\

    \hyperlink{n5}{евклидово кольцо}\

    \hyperlink{n54}{замена базиса}\

    \hyperlink{n9}{идеал}\

    \hyperlink{n22}{изоморфизм}\

    \hyperlink{n35}{интерполяция по Лагранжу}\

    \hyperlink{n36}{интерполяция по Эрмиту}\
    
    \hyperlink{n1}{кольцо, а также его вариации}\

    \hyperlink{n14}{кольцо вычетов}\

    \hyperlink{n28}{кольцо многочленов}\

    \hyperlink{n51}{коразмерность}\

    \hyperlink{n15}{КТО}\

    \hyperlink{n31}{лемма Гаусса}\

    \hyperlink{n43}{линейная зависимость}\

    \hyperlink{n48}{линейное отображение}\

    \hyperlink{n55}{матрица перехода}\

    \hyperlink{n60}{метод Гаусса}\

    \hyperlink{n27}{многочлен}\

    \hyperlink{n7}{неприводимые}\

    \hyperlink{n6}{НОД}\

    \hyperlink{n21}{образ}\

    \hyperlink{n10}{ОГИ}\

    \hyperlink{n13}{ОТАр}\

    \hyperlink{n40}{ОТАл}\

    \hyperlink{n52}{относительный базис}\

    \hyperlink{n2}{область целостности}\

    \hyperlink{n49}{подмодуль}\

    \hyperlink{n16}{поле}\

    \hyperlink{n38}{поле комплексных}\

    \hyperlink{n37}{поле разложения}\

    \hyperlink{n18}{поле частных}\

    \hyperlink{n56}{полный ранг}\

    \hyperlink{n8}{простые}\ 

    \hyperlink{n58}{псевдоотражение}\

    \hyperlink{n41}{разложение рациональных ф-й}\

    \hyperlink{n47}{размерность}\

    \hyperlink{n53}{ранг отображения}\

    \hyperlink{n63}{расширенный вид матрицы}\

    \hyperlink{n44}{система образующих}\

    \hyperlink{n62}{СЛУ}\

    \hyperlink{n30}{содержание}\

    \hyperlink{n23}{сравнимость}\

    \hyperlink{n29}{степень многочлена}\

    \hyperlink{n34}{теорема Безу}\

    \hyperlink{n32}{теорема Гаусса}\

    \hyperlink{n64}{теорема Кроненера-Капелли}\

    \hyperlink{n25}{теорема о гомоморфизме}\

    \hyperlink{n57}{трансвекция}\

    \hyperlink{n33}{универсальное св-во кольца мн-ч}\

    \hyperlink{n26}{универсальное св-во фактор-кольца}\

    \hyperlink{n11}{УОВЦГИ}\

    \hyperlink{n50}{фактормодуль}\

    \hyperlink{n12}{факториальность}\

    \hyperlink{n24}{фактор-кольцо}\

    \hyperlink{n59}{элементарные преобразования}\

    \hyperlink{n20}{ядро}\
    

\end{multicols}



\end{document}