\documentclass[a4paper,100pt]{article}

\usepackage[utf8]{inputenc}
\usepackage[unicode, pdftex]{hyperref}
\usepackage{cmap}
\usepackage{mathtext}
\usepackage{multicol}
\setlength{\columnsep}{1cm}
\usepackage[T2A]{fontenc}
\usepackage[english,russian]{babel}
\usepackage{amsmath,amsfonts,amssymb,amsthm,mathtools}
\usepackage{icomma}
\usepackage{euscript}
\usepackage{mathrsfs}
\usepackage{geometry}
\usepackage[usenames]{color}
\hypersetup{
     colorlinks=true,
     linkcolor=red,
     filecolor=red,
     citecolor =black,      
     urlcolor=cyan,
     }
\usepackage{fancyhdr}
\pagestyle{fancy} 
\fancyhead{} 
\fancyhead[LE,RO]{\thepage} 
\fancyhead[CO]{\hyperlink{t2}{к списку объектов}}
\fancyhead[LO]{\hyperlink{t1}{к содержанию}} 
\fancyhead[CE]{текст-центр-четные} 
\fancyfoot{}
\newtheoremstyle{indented}{0 pt}{0 pt}{\itshape}{}{\bfseries}{. }{0 em}{ }

%\geometry{verbose,a4paper,tmargin=2cm,bmargin=2cm,lmargin=2.5cm,rmargin=1.5cm}
\title{Алгебра}
\author{Мастера конспектов}
\date{22 января 2020 г.}

\theoremstyle{indented}
\newtheorem{theorem}{Теорема}
\newtheorem{lemma}{Лемма}

\theoremstyle{definition} 
\newtheorem{defn}{Определение}
\newtheorem{exl}{Пример(ы)}

\theoremstyle{remark} 
\newtheorem{remark}{Примечание}
\newtheorem{cons}{Следствие}

\begin{document}

\newcommand{\resetexlcounters}{%
  \setcounter{exl}{0}%
  
} 

\newcommand{\resetremarkcounters}{%
  \setcounter{remark}{0}%
  
} 

\newcommand{\reseconscounters}{%
  \setcounter{cons}{0}%
  
} 

\newcommand{\resetall}{%
    \resetexlcounters
    \resetremarkcounters
    \reseconscounters%

}

\maketitle 

\newpage

\hypertarget{t1}{Честно говоря, ненависть} к этой вашей топологии просто невообразимая.
\tableofcontents

\newpage



\section{Билеты}



\subsection{Определение кольца. Простейшие следствия из аксиом. Примеры. Области целостности}

\medskip

\begin{defn}
    \hypertarget{n1}{\textcolor{red}{\textit{Кольцом}}} называется множество $R$ вместе с бинарными операциями $+$ и $\cdot$ (которые называются сложением и умножением соответственно), удовлетворяющим аксиомам:\

    \begin{itemize}
        \item операция сложения ассоциативна;
        \item по отношению к сложению существует нейтральный элемент;
        \item у каждого элемента есть обратный по сложению
        \item операция сложения коммутативна;
        \item умножение ассоциативно;
        \item умножение дистрибутивно по сложеиню.
    \end{itemize}
\end{defn}

Также можно добавить, что если на множестве выполныны три первые аксиомы, то оно будет называться \textit{группой}, а если выполнены первые четыре, то это уже \textit{абелева группа}. Нейтральный по сложению элемент кольца называют \textit{нулём}.

\begin{exl}
    Кольцо называется:\

    \begin{itemize}
        \item \textit{коммутативным}, если оно коммутативно по умножению;
        \item \textit{кольцом с единицей}, если оно содержит нейтральный элемент по умножению (единица);
        \item \textit{телом}, если в нём есть 1, и для любых $a\neq 0 \rightarrow a \cdot a^{-1}=a^{-1}\cdot a=1$;
        \item \textit{полем}, если это коммутативное тело;
        \item \textit{полукольцом}, если нет требования противоположного элемента по сложению.
    \end{itemize}
\end{exl}

\begin{cons}
    Некоторые следствия из аксиом:\

    \begin{itemize}
        \item $0\cdot a = 0$
        \begin{proof}
            \[
               0\cdot a = (0+0)\cdot a = 0\cdot a + 0\cdot a
            \]
            Прибавим к обеим частям $-0\cdot a$ и получим требуемое.
        \end{proof}
        \item Нейтральный элемент по сложению единственный
        \begin{proof}
            Рассмотрим их сумму справа и слева.
        \end{proof}
    \end{itemize}
\end{cons}

\begin{defn}
    Коммутативное кольцо $R$ с единицей, обладающее свойством
    \[
        xy=0 \Longrightarrow x=0 \vee y=0 \text{ }(\forall x, y\in R)
    \]
    называется \hypertarget{n2}{\textcolor{red}{\textit{областью целостности}}} или просто \textit{областью}.
\end{defn}

\begin{defn}
    Число $d\neq 0$ называется \hypertarget{n3}{\textcolor{red}{\textit{делителем нуля}}}, если существует такое $d'\neq 0$, что $dd'=0$.
\end{defn}

Нетрудно понять, что область целостности - в точности коммутативное кольцо с единицей без делителей нуля.

\resetall

\subsection{Евклидовы кольца. Евклидовость $\mathbb{Z}$. Неприводимые и простые элементы.}

\subsection{Идеалы, главные идеалы. Евклидово кольцо как кольцо главных идеалов}

\subsection{Основная теорема арифметики}

\subsection{Кольцо вычетов $\mathbb{Z}/_{n\mathbb{Z}}$. Китайская теорема об остатках}

\subsection{Определение поля. $\mathbb{Z}/_{p\mathbb{Z}}$ как поле. Поле частных целостного кольца}

\subsection{Определение гомоморфизма и изоморфизма колец. Фактор-кольцо}

\subsection{Теорема о гомоморфизме}

\subsection{Кольцо многочленов. Целостность и евклидовость кольца многочленов над полем}

\subsection{Лемма Гаусса}

\subsection{Факториальность кольца многочленов}

\subsection{Теорема Безу. Производная многочлена и кратные корни}

\subsection{Интерполяция Лагранжа}

\subsection{Интерполяция Эрмита}

\subsection{Поле разложение многочлена}

\subsection{Комплексные числа. Решение квадратных уравнений в $\mathbb{С}$}

\subsection{Основная теорема алгебры}

\subsection{Разложение рациональной функции в простейшие дроби над $\mathbb{C}$ и над $\mathbb{R}$}

\subsection{Определение векторного пространства. Линейная зависимость. Существование базиса}

\subsection{Размерность векторного пространства}

\subsection{Линейные отображения векторных пространств. Подпространство, фактор-пространство. Ранг линейного отображения}

\subsection{Матрица линейного отображения. Композиция линейных отображений и произведение матриц. Кольцо матриц}

\subsection{Элементарные преобразования. Метод Гаусса. Системы линейных уравнений}

\subsection{Теорема Кронекера-Капелли}

\subsection{Определение группы. Циклическая группа. Порядок элемента}

\subsection{Группа перестановок. Циклы, транспозиции. Знак перестановки}

\subsection{Действие группы на множестве. Орбиты. Классы сопряженности}

\subsection{Группа обратимых элементов кольца. Вычисление обратимых элементов $\mathbb{Z}/_{n\mathbb{Z}}$. Функция Эйлера}

\subsection{Гомоморфизмы и изоморфизмы групп. Смежные классы, теорема Лагранжа. Теорема Эйлера}

\subsection{Многочлены деления круга}

\subsection{Конечные поля (существование, единственность, цикличность мультипликативной группы)}

\subsection{Фактор-группа, теорема о гомоморфизме}

\subsection{Определитель матрицы. Инвариантность при элементарных преобразованиях, разложение по строчке и столбцу}

\subsection{Присоединенная матрица. Формула Крамера. Определитель транспонированной матрицы}

\subsection{Вычисление определителя методом Гаусса}

\subsection{Принцип продолжения алгебраических тождеств. Определитель произведения матриц}



\newpage

\hypertarget{t2}{И в заключение...}



\section{Пофамильный указатель всех мразей}

\begin{multicols}{2}
    [
    Быстрый список для особо заебавшегося поиска.
    ]

    \hyperlink{n3}{делитель нуля}\
    
    \hyperlink{n1}{кольцо, а также его вариации}\

    \hyperlink{n2}{область целостности}\
    

\end{multicols}



\end{document}