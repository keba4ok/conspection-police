\documentclass[a4paper,100pt]{article}

\usepackage[utf8]{inputenc}
\usepackage[unicode, pdftex]{hyperref}
\usepackage{cmap}
\usepackage{mathtext}
\usepackage{multicol}
\setlength{\columnsep}{1cm}
\usepackage[T2A]{fontenc}
\usepackage[english,russian]{babel}
\usepackage{amsmath,amsfonts,amssymb,amsthm,mathtools}
\usepackage{icomma}
\usepackage{euscript}
\usepackage{mathrsfs}
\usepackage{geometry}
\usepackage[usenames]{color}
\hypersetup{
     colorlinks=true,
     linkcolor=magenta,
     filecolor=magenta,
     citecolor = black,      
     urlcolor=cyan,
     }
\usepackage{fancyhdr}
\pagestyle{fancy} 
\fancyhead{} 
\fancyhead[LE,RO]{\thepage} 
\fancyhead[CO]{\hyperlink{t2}{к списку объектов}}
\fancyhead[LO]{\hyperlink{t1}{к содержанию}} 
\fancyhead[CE]{текст-центр-четные} 
\fancyfoot{}
\newtheoremstyle{indented}{0 pt}{0 pt}{\itshape}{}{\bfseries}{. }{0 em}{ }

%\geometry{verbose,a4paper,tmargin=2cm,bmargin=2cm,lmargin=2.5cm,rmargin=1.5cm}
\title{Геометрия и Топология}
\author{Мастера конспектов}
\date{14 января 2020 г.}

\theoremstyle{indented}
\newtheorem{theorem}{Теорема}

\begin{document}

\maketitle 

\newpage

\hypertarget{t1}{Table of contents}
\tableofcontents

\newpage

\section{Билет} \

\medskip

\textbf{Метрические пространства, произведение метрических пространств, пространство $\mathbb{R}^n$.} \\

Функция $d: X \times X \rightarrow \mathbb{R}_+ = \{x \in \mathbb{R} : x\geq 0\}$ называется \hypertarget{n1}{\textit{метрикой}} (или \textit{расстоянием}) 
в множестве $X$, если \

\medskip

1. $d(x, y)=0 \Leftrightarrow x=y$; \\
\indent
2. $d(x, y)= d(y, x)$ для любых $x, y \in X$; \\
\indent
3. $d(x, y)\leq d(x, z)+d(z, y)$. \\

Пара $(X, d)$, где $d$ - метрика в $X$, называется \textit{метрическим пространством}. \\

\begin{theorem}
(Прямое произведение матриц). Пусть $(X, d_X)$ и $(Y, d_Y)$ - метрические пространства. Тогда функция
\[
    d((x_1, y_1), (x_2, y_2))=\sqrt{d_X(x_1, x_2)^2+d_y(y_1, y_2)^2}
\]
задаёт метрику на $X\times Y$.
\end{theorem}

\begin{proof}
1 и 2 аксиомы очевидны. Проверим выполнение третьей. Сделать это несложно, нужно всего лишь написать неравенство и дважды возвести в квадрат. Можно как-нибудь поиспользовать Коши или КБШ, на ваш вкус.
\end{proof}

\medskip

Пространство $X=\mathbb{R}^n, x=(x_1, \dots, x_n), y=(y_1, \dots, y+n)$, на котором задана метрика

\[
    d(x, y)=\sqrt{(x_1-y_2)^2+\dots+(x_n-y_n)^2}
\]


(которая называется \textit{евклидовой}), есть $\mathbb{R}^n$.

\section{Билет} \

\medskip

\textbf{Шары и сферы. Открытые множества в метрическом пространстве. Объединения и пересечения открытых множеств.}
    \begin{itemize}
    \item Пусть $\left(X, d\right)$ --- метрическое пространство, $a \in X, r \in \mathbb R, r > 0.$
    
    Множества 
    \[
    \begin{aligned}
        B_r(a) &= \{x \in X : d(a,x) < r\}, \\
        \overline{B_r} (a) = D_r &(a) = \{x \in X: d(a,x) \leq r\}. \\
    \end{aligned}
    \]
    называются, соответственно, открытым шаром (или просто шаром) и замкнутым шаром пространства $\left(X,d\right)$ с центром в точке $a$ и радиусом $r$.
    \item Пусть $\left(X, d\right)$ --- метрическое пространство, $A \subseteq X$. Множество $A$ называется открытым в метрическом пространстве, если
    \[
        \forall a \in A \exists r > 0: B_r(a) \subseteq A.
    \]
    \textbf{Примеры:} \begin{itemize}
        \item $\varnothing$, $X$ и $B_r(a)$ открыты в произвольном метрическом пространстве $X$.
        \item В пространстве с дискретной метрикой любое множество открыто.
    \end{itemize}
    \item 
    \begin{theorem}
    
    В произвольном метрическом пространстве $X$ 
    \begin{enumerate}
        \item объединение любого набора открытых множеств открыто;
        \item пересечение конечного набора открытых множеств открыто.
    \end{enumerate}
    \end{theorem}
    \begin{proof}
    \
    \begin{enumerate}
        \item Пусть $\left\{U_i\right\}_{i \in I}$ --- семейство открытых множеств в $X$.
        Хотим доказать, что $U = \bigcup_{i \in I}U_i$ --- открыто.
        \[
            x \in U \Rightarrow \exists j \in I : x \in U_j \Rightarrow \exists r > 0: B_r(x) \subseteq U_j \subseteq U.
        \]
        \item
        Пусть семейство $\left\{U_i\right\}_{i=1}^n$ --- семейство открытых множеств в $X$.
        Хотим доказать, что $U = \bigcap_{i=1}^n U_i$ --- открыто.
        \[
            \begin{aligned}
        x \in U \Rightarrow \forall i&: x \in U_i \Rightarrow \exists r_i: B_{r_i} (x) \subseteq U_i; \\ 
        r :&= min\{r_i\} \Rightarrow B_r \subseteq U.
            \end{aligned}
        \]
    \end{enumerate}
    \end{proof}
    \end{itemize}



  
\section{Билет} \

\medskip 

\textbf{Топологические пространства. Замкнутые множества, их объединения и пересечения. Замкнутость канторова множества.}\\

\textbf{Определение:} Пусть $X$ - произвольное множество, и множество $\Omega \subset \rho (X)$ обладает следующими свойствами:
\begin{itemize}
  \item $\emptyset, X \in \Omega$
  \item Объединение любого набора множеств из $\Omega$ также лежит в $\Omega$
  \item Пересечение любого конечного набора множеств из $\Omega$ также лежит в $\Omega$
\end{itemize}
В таком случае:
\begin{itemize}
 \item $\Omega$ - $\emph{топологическая структура}$ (или $\emph{топология}$) на $X$.
 \item Множество $X$ с выделенной топологической структурой называется $\emph{топологическим пространством}$
 \item Элементы множества $\Omega$ называются $\emph{открытыми множествами}$ пространства $(X, \Omega)$
\end{itemize}
\textbf{Определение:} Множество $F \subseteq X$ называется \textit{замкнутым} в $X$, если $X \backslash F$ открыто
\\
\begin{theorem}
    В произвольном топологическом пространстве $X$:
\begin{enumerate}
    \item $\emptyset$ и $X$ замкнуты
    \item Объединение любого конечного набора замкнутых множеств замкнуто
    \item Пересечение любого набора замкнутых множеств замкнуто
\end{enumerate}
\end{theorem}

\begin{proof}
Замкнутость множеств их всех трёх пунктов проверяется по определению: 
\begin{enumerate}
    \item $\emptyset=X \backslash X$ и $X=X \backslash \emptyset$
    \item $X \backslash \bigcap F_i = \bigcup (X \backslash F_i)$
    \item $X \backslash \bigcup F_i = \bigcap (X \backslash F_i)$
\end{enumerate}
В пунктах (b) и (c) мы использовали формулы Де Моргана.
\end{proof}

\textbf{Примеры:} 
\begin{itemize}
 \item В дискретной топологии все множества замкнуты
 \item В антидискретной топологии замкнуты только $\emptyset$ и $X$
 \item В метрическом пространстве любое одноточечное множество замкнуто.
 \begin{proof}
    $X \backslash \{a\}=\bigcup_{b \in X \backslash \{a\}} B_{d(b, a)}(b)$ - открыто.
 \end{proof}
 \item В метрическом пространстве любой замкнутый шар замкнут
 \begin{proof}
    Для каждой точки $b \in X \backslash D_r(a)$ можно выбрать открытый шар $B_{d(b, a)-r}(b)$, который, во-первых, корректно определён (так как $b \notin D_r(a) \Rightarrow d(b, a) > r$), а во-вторых, не содержит точек из $D_r(a)$ (так как если $c \in B$ и $c \in D$, то $\Rightarrow d(c, b) < d(b, a)-r \Rightarrow d(c, a) \geq d(c, b) + d(b, a) > r$ и $ d(c, a) \leq r$, противоречие)
 \end{proof}
 \item Канторово множество замкнуто в стандартной топологи на $\mathbb{R}$
 \begin{proof}
    Следует из построения множества.
 \end{proof}
\end{itemize}

\textbf{Утверждение-сюрприз от leon.tyumen: }Пусть $U$ открыто в $X$, а $V$ замкнуто. Тогда:
\begin{itemize}
    \item $U \textbackslash V$ открыто в $X$.
    \begin{proof}
     $U \textbackslash V = U \cap (X \textbackslash V)$
    \end{proof}
    \item $V \textbackslash U$ замкнуто в $X$.
    \begin{proof}
     $V \textbackslash U = V \cap (X \textbackslash U)$
    \end{proof}
\end{itemize}

\section{Билет} \

\medskip
    
\textbf{Внутренность, замыкание и граница множества: определение и свойства включения, объединения, пересечения.} \\

Пусть $(X, \Omega)$ - топологическое пространство и $A\subseteq X$. \textit{Внутренностью} множества $A$ называется объединение всех открытых множество, содержащихся в $A$, т. е.:
\[
    \text{Int} A = \bigcup_{U\in \Omega, U\subseteq A} U.
\]

Свойства:
\begin{itemize}
    
    \item $\text{Int} A $ - открытое множество;
    \item $\text{Int} A \subseteq A$;
    \item $B$ открыто, $B\subseteq A \Rightarrow B\subseteq \text{Int} A$;
    \item $A = \text{Int}A \Leftrightarrow A$ открыто;
    \item $\text{Int}(\text{Int} A)=\text{Int}A$;
    \item $A \subseteq B \Rightarrow \text{Int} A \subseteq \text{Int} B$;
    \item $\text{Int}(A\cap B) = \text{Int}A\cap\text{Int}B$; \\ 
    \textit{Доказательство}:\\
    $\subseteq: A\cap B\subseteq A \Rightarrow \text{Int} (A\cap B)\subseteq \text{Int}A \dots$; \\
    $\supseteq: \text{Int}A\cap\text{Int} B \subseteq A\cap B \Rightarrow \text{Int}A\cap\text{Int}B \subseteq \text{Int}(A\cap B)$.
    \item $\text{Int}(A\cup B)\supseteq \text{Int}A\cup\text{Int}B$; \\
    \textit{Доказательство $\neq$}:\\
    $X = \mathbb{R}, A = \mathbb{Q}, B=\mathbb{R}\backslash \mathbb{Q}$, \\
    $\text{Int}A=\text{Int}B = \emptyset , \text{Int}(A\cup B) = \text{Int}\mathbb{R} = \mathbb{R}$


\end{itemize}

Пусть $(X, \Omega)$ - топологическое пространство и $A\subseteq X$. \textit{Замыканием} множества $A$ называется пересечение всех замкнутых множество, содержащих $A$, т. е.:
\[
    \text{Cl} A = \bigcap_{X\backslash V\in \Omega, V\supseteq A} V.
\]

Свойства:
\begin{itemize}
    
    \item $\text{Cl}A$ - замкнутое множество;
    \item $A\subseteq \text{Cl}A$;
    \item $B$ замкнуто, $B\supseteq A \rightarrow B \supseteq \text{Cl}A$;
    \item $A = \text{Cl}A \Leftrightarrow A$ замкнуто;
    \item $\text{Cl}(\text{Cl}A)=\text{Cl}A$;
    \item $A\subseteq B \rightarrow \text{Cl}A\subseteq \text{Cl}B$;
    \item $\text{Cl}(A\cup B)=\text{Cl}A\cup \text{Cl}B$;
    \item $\text{Cl}(A\cap B)\subseteq \text{Cl}A\cap\text{Cl}B $ (на самом деле, даже $\neq$);
    \item \textcolor{magenta}{$\text{Cl}A=X\backslash \text{Int}(X\backslash(X\backslash A))$}.

\end{itemize}

Пусть $(X, \Omega)$ - топологическое пространство и $A\subseteq X$. Тогда \textit{границей} множества $A$ называется разность его замыкания и внутренности: $\text{Fr}A = \text{Cl}A\backslash \text{Int}A$. \\

Свойства:
\begin{itemize}

    \item $\text{Fr} A$ - замкнутое множество;
    \item $\text{Fr} A = \text{Fr}(X\backslash A)$;
    \item $A$ замкнуто $\Leftrightarrow A \supseteq \text{Fr} A$;
    \item $A$ открыто $\Leftrightarrow A \cap \text{Fr} A = \emptyset$.

\end{itemize}

\section{Билет} \

\medskip

\textbf{ Расположение точки относительно множества: внутренние и граничные точки, точки прикосновения, предельные и изолированные точки. Внутренность, замыкание и граница множества: из каких точек они состоят.}
\begin{itemize}
    \item Определения($A$ - множество в топологическом пространстве):
    \begin{enumerate}
        \item Окрестностью точки топологического пространства называется любое открытое множество, содержащее эту точку.
        \item Точка называется внутренней для $A$, если некоторая её окрестность содержится в $A$.
        \item Точка называется точкой прикосновения для $A$, если любая её окрестность пересекается с $A$.
        \item Точка называется граничной для $A$, если любая её окрестность пересекается с $A$ и с дополнением $A$
        \item Точка называется изолированной для $A$, если она лежит в $A$ и некоторая её окрестность пересекается по $A$ ровно по этой точке.
        \item Точка называется предельной для $A$, если любая её выколотая окрестность пересекается с $A$.
    \end{enumerate}
    \textcolor{magenta}{Примеры... :(}
    
    \item 
    \begin{enumerate}
        \item Внутренность множества есть множество его внутренних точек:
            \begin{itemize}
                \item $b$ --- внутр. точка для $A \Rightarrow \exists U_\varepsilon(b) \subseteq A \Rightarrow U_\varepsilon(b) \subseteq Int A  \Rightarrow b \in IntA$;
                \item $b \in Int A \Rightarrow b \text{ лежит в $A$ вместе с окрестностью } IntA \Rightarrow b$ --- внутренняя точка для $A$. 
            \end{itemize}
        \item Замыкание множества есть множество его точек прикосновения:
                $b$ --- точка прикосновения для $A \iff b \notin Int(X \setminus A) \iff b \in ClA$
        \item Граница множества есть множество его граничных точек:
        
            $b$ --- граничная точка множества $A \iff (b \in ClA) \wedge (b \in Cl(X \setminus~ A)) \iff (b\in ClA) \wedge (b \notin IntA) \iff b \in FrA$.
        \item Замыкание множества есть объединение множеств предельных и изолированных точек:
        
        $b \in ClA \iff b$ --- точка прикосновения $\iff$ любая окрестность $b$ пересекается с $A \iff$ либо любая выколотая окрестность $b$ пересекается с $A$, либо существует выколотая окрестность, не пересекающаяся с $A$(тогда $b \in A$)$\iff$ либо $b$ --- предельная точка, либо $b$ --- изолированная точка.     
        \item Замыкание множества есть объединение граничных и внутренних точек:
        
        $b \in ClA \iff b$ --- точка прикосновения $\iff$ любая окрестность $b$ пересекается с $A \iff$ либо любая окрестность $b$ пересекается с $X \setminus A$, либо существует окрестность, которая не пересекается $X \setminus A \iff$ либо $b$ --- граничная точка, либо $b$ --- внутренняя точка.
    \end{enumerate}
\end{itemize}


\section{Билет} \

\medskip 

\textbf{Cравнение метрик и топологий (грубее/тоньше). Липшицево эквивалентные метрики.}\\

\textbf{Определение: } Топология $\Omega_1$ \textit{слабее (грубее)} топологии $\Omega_2$ на $X$, если $\Omega_1 \subseteq \Omega_2$. В этом случае топология $\Omega_2$ \textit{сильнее (тоньше)} топологии $\Omega_1$
\\

\textbf{Пример: } Из всех топологичских структур на $X$ антидискретная топология - самая грубая; дискретная топология - самая тонкая.
\\
\begin{theorem}
 Топология метрики $d_1$ грубее топологии метрики $d_2$ $\Longleftrightarrow$ в любом шаре метрики $d_1$ содержится шар метрики $d_2$ с тем же центром
\end{theorem}

\begin{proof} 
    $"\Rightarrow"$ шар $B_r^{d_1}(a)$ открыт в $d_2$ $\Rightarrow$ точка $a$ входит в $B_r^{d_2}(a)$ вместе с некоторой своей окрестностью $B_q^{d_2}(a)$ 
$"\Leftarrow"$ $U$ открыто в $d_1$ $\Rightarrow$ $\forall a \in U \exists q>0 : B_q^{d_1}(a) \subseteq U \Rightarrow \exists r > 0 : B_r^{d_2}(a) \subseteq B_q^{d_1}(a) \subseteq U \Rightarrow U$ открыто в $d_2$ 
\end{proof}

\textbf{Следствие 1:} Пусть $d_1, d_2$ - две метрики на $X$. Если $d_1(a, b) \leq d_2(a, b)$ для любых $a, b \in X$, то топология $d_1$ грубее топологии $d_2$

\begin{proof} $d_1 \leq d_2 \Rightarrow \forall r > 0 \forall a \in X B_r^{d_2}(a) \subseteq B_r^{d_1}(a) \Longleftrightarrow$ топология $d_1$ грубее топологии $d_2$
\end{proof}

\textbf{Определение:} Две метрики в одном множестве называются эквивалентными, если они порождают одну и ту же топологию.
\\

\textbf{Лемма: } Пусть $(X, d)$ - метрическое пространство. Тогда для любого $C>0$ функция $C \cdot d$ - тоже метрика, причём эквивалентная метрике $d$.

\begin{proof} НУ ОЧЕВИДНО ЖЕ
\end{proof}

\textbf{Следствие 2:} Пусть $d_1, d_2$ - две метрики на $X$, причём для любых $a, b \in X$ выполнено $d_1(a, b) \leq Cd_2(a, b)$. Тогда топология $d_1$ грубее топологии $d_2$.

\begin{proof} По лемме $d_2$ и $C \cdot d_2$ эквивалентны, а по следствию 1 $d_1$ грубее $C \cdot d_2$
\end{proof}

\textbf{Определение: } Метрики $d_1, d_2$ называются \textit{липшицево эквивалентными}, если существуют $c, C > 0$ такие, что для любых $a, b \in X$ $c \cdot d_2(a, b) \leq d_1(a, b) \leq C \cdot d_2(a, b)$
\\
\begin{theorem}
Если метрики $d_1$ и $d_2$ липшицево эквивалентны, то они эквивалентны.
\end{theorem}

\begin{proof} Согласно следствию 2, каждая из метрик грубее другой $\Rightarrow$ они эквивалентны.
\end{proof}
 
\textbf{Упражнение:} Верно ли обратное утверждение?
\\

\textbf{Ответ на упражнение: } Ну вроде верно (порождаемые топологии-то совпадают), но что-то как-то странно.
\\

\textbf{Пример: } Три метрики на $\mathbb{R}^2$ - Евклидова, $\max\{|x_1-x_2|, |y_1-y_2|\}$ и $|x_1-x_2|+|y_1-y_2|$ эквивалентны (точки - это $(x_1, y_1)$ и $(x_2, y_2)$).

\begin{proof} Нетрудно проверить, что $\max\{|x_1-x_2|, |y_1-y_2|\}<\sqrt{(x_1-x_2)^2+(y_1-y_2)^2} < 2 \cdot \max\{|x_1-x_2|, |y_1-y_2|\}$, поэтому первая и вторая метрики эквивалентны по предыдущей теореме. Аналогично, $\max\{|x_1-x_2|, |y_1-y_2|\}< |x_1-x_2|+|y_1-y_2| < 2 \cdot \max\{|x_1-x_2|, |y_1-y_2|\}$, поэтому вторая и третья метрики также эквивалентны.
\end{proof}

\textbf{Определение: } Топологическое пространство называется \textit{метризуемым}, если существует метрика, порождающая его топологию.
\\

\textbf{Примеры: } 
\begin{itemize}
    \item Дискретная топология порождается дискретной метрикой 
    \item $X$ с антидискретной топологией неметризуемо при $|X|>1$
    
    \begin{proof} Пусть $a, b \in X$ ($a \neq b$), и $r=d(a, b)$. Тогда шар $B_r(a)$ открыт, непуст (так как $a \in B_r(a)$) и не совпадает со всем пространством (так как $b \notin B_r(a)$)
    \end{proof}
\end{itemize}

\section{Билет} \

\medskip

\textbf{База топологии: два определения и их эквивалентность. Критерий базы}\\

\textit{Базой} топологии $\Omega$ называется такой набор $\Sigma$ открытых множеств, что всякое открытое множество представимо в виде объединения множество из $\Sigma$.

\[
    \Omega \supseteq \Sigma \text{ - база } \Leftrightarrow \forall U \in \Omega \exists \Lambda \subseteq \Sigma : U = \bigcup_{W\in \Lambda} W.
\]

\begin{theorem}
    (Второе определение базы). Пусть $(X, \Omega)$ - топологическое пространстви и $\Sigma \subseteq \Omega$. $\Sigma$ - база топологии $\Omega$ $\Longleftrightarrow \forall U \in \Omega \forall a \in U \exists V_a \in \Sigma : a\in V_a \subseteq U$.
\end{theorem}

\begin{proof} Совсем немного формулок: \

    \begin{itemize}
        \item $\forall U\in \Omega$ и $\forall a\in U$. \\
        $\Sigma - \text{база} \Rightarrow \exists \Lambda \subseteq \Sigma : U = \bigcup _{W\in \Lambda}W \Rightarrow \exists V_a \in \Lambda : a\in V_a$
        \item $\forall U \in \Omega : U = \bigcup_{a\in U}V_a$.
    \end{itemize}
\end{proof}

\begin{theorem}
    (Критерий базы). Пусть $X$ - произвольное множество и $\Sigma = \{ A_i \}_{i\in I}$ - его покрытие. $\Sigma$ - база некоторой топологии $\Longleftrightarrow \forall A_s, A_m \in \Sigma \exists J_{s, m}\subseteq I:A_s\cap A_m = \bigcup_{j\in J_{s, m}}A_j$.
\end{theorem}   

\begin{proof}
    Докажем факт в обе стороны: \ 
    \textcolor{magenta}{$\Rightarrow$} По определению базы и открытости множеств $A_s \cup A_m$. \ 
    \textcolor{magenta}{$\Leftarrow$} Пусть $\Omega$ - совокупность всевозможных объединений множеств из $\Sigma$. Докажем, что $\Omega $ - топология на $X$. \
    \begin{itemize}
        \item $\Sigma$ - покрытие для $X \Rightarrow X \in \Omega$;
        \item объединение объединений есть объединение;
        \item $U, V \in \Omega \Rightarrow U = \bigcup_{s\in S\subseteq I}A_s $ и $V = \bigcup_{m\in M\subseteq I} A_m$, 
        \[
            U\cap V = \bigcup_{s,m}(A_s\cap A_m) = \bigcup_{s,m}\biggl( \bigcup_{j\in J_{s,m}}A_j\biggr)\in \Omega.
        \]
    \end{itemize}

\end{proof}    

\section{Билет} \

\textbf{База топологии в точке. Связь между базой топологии и базами в точках. Предбаза топологии, как из неё получается база.}\\
    \begin{itemize}
    \item 
    Пусть $\left(X, \Omega\right)$ --- топологическое пространство, $a \in X$ и $\Lambda \subseteq \Omega. \Lambda$ называется базой топологии(базой окерестностей) в точке $a$, если:
    \begin{enumerate}
        \item $\forall U \in \Lambda: a \in U;$ 
        \item $\forall U_\varepsilon (a) \exists V_a \in \Lambda: V_a \subseteq U_\varepsilon (a).$
    \end{enumerate}
    
    \textit{Следствия}
    
    \begin{enumerate}
        \item $\Sigma$ --- база топологии $\Rightarrow \forall a \in X \; \Sigma_a := \left\{ U \in \Sigma: a \in U \right\}$ --- база в точке $a$. 
        \item Пусть $\left\{ \Sigma_a\right\}_{a \in X}$ --- семейство баз во всех точках. Тогда $\bigcup_{a \in X} \Sigma_a$ --- база топологии.
    \end{enumerate}
    
    \textbf{Пример}
    
    Множество $\Sigma_a = \left\{B_r(a): r \in \mathbb R_+ \right\}$ является базой метрического пространства в точке а.
    \item
    Набор $\Delta$ открытых множеств топологического пространства $\left(X, \Omega \right)$ называется предбазой топологии, если $\Omega$ --- наименьшая по включению топология, содержащая $\Delta$.
    
    \begin{theorem}
        Любой набор $\Delta$ подмножеств множества $X$ является предбазой некоторой топологии на $X$.
    \end{theorem}
    \begin{proof}
    
         Очевидно, $\Delta$ будет предбазой \textit{топологии объединений конечных пересечений подмножеств $\Delta$} ($X \bigcup \left\{ \cup\{\cap_{i=1}^k W_i\}\right\}, W_i\in~ \Delta$).
    \end{proof}
    \textbf{Пример} (\textit{Следствие})
    
    База топологии является её предбазой.
    \end{itemize}


\section{Билет} \

\medskip

\textbf{Топология подпространства. Свойства: открытость и замкнутость подмножеств, база индуцированной топологии, транзитивность, согласованность с метрическим случаем.}\\

\textbf{Определение:} Пусть $(X, \Omega)$ - топологическое пространство, и $A \subseteq X$. Тогда совокупность $\Omega_A = \{U\cap A : U \in \Omega \}$ - топология на множестве $A$.

\begin{proof} Просто проверка аксиом топологии.
\end{proof}

\textbf{Определение:} 
\begin{itemize}
    \item $\Omega_A$ - \textit{индуцированная топология}
    \item $(A, \Omega_A)$ - \textit{подпространство} пространства $(X, \Omega)$.
\end{itemize}
\textbf{Свойства:}
\begin{itemize}
    \item Множества, открытые в подпространстве, не обязательно открыты в самом пространстве
    \\
    \textbf{Пример:} $X=\mathbb{R}, A=[0, 1]$. Тогда $[0, 1)$ открыто в $A$, но не в $X$.
    
    \item Открытые множества открытого подпространства открыты и во всём пространстве.
    \begin{proof} $U$ открыто в $A \subseteq X$ $\Rightarrow \exists V \in \Omega : U=V \cap A$, т.е. открыто в $X$, как пересечение двух открытых множеств.
    \end{proof}

    \item Множества, замкнутые в подпространстве, не обязательно замкнуты в самом пространстве
    \\
    \\
    \textbf{Пример:} $X=\mathbb{R}, A=(0, 1)$. Тогда $(0, \frac{1}{2}]=(0, 1) \backslash (\frac{1}{2}, 1))$ замкнуто в $A$, но не в $X$.
    \item Замкнутые множества замкнутого подпространства замкнуты и во всём пространстве.
    \begin{proof} $U$ замкнуто в $A \subseteq X$ $\Rightarrow \exists V \in \Omega : A \textbackslash U=V \cap A$, но тогда $X \textbackslash U = (X \textbackslash A) \cup V$ т.е. открыто в $X$, как объединение двух открытых множеств. Значит, $U$ замкнуто в $X$.
    \end{proof}
    \item \textbf{База индуцированной топологии:} Если $\Sigma$ - база топологии $\Omega$, то $\Sigma_A=\{U \cap A : U \in \Sigma\}$ - база топологии $\Omega_A$.
    \begin{proof}
    Просто проверка определения базы.
    \end{proof}
    \item \textbf{"Транзитивность" индуцированных топологий:} Пусть $X$ - топологическое пространство, и $B \subseteq A \subseteq X$. Тогда $(\Omega_A)_B=\Omega_B$
    \begin{proof}
    Так как $U \cap B = (U \cap A) \cap B$, то $\Omega_B \subseteq (\Omega_A)_B$. Покажем обратное. Пусть $U \in (\Omega_A)_B$. Это значит, что существует открытое в $A$ множество $V$ такое, что $U=B \cap V$. $V$ открыто в $A$ $\Rightarrow$ существует открытое в $X$ множество $W$ такое, что $V=X \cap W$. Но тогда $U=B \cap (X \cap W) = B \cap W$ $\Rightarrow$ $U$ открыто в $X$.
    \end{proof}
    \item \textbf{Связь с метрическим случаем:} Пусть $(X, d)$ - метрическое пространство, и $A \subseteq X$. Рассмотрим метричесоке пространство $(A, d_{|A})$, а также порождаемую его метрикой топологию $\Omega_A''$. Кроме того, рассмотрим топологичесоке пространство  $(X, \Omega)$, порождаемую метрикой $d$, и его сужение $(A, \Omega_A)$ на $A$. Тогда $\Omega_A=\Omega_A''$
    \begin{proof}
    $U \in \Omega_A'' \Longleftrightarrow U = \bigcup B_{r_i}^A(a_i) \Longleftrightarrow U \overset{a_i \in A}{=} A \cup \Big ( \bigcup B_{r_i}^X(a_i) \Big ) \overset{(!)}{\Longrightarrow} U=A \cap V \Longleftrightarrow U \in \Omega_A$. Таким образом, мы доказали одно вложение, и для полного счастья нам не хватает только равносильности в моменте $(!)$. Мы победим, если для данного $U$, открытого в $A$, сможем выбрать открытое $V \in X$ такое, что $U=V \cap A$, и $V$ представляется в виде объединения шаров из $X$ с центрами из $A$. Но действительно, поскольку $U$ открыто в $A$, существует какое-то $V' : U=V' \cap A$. Рассмотрим $V= \bigcup_{a_i \in V' \cap A} B_{r_i}^X(a_i)$, где $B_{r_i}^X(a_i)$ - шары, полностю содержащиеся в $X$ (они существуют в силу его открытости). Тогда $U=V \cap A$, $V$ открыто в $X$ и удовлетворяет условию, которое мы так от него ждали.
    \end{proof}
\end{itemize}

\section{Билет} \

\medskip

\textbf{Непрерывные отображения. Непрерывность композиции и сужения, замена области значений.}\\

Пусть $X, Y$ - топологические пространства. Отображение $f: X \rightarrow Y$ называется \textit{непрерывным}, если прообраз любого открытого множества пространства $Y$ является открытым подмножеством пространства $X$.\\

Также можно упомянуть, что отображение непрерывно тогда и только тогда, когда прообраз любого замкнутого множества замкнут. 

\begin{proof}
    $f^{-1}(Y\backslash U)=X\backslash f^{-1}(U)$.
\end{proof}

\begin{theorem}
    (О композиции непрерывных). Композиция непрерывных отображений непрерывна.
\end{theorem}

\begin{proof}
    Пусть $f:X\rightarrow Y$, $g:Y\rightarrow Z$ - непрерывны. Если $U\in \Omega_Z$, то $g^{-1}(U)\in \Omega_Y$, значит, $(g\circ f)^{-1}(U)=f^{-1}(g^{-1}(U))\in\Omega_X$.
\end{proof}

\begin{theorem}
    (О сужении отображения). Пусть $Z$ - подпространство $X$ и $f: X\rightarrow Y$ непрерывно. Тогда $f|_Z:Z\rightarrow Y$ непрерывно.
\end{theorem}

\begin{proof}
    $\text{in}_Z: Z\rightarrow X$ непрерывно, а также $f|_Z = f\circ \text{in}_Z$.
\end{proof}

\begin{theorem}
    (Об изменении области значений). Пусть $Z$ - подпространство $Y$, $f:X\rightarrow Y$ - отображение и $f(X)\subseteq Z$. Пусть $\tilde{f}: X\rightarrow Z$, т. ч. $\tilde{f}(x)=f(x)$. Тогда $f$ непрерывная $\Longleftrightarrow$ $\tilde{f}$ непрерывна.
\end{theorem}

\begin{proof}
    Докажем факт в обе стороны: \
    \textcolor{magenta}{$\Leftarrow$} $f=\text{in}_Z\circ \tilde{f}$.\
    \textcolor{magenta}{$\Rightarrow$} $\forall U \in \Omega_Z \exists W \in \Omega_Y : U=W\cup Z$. $\tilde{f}^{-1}(U) = f^{-1}(W)\in \Omega_X$.
\end{proof}

\section{Билет} \

\textbf{Непрерывность в точке. Глобальная непрерывность эквивалентна непрерывности в каждой точке. Непрерывность и база окрестностей в точке.}
    \begin{itemize}
        \item Отображение $f: X \rightarrow Y$ называется непрерывным в точке $a \in X$, если
        $\forall U_\varepsilon (f(a)) \; \exists V_\delta (a) : f(V_\delta (a)) \subseteq U_\varepsilon (f(a))$.
        \textcolor{magenta}{Пример:( ...}
        \begin{theorem}
            Отображение $f: X \rightarrow Y$ непрерывно $\iff$ оно непрерывно в каждой точке пространства.
        \end{theorem}
        \textit{Доказательство:}
    
            \textcolor{magenta}{($\Rightarrow$)} Очевидно, $V = f^{-1} (U).$
        
            \textcolor{magenta}{($\Leftarrow$)}
            Пусть $U \in \Omega_Y \Rightarrow \forall a \in f^{-1}(U) \; \exists V_{\varepsilon}(a) 
            \subseteq f^{-1}(U) \Rightarrow a$ --- внутренняя точка $f^{-1}(U) \Rightarrow f^{-1}(U) \in \Omega_X$
            
            \qed
            
            \item 
            \begin{theorem}
                Пусть $X, Y$ --- топологические пространства, $a \in X, f~{: X \rightarrow Y}$ --- отображение, $\Sigma_a$ --- база окрестностей в точке $a$ и $\Lambda_{f(a)}$ --- база окрестностей в точке $f(a)$. Тогда $f$ непрерывно в точке $a \in X \iff \forall U \in \Lambda_{f(a)} \; \exists V_a \in \Sigma_a : f(V_a) \subseteq U.$
            \end{theorem}
            
            \textit{Доказательство:} 
            
            \textcolor{magenta}{($\Rightarrow$)} $f$ непрерывно в точке $a \Rightarrow \left( \forall U \in \Lambda_{f(a)} \: \exists W_\varepsilon(a): f(W_\varepsilon(a)) \subseteq U) \right) \wedge \left( \exists V \in \Sigma_a: V \in W_\varepsilon(a)\right)$.
            
            \textcolor{magenta}{($\Leftarrow$)} $\left(\forall U_\varepsilon(f(a)) \exists U \in \Lambda_{f(a)}: U \subseteq U_\varepsilon(f(a))\right)\wedge\left(\exists V \in \Sigma_a : f(V) \subseteq U \subseteq U_\varepsilon(f(a))\right) $
            
            \qed
    \end{itemize}

\section{Билет} \

\section{Билет} \

\medskip

\textbf{Фундаментальные покрытия. Их применение для доказательства непрерывности функций. Фундаментальность открытых и конечных замкнутых покрытий.}\\

Покрытие $\Gamma = \{A_i\}_{i\in I}$ топологического пространства $X$ называется \textit{фундаментальным}, если 
\[
    \forall U\subseteq X : (\forall A_i \in \Gamma, U\cap A_i \text{ открыто в } A_i) \Rightarrow (U \text{ открыто в } X).
\] 

\begin{theorem}
    Пусть $X, Y$ - топологические пространства, $\Gamma = \{A_i\}_{i\in I}$ - фундаментальное покрытие $X$ и $f: X\rightarrow Y$ - отображение. Если $\forall A_i \in \Gamma$ сужение $\Gamma |_{A_i}$ непрерывно, то и само отображение $f$ непрерывно.
\end{theorem}

\begin{proof}
    Хотим показать, что прообраз любого $V$, открытого в $Y$, открыт в $X$. Открытое в $A_i$ $(f|_{A_i})^{-1}(V)=f^{-1}(V)\cap A$, так как $f|_{A_i}$ непрерывна. Тогда $f^{-1}(V)\cap A_i$ открыто в $A_i$. Пусть $U = f^{-1}(V)\in X$, и для любого $i$ мы тогда знаем, что $U\cap A_i$ открыто в $A_i$. Тогда из фундаментальности, $U$ открыто в $X$.
\end{proof}

Покрытие топологического пространства называется:
\begin{itemize}
    \item \textit{открытым}, если оно стоит из открытых множеств;
    \item \textit{замкнутым}, если оно состоит из замкнутых множеств;
    \item \textit{локально конечным}, если каждая точка пространства обладает окрестностью, пересекающейся лишь с конечным числом элементов покрытия.
\end{itemize}

\begin{theorem}
    Всякое открытое покрытие фундаментально.
\end{theorem}

\begin{proof}
    \textrm(by lounres.) Пусть дано покрытие $\Gamma$ и $U \subseteq X$, что для всякого $A \in \Gamma$ множество $U \cap A$ открыто в $A$, а значит открыто в $X$. Тогда
        \[
            U = U \cap X = \bigcup_{A \in \Gamma} U \cap A
        \]
    есть объединение открытых множеств, а значит само открыто. Таким образом $\Gamma$ фундаментально.
\end{proof}

\begin{theorem}
    Всякое конечное замкнутое покрытие фундаментально.
\end{theorem}

\begin{proof}
    \textrm(by lounres.) Пусть дано покрытие $\Gamma$ и $U \subseteq X$, что для всякого $A \in \Gamma$ множество $U \cap A$ замкнуто в $A$, а значит замкнуто в $X$. Тогда
        \[
            U = U \cap X = \bigcup_{A \in \Gamma} U \cap A
        \]
    есть конечное объединение замкнутых множеств, а значит само замкнуто. Таким образом $\Gamma$ фундаментально.
\end{proof}


\section{Билет} \

\textbf{Фундаментальность локально конечных замкнутых покрытий.}\\
    
    Покрытие называют \textit{\textcolor{black}{локально конечным}}, если каждая точка пространства обладает окрестностью, пересекающейся лишь с конечным числом элементов покрытия. 
    
    \textcolor{magenta}{Пример... :(}
    
    \begin{theorem}
        Всякое локально конечное замкнутое покрытие фундаментально.
    \end{theorem}
    \textit{Доказательство:}
    
    Пусть $\left\{A_i\right\}$ --- локально конечное замкнутое покрытие. Хотим проверить его фундаментальность;
        \begin{enumerate}
            \item Пусть $U$ --- произвольное множество, такое что $U \cap A_i$ --- открытое в $A_i$.
            \item В каждой точке $b$ пространства рассмотрим окрестность $U_b$, перескающуюся с конечным числом множеств покрытия(локальная конечность). Тогда $\left\{U_b\right\}$ --- открытое покрытие пространства $\Rightarrow$ оно фундаментально(13 билет). 
            \item Зафиксируем $b$, тогда $\left\{U_b \cap A_i\right\}$ --- конечное замкнутое покрытие $U_b \Rightarrow \left\{U_b \cap A_i\right\}$ --- фундаментальное покрытие $U_b$(13 билет).
        \end{enumerate}
        
            Покажем, что $\forall b \left((U \cap U_b) \cap (U_b \cap A_i)\right)$ --- открыто в $U_b \cap A_i \Rightarrow U \cap U_b$ --- открыто в $U_b$(фундаментальность из п.3) $\Rightarrow U$ --- открыто в пространстве(фундаментальность п.2).
            
            Действительно, $(U \cap U_b) \cap (U_b \cap A_i) = (U \cap A_i) \cap (U_b \cap A_i), U \cap A_i$ --- открытое в $A_i$(п.1) $\Rightarrow U \cap A_i = V \cap A_i$, где $V$ --- открытое во всём пространстве(определение открытого в подпространстве) $\Rightarrow(U \cap A_i) \cap (U_b \cap A_i) = (V \cap A_i) \cap (U_b \cap A_i) = V \cap (U_b \cap A_i)$ --- открытое в $U_b \cap A_i$
    
    \qed

\section{Билет} \

\section{Билет} \

\medskip

\textbf{Непрерывность и произведение: проекции, теорема о покоординатной непрерывности.}\\

$X = \Pi_{i\in I}X_i$ - произвеиение топологических пространств.\\

\begin{theorem}
    Координатные проекции $p_i: X\rightarrow X_i$ непрерывны.
\end{theorem}

\begin{proof}
    $\forall U$ открытого в $X_i$: $p_i^{-1}(U)$ - элемент предбазы Тихоновской топологии (по определению), следовательно открыт в $X$.
\end{proof}

\textbf{(Отображение в произведение двух)} Пусть $X$, $Y$, $Z$ – топологические пространства. Любое отображение $f: Z \to X \times Y$ имеет вид
    \[
        f(z) = (f_1(z), f_1(z)), \text{ для всех $z \in Z$},
    \]
где $f_1: Z \to X$, $f_2: Z \to Y$ - некоторые отображения, называемые \textit{компонентами} отображениями $f$. \\ 

\textbf{(Отображение в произведение дохуя)} Пусть $Z$ и $\{X_i\}_{i \in I}$ -  топологические пространства. \textit{Компонентами} отображения $f: Z \to \prod_{i \in I} X_i$ называются отображения $f_i: Z \to X_i$, задаваемые формулами
     \[
        f_i := p_i \circ f
    \]

\begin{theorem}
    (О покоординатной непрерывности). Пусть $Z$ и $\{X_i\}_{i\in I}$ - топологические пространства, $X = \Pi_{i\in I}X_i$ - тихоновское произведение. Тогда отображение $f: Z\rightarrow \Pi_{i\in I}X_i$ непрерывно, равносильно тому что каждая его компонента $f_i$ непрерывна.
\end{theorem}

\begin{proof} Докажем в обе стороны:\\

    \textcolor{magenta}{$\Rightarrow$} $f_i = p_i \circ f$, при этом $p_i$ и $f$ непрерывны, следовательно, и $f_i$ непрерывна.\\

    \textcolor{magenta}{$\Leftarrow$} Сначала для любого $U$ из предбазы $X$ существует такой индекс $i\in I$ и $V\in \Omega_i$ такой, что $U=p_i^{-1}(V)$. Тогда $f^{-1}(U)=f^{-1}(p_i^{-1}(V))=(p_i\circ f)^{-1}(U)= f_i^{-1}(V) $ - открытое, так как $f_i$ непрерывно. \\

    $\forall W$ открытого в $X$, $W = \bigcup $ (конечных пересечений эл-в предбазы) (далее - $\bigcup_{fuck}$)\\

    $f^{-1}(W)=f^{-1}(\bigcup_{fuck}) = \bigcup f^{-1}($конечных пересечений$)=\bigcup $ (конечных пересечений прообразов элементов предбазы)

\end{proof}

\textbf{Дополнительно от keba4ok:} Также для проверки на непрерывность $f: X \to Y$ достаточно проверить открытость $f^{-1}(U)$ для всякого $U$ из какой-либо базы или предбазы $Y$.

\section{Билет} \

\medskip

\textbf{Пример функции на плоскости, непрерывной по каждой координате, но разрывной.}\\
    
    Функция $: \mathbb R^2 \rightarrow \mathbb R$, заданная уравнением 
    \[
    f(x,y) = 
    \begin{cases} 
        \frac{2xy}{x^2+y^2}, &\text{если } (x,y) \neq (0,0); \\
        0, &\text{если } x = y = 0.
    \end{cases}
    \]
    Непрерывна по каждой координате, но разрывна в точке $(0,0)$.
    \begin{proof}
        Док-во непрерывности функции $f(x) = \frac{2cx}{x^2+c^2}$ полагаем, не представляет труда доказать студентам, получившим 5 за матанализ. А в точке $(0,0)$ функция разрывна, так как при $x = y \neq 0$ функция равна 1. 
    \end{proof}

\section{Билет} \

\medskip

\section{Билет} \

\medskip

\textbf{Гомеоморфизм. Гомеоморфные интервалы на прямой, $S^n \backslash \{p\}$ и $\mathbb{R}^n$.}\\

Пусть $X, Y$ - топологические пространства. Отображение $f: X \rightarrow Y $ называется \textit{гомеоморфизмом}, если 
\begin{itemize}
    \item $f$ - биекция;
    \item $f$ - непрерывно;
    \item $f^{-1}$ - непрерывно;
\end{itemize}

Говорят, что пространство $X$ \textit{гомеоморфно} пространству $Y$ ($X \simeq Y$), если существует гомеоморфизм $X \rightarrow Y$. \\

\textbf{Дополнительно от keba4ok:} Гомеоморфность - отношение эквивалентности среди топологических пространств. \\

\textcolor{magenta}{Примеры на прямой, плоскости и т.д....:(}



\section{Билет} \

\medskip

\textbf{Аксиомы счётности. Теорема Линделёфа.}\\
    
    \textbf{\textcolor{black}{Будем считать, что множество $X$ счётно, если существует инъекция $X \rightarrow \mathbb N$}} (всякое подмножество счётного - счётно).
    
    \begin{enumerate}
        \item Топологическое пространство удовлетворяет \textit{\textcolor{magenta}{первой аксиоме счётности}}, если оно обладает счётными базами во всех своих точках.
        \item Топологическое пространство удовлетворяет \textit{\textcolor{magenta}{второй аксиоме счётности}}, если оно имеет счётную базу.
        \item Топологическое свойство называется \textit{\textcolor{black}{наследственным}}, если из того, что пространство $X$ обладает этим свойством, следует, что любое его подпространство тоже им обладает(аналогично про наследование при произведении).
    \end{enumerate}
    \textbf{Свойства:} 
    \begin{enumerate}
        \item $2AC \Rightarrow 1AC$:
        
        см. 8 билет
        
        \item Обратное неверно:
        
        $X$ --- несчётное множество с дискретной метрикой. Тогда в каждой точке есть счётная база - сама точка, но при этом счётной базы всего пространства нет --- каждый элемент должен входить в базу.
        
        \item Всякое метрическое пространство удовлетворяет $1AC$:
        
        Шары вида $B_{\frac{1}{n}}(a)$, где $n \in \mathbb N$ --- база в точке $a$.
        
        \item $2AC$ наследственна(в обоих смыслах):
        \begin{itemize}
            \item Пересечём базу с подмножеством --- получится счётная база подпространства.
            \item Рассмотрим декартово произведение счётных баз --- получим счётную базу декартова произведения пространств. 
        \end{itemize}
    \end{enumerate}
    
    \begin{theorem}[Теорема Линделёфа]
        Если пространство удовлетворяет $2AC$, то из всякого его открытого покрытия можно выделить счётное подпокрытие.
    \end{theorem}
    
    \begin{proof}
        
        Пусть $\{U_i\}$ --- открытое покрытие $X$, a $\Sigma$ --- счётная база. Рассмотрим $\Lambda: = \{V \in \Sigma | \exists U_i: V \in U_i\}$. Заметим, что $\Lambda$ --- покрытие любого $U_i$(из определения базы) $\Rightarrow \Lambda$ --- покрытие $X$, тогда каждому $V \in \Lambda$ сопоставим произвольное $U_j$, в котором оно лежит. Тогда $\{U_j\}$ --- счётное покрытие X. 
    \end{proof}

\section{Билет} \

\medskip

\section{Билет} \

\medskip

\textbf{Аксиомы отделимости $T_1 - T_3$. Замкнутость диагонали в $X\times X$. Критерий регулярности.}\\

Говорят, что топологическое пространство удовлетворяет \textcolor{magenta}{\textit{первой аксиоме отделимости}} $T_1$, если каждая из любых двух различных точек пространства обладает окрестностью, не содержащей другую из этих точек.\\

Говорят, что топологическое пространство удовлетворяет \textcolor{magenta}{\textit{второй аксиоме отдельности}} $T_2$, если любые две различные точки пространства обладают непересекающимися окрестностями. Пространства, удовлетворяющий аксиоме $T_2$, называются \textit{хаусдорфовыми}.\\

\begin{theorem}
    (Замкнутость ёбаной диагонали). $X$ хаусдорфово равносильно тому, что $\{(a,a):a\in X\}$ замкнуто в $X\times X$.
\end{theorem}

\begin{proof}
    (by lounres) \\

    \textcolor{magenta}{$\Rightarrow$} Покажем, что $(X \times X) \setminus \Delta$ открыто. Пусть $(b, c) \notin \Delta$. Тогда по $T_2$ есть окрестности $U_b$ и $U_c$ точек $b$ и $c$ в $X$, что $U_b \cap U_c = \varnothing$. Следовательно $(U_b \times U_c) \cap \Delta = \varnothing$, тогда $U_b \times U_c$ --- окрестность $(b, c)$, лежащая в $(X \times X) \setminus \Delta$ как подмножество. \\

    \textcolor{magenta}{$\Leftarrow$} Пусть $b$ и $c$ - различные точки $X$. Тогда $(b, c) \notin \Delta$. Поскольку $\Delta$ замкнуто, то $(X \times X) \setminus \Delta$ открыто. Поскольку $\{U \times V \mid U, V \in \Omega_X\}$ --- база $X \times X$, то есть некоторые открытые в $X$ множества $U$ и $V$, что
        \[
            (b, c) \in U \times V \subseteq (X \times X) \setminus \Delta.
        \]
            
        Следовательно, $(U \times V) \cap \Delta = \varnothing$, а значит, $U \cap V = \varnothing$. При этом $b \in U$, а $c \in V$. Значит $U$ и $V$ --- непересекающиеся окрестности $b$ и $c$. Поскольку $b$ и $c$ случайны, то выполнена $T_2$. 
\end{proof}

Говорят, что топологическое пространство удовлетворяет \textcolor{magenta}{\textit{третьей аксиоме отделимости}} $T_3$, если в нём любое замкнутое множество илюбая не содержащаяся в этом множестве точка обладают непересекающимися окрестностями. Пространства, одновременно удовлетворяющие аксиомам $T_1$ и $T_3$, называются \textit{регулярными}.\\

\begin{theorem}
    (Критерий блядской регулярности). $X$ регулярно тогда и только тогда, когда удовлетворяет $T_1$ и $\forall a\in X $ любой окрестности $U_a$ существует окрестность $V_a$ такая, что $\text{Cl}V_a\subseteq U_a$.
\end{theorem}

\begin{proof}

    (by lounres)\\

    \textcolor{magenta}{$\Rightarrow$} Пусть $U_a$ --- некоторая окрестность некоторой точки $a$ в $X$. Тогда $X \setminus U_a$ замкнуто. По $T_3$ у $X \setminus U_a$ и $a$ есть непересекающиеся окрестности $W_a$ и $V_a$ соответственно. Тогда $X \setminus W_a$ замкнуто; при этом $W_a \supseteq X \setminus U_a$, следовательно $X \setminus W_a \subseteq U_a$; аналогично имеем, что $V_a \subseteq X \setminus W_a$. Следовательно
        \[
            \text{Cl}(V_a) \subseteq X \setminus W_a \subseteq U_a.
        \]
    
    Таким образом мы нашли искомую окрестность $V_a$.\\

    \textcolor{magenta}{$\Leftarrow$} Пусть даны замкнутое $F$ и точка $a$ вне него. Тогда $U_a := X \setminus F$ --- окрестность $a$. Тогда есть окрестность $V_a$ точки $a$, что $\text{Cl}(V_a) \subseteq U_a$. Следовательно $\text{Int}(X \setminus V_a) \supseteq X \setminus U_a = F$. Значит $\text{Int}(X \setminus V_a)$ и $V_a$ --- непересекающиеся окрестности $F$ и $a$.
\end{proof}


\section{Билет} \

\medskip

\textbf{Аксиома отделимости $T_4$. Нормальномть метрических пространств.}\\
        
    Говорят, что топологическое пространство удовлетворяет \textit{\textcolor{magenta}{четвёртой аксиоме отделимости}}, если в нём любые два непересекающихся замкнутых множества обладают непересекающимися окрестностями. \\
    
    Пространство, удовлетворяющее аксиомам $T_1$ и $T_4$, назыывается \textit{\textcolor{black}{нормальным}}.\\
    
    \begin{theorem}
        Всякое метрическое пространство нормально.
    \end{theorem}
    
    \begin{proof}
        Пусть $(X,d)$ --- метрическое пространство. В нём выполняется $T_1(r = \frac{d(x,y)}{2})$. Покажем $T_4$ --- пусть $A,B$ --- непересекающиеся замкнутые множества. $X \setminus B$ --- открытое и содержит $A \Rightarrow \forall a \in A \exists r_a: B_{r_a}(a) \subseteq X \setminus B \iff B_{r_a}(a) \cap B = \varnothing$. Аналогично для каждой точки $b$ из $B$ находим окрестность $B_{r_b}(b)$, не пересекающуюся с $A$. Рассмотрим $U = \bigcup B_{\frac{r_a}{2}}(a)$  и $V = \bigcup B_{\frac{r_b}2}(b)$. Допустим $z \in (U \cap V)$(иначе мы нашли две непересекающиеся окрестности, т.к. объединение открытых - открыто). $z \in (U \cap V) \Rightarrow \exists x \in A, y \in B: z \in (B_{\frac{r_x}2}(x) \cap B_{\frac{r_y}2}(y)) \Rightarrow d(x,y) \leq \frac{r_x}2 + \frac{r_y}2 \leq max(r_x,r_y) \Rightarrow (x\in B_{r_y}y) or (y \in B_{r_x}(x))$--- противоречие. 
    \end{proof}
    
    \textbf{Дополнительно от artemi.sav:}
    
    $X$ --- нормально $\Rightarrow X$ --- регулярно $\Rightarrow X$ --- хаусдорфо $\Rightarrow X$ удовлетворяет $T_1$.
    
    Заметим, что из $T_1$ следует, что любая точка - замкнута, из чего следует, что каждое следующее условие ---  частный случай предыдущего (например, $X$ --- нормально, то есть удовлетворяет $T_4$ и $T_1$, вместо одного замкнутого множества в условии $T_4$ можем взять точку и получить $T_3$, то есть регулярность).


\section{Билет} \

\medskip

\section{Билет} \

\medskip

\textbf{Непрерывный образ связного пространства. Теорема о промежуточном значении.}\\

\begin{theorem}
    (Непрерывный обрах связного пространства связен). Если $f: X\rightarrow Y$ - непрерывное отображение и пространство $X$ связно, то и множество $f(X)$ связно.
\end{theorem}

\begin{proof}
    От противного, пусть $f(X)$ несвязно. Тогда $f(X)=U\cup V$, $U\cap V = \emptyset$, где $U, V$ непусты и открыты.\\

    Следовательно, мы имеем разбиение пространства $X$ на два непустых открытых множества - $f^{-1}(U)$ и $f^{-1}(V)$, что противоречит связности пространства $X$.
\end{proof}

\begin{theorem}
    (О промежуточном значении). Если $f: X\rightarrow \mathbb{R}$ - непрерывное отображение, и пространство $X$ связно, тогда для любых $a, b\in f(X)$ множество $f(X)$ содержит все числа между $a$ и $b$.
\end{theorem}

\begin{proof}
    $f(X)$ связно $\Rightarrow$ $f(X)$ выпукло $\Rightarrow$ $f(x)$ содержит $[a,b]$.
\end{proof}

\section{Билет} \

\medskip

\textbf{Компоненты связности. Разбиение пространства на компоненты связности.}\\
    
    \textit{\textcolor{magenta}{Компонентой связности}} пространства $X$ называется всякое его максимальное по включению связное подмножество.\\
    
    \textbf{Лемма.} \textit{Объединение любого семейства попарно пересекающихся связных множеств связно.}
    
    \begin{proof}
        Обозначим это семейство множеств за $\{A_i\}, Y:=\bigcup A_i$. Допустим $Y$ --- несвязно, тогда $Y = U \cup V$, где $U$ и $V$ --- открытые непересекающиеся множества. Заметим, что $\forall A_i: A_i \subseteq U \vee A_i \subseteq V$(иначе $A_i = (A_i\cap U)\cup(A_i \cap V)$, где $A_i \cap U$ и $A_i \cap V$ --- непустые открыте подмножества в $A_i$). Зафиксируем $A_0$, НУО $A_0 \subseteq V \Rightarrow \forall A_i (A_i \cap A_0 \neq \varnothing) \Rightarrow \forall A_i \subseteq V \Rightarrow U = \varnothing$, противоречие.
    \end{proof}
    
    \begin{theorem}
        \begin{enumerate}
            \item Каждая точка пространства $X$ содержится в некоторой компоненте связности.
            \item Различные компоненты связности пространства $X$ не пересекаются.
        \end{enumerate}
    \end{theorem}
    
    \begin{proof} \textcolor{white}{-}\\
    \begin{enumerate}
        \item Пусть $x \in X$. Тогда множество $A$ --- объединение всех связных множеств, содержащих $x$, является искомой компонентой связности(оно связно по лемме и наибольшее по включению по своему определению).
        \item Пусть $U,V$ --- пересекающиеся компоненты связности, тогда $U \cup V$ --- связное множество(по лемме), содержащее $U$ и $V$, что противоречит определению компоненты связности.
    \end{enumerate}
    \end{proof}



\newpage

\hypertarget{t2}{Table of objects}



\section{Пофамильный указатель всего на свете}

\begin{multicols}{2}
    [
    Быстрый список для особо заебавшегося поиска.
    ]
    
    \hyperlink{n1}{метрика}

    \end{multicols}


\end{document}