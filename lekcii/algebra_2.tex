\documentclass[a4paper,100pt]{article}

\usepackage[utf8]{inputenc}
\usepackage[unicode, pdftex]{hyperref}
\usepackage{cmap}
\usepackage{mathtext}
\usepackage{multicol}
\setlength{\columnsep}{1cm}
\usepackage[T2A]{fontenc}
\usepackage[english,russian]{babel}
\usepackage{amsmath,amsfonts,amssymb,amsthm,mathtools}
\usepackage{icomma}
\usepackage{euscript}
\usepackage{mathrsfs}
\usepackage{geometry}
\usepackage[usenames]{color}
\hypersetup{
     colorlinks=true,
     linkcolor=magenta,
     filecolor=red,
     citecolor=black,      
     urlcolor=cyan,
     }
\usepackage{fancyhdr}
\pagestyle{fancy} 
\fancyhead{} 
\fancyhead[LE,RO]{\thepage} 
\fancyhead[CO]{\hyperlink{t2}{к списку объектов}}
\fancyhead[LO]{\hyperlink{t1}{к содержанию}} 
\fancyhead[CE]{текст-центр-четные} 
\fancyfoot{}
\newtheoremstyle{indented}{0 pt}{0 pt}{\itshape}{}{\bfseries}{. }{0 em}{ }

%\geometry{verbose,a4paper,tmargin=2cm,bmargin=2cm,lmargin=2.5cm,rmargin=1.5cm}

\title{Алгебра. Конспект 2 сем.}
\author{Мастера Конспектов\\ \\ (по материалам лекций В. А. Петрова,\\ а также других источников)}
\date{12 февраля 2021 г.}

\theoremstyle{indented}
\newtheorem{theorem}{Теорема}
\newtheorem{lemma}{Лемма}

\theoremstyle{definition} 
\newtheorem{defn}{Определение}
\newtheorem{exl}{Пример(ы)}

\theoremstyle{remark} 
\newtheorem{remark}{Примечание}
\newtheorem{cons}{Следствие}
\newtheorem{stat}{Утверждение}

\DeclareMathOperator{\Ker}{Ker}
\DeclareMathOperator{\Tors}{Tors}
\DeclareMathOperator{\Frac}{Frac}
\DeclareMathOperator{\Imf}{Im}
\DeclareMathOperator{\End}{End}
\DeclareMathOperator{\cont}{cont}
\DeclareMathOperator{\id}{id}
\DeclareMathOperator{\ev}{ev}
\DeclareMathOperator{\lcm}{lcm}
\DeclareMathOperator{\chard}{char}
\DeclareMathOperator{\CC}{\mathbb{C}}
\DeclareMathOperator{\ZZ}{\mathbb{Z}}
\DeclareMathOperator{\RR}{\mathbb{R}}
\DeclareMathOperator{\NN}{\mathbb{N}}
\DeclareMathOperator{\codim}{codim}
\DeclareMathOperator{\rank}{rank}
\DeclareMathOperator{\ord}{ord}
\DeclareMathOperator{\adj}{adj}
\DeclareMathOperator{\Ann}{Ann}

\begin{document}

\newcommand{\resetexlcounters}{%
  \setcounter{exl}{0}%
} 

\newcommand{\resetremarkcounters}{%
  \setcounter{remark}{0}%
} 

\newcommand{\reseconscounters}{%
  \setcounter{cons}{0}%
} 

\newcommand{\resetall}{%
    \resetexlcounters
    \resetremarkcounters
    \reseconscounters%
}

\maketitle 

\newpage

\hypertarget{t1}{Некоторые} записи по алгебре.
\tableofcontents

\newpage


\section{Лекция 30.}

Пусть $R$ - кольцо главных идеалов, а $M$ - конечно порождённый $R$-модуль (левый). 

\[
    m_1, \ldots, m_n\in M, \:\:M=\{\sum r_im_i \vert r_i\in R\}
\]

Пусть $\varphi: R^n\rightarrow M$ - функция, которая действует по правилу $e_i\mapsto m_i$ (базисные элементы $R^n$ (именно тривиального базиса) в элементы $m_i$). \

Тогдя ядро $\Ker \varphi \leq R^n$ - подмодуль. Причём равен он $\{(r_i)\vert \sum r_im_i=0\}$ - \textit{соотношения} (линейные) между $m_i$. А также он есть \textit{свободный} модуль $R^k, \: k\leq n$. 

\[
    \Ker \varphi =R^k, \:\: R^k\leq R^n 
\]
\[
    \psi: R^k\rightarrow R^n
\]

Подходящей заменой базиса в $R^k$ и $R^n$ можно добиться того, чтобы $\psi$ стала диагональной матрицей (с нижними нулевыми строками, естественно) и числами $d_1\vert d_2\vert \ldots\vert d_k$ на диагонали.

Тогда $M\cong R^{n-k}\oplus R/(d_i)\oplus\ldots \oplus R/(d_k)$ (это планируется доказывать, но перед этим нужно ввести несколько определений).\ 

\begin{defn} 
    Пусть $R$ кольцо (не обязательно коммутативное), тогда $M$ - \textit{циклический}, если он порождён одним элементом ($M=\{rm\vert r\in R\}$).
\end{defn}

Пусть $\theta: R\rightarrow M$ - гомоморфизм $R$-модулей, действующий по правилу $r\mapsto rm$, он сюръективен и $M\simeq R/\Ker \theta$ по теореме о гоморфизме.

\[
    \Ker\theta = \{r\in R\vert rm=0\} \leq R, 
\]
что также является левым идеалом.\ 

А если $R$ - область главных идеалов, то циклический модуль выглядит как $R/(d)$. Если $d = 0$, то $R$ - свободный модуль ранга $1$, а если он не равен нулю, то это есть \textit{модуль кручения} $\forall x\in M \: dx=0$.\\

\begin{theorem}
    Конечнопорождённый модуль над областью главных идеалов - конечная прямая сумма циклических модулей.
\end{theorem}\

Была доказана в прошлом семестре (не у нас). Однаком мы можем сформулировать следствие:

\begin{cons}
    Конечнопорождённая абелева группа - конечная прямая сумма циклических групп.
\end{cons}

Пусть $R$ - область, $M$ - $R$-модуль, тогда подмодуль кручения - 

\[
    \Tors (M) = \{m\in M\vert \exists r\neq 0, \: rm=0\}
\]

\begin{stat}
    $\Tors (M)$ - модмодуль в $M$.
\end{stat}

Нужно выполнить проверку этого утверждения, но для этого достаточно проверить, что всё хорошо с нулём (он там лежит и $1\cdot 0 = 0$), а затем несколько свойств:

\[
    m_1, m_2\in \Tors (M), \: r_1, r_2 \neq 0, \: r_1m_1=r_2m_2=0, 
\]
тогда
\[
    r_1r_2(m_1+m_2)=0, \: r_1r_2\neq 0, 
\]
а также, если
\[
    m\in \Tors (M), \: s\in R, \: rm=0 \Rightarrow r(sm)=rsm=s(rm)=0.
\]

Пусть $r\in R$, $r\neq 0$, $M[r]:=\{m\in M:\: rm=0\}\leq M$ - подмодуль, $p$ - пргстой элемент $R$. Рассмотрим $M[p]\leq M[p^2]\leq M[p^3]\leq \ldots$ - получили цепочку вложенных модулей.\ 

$M_p:=\bigcup_{i\geq 1}M[p^i]$ - подмодуль, $p$-кручение в $M$. \\

Сейчас начнётся пиздец. Наша цель: доказать, что $\Tors (M)\cong \bigoplus_{p-\text{простое}} M_p$.\ 

$N_i$ - модули $i\in I$, $\bigoplus :=\{(n_i)_{i\in I}\vert n_i\in N_i, \text{ почти все }n_i=0\}$, операции покомпонентные. Это, получается, (бесконечная) прямая сумма модулей.\\

\begin{theorem}
    (О примарном разложении). Пусть $R$ - область главных идеалов, $M$ - $R$-модуль. Тогда $\bigoplus M_p \rightarrow \Tors (M)$, дествующий по правилу $(m_p)\mapsto \sum m_p$ (конечная сумма) - изоморфизм модулей.
\end{theorem}

\begin{proof}
    Докажем всё по порядку:\

    \begin{itemize}
        \item Докажем, что это гомоморфизм. $(m_p+n_p)\mapsto \sum m_p+n_p=\sum m_p+\sum n_p$, а также $(rm_p)\mapsto \sum rm_p=r(\sum m_p)$. 
        \item Теперь нужно доказать сюръективность. $m\in \Tors (m)$, $rm=0$, $r=\Pi_{i=1}^np_i^{\alpha_i}$, где $p_i$ - простое. Рассмотрим линейное разложение $\text{НОД}$: 
        \[
            r_1p_2^{\alpha_2}\ldots p_n^{\alpha_n}+\ldots+r_np_1^{\alpha_1}\ldots p_{n-1}^{\alpha_{n-1}}=1.
        \]
        Тогда если мы домножим равенство на $m$, получим, что $r_i=\frac{rm}{p_i^{\alpha_i}}\in M_{p_i}$, тогда получили, что $(r_1p_2^{\alpha_2} \ldots  p_n^{\alpha_n}m, \ldots, r_np_1^{\alpha_1}\ldots p_{n-1}^{\alpha_{n-1}}m)\mapsto m$.
        \item Осталась инъективность. Пусть $0\neq(m_p)\mapsto 0$, возьмём наименьшее число индексов, что $\sum m_p=0$. А теперь начнём его уменьшать. Пусть у нас есть $p_1, \ldots , p_n$, $p_i^{\alpha_i}m_{p_i}=0$. Всё домножим на $p_n^{\alpha_n}$, получим $\sum p_n^{\alpha_n}m_p=0$. Тогда раньше было $m_{p_n}\neq 0$, а теперь $p_n^{\alpha_n}m_{p_m}=0$. Докажем, что ничего, кроме последнего не обнулилось. Предположим противное, $p_1^{\alpha_1}m_1=0$, $p_n^{\alpha_n}m_1=0$, но $p_1^{\alpha_1}, \: p_n^{\alpha_n}$ - взаимно просты, тогда есть линейное разложение $r_1p_1^{\alpha_1}+ r_n p_n^{\alpha_n}=1$, домножим на $m$, получим $r_1p_1^{\alpha_1}m_1+ r_n p_n^{\alpha_n}m_1=m_1$, но оба они не могут быть равны нулю.
    \end{itemize}
\end{proof}

Сейчас будем заниматься в основном кольцом многочленов. Пусть $R=F[t]$, $F$ - поле, $V$ - $R$-модуль. В частности, $V$ - $F$-модуль, то векторное пространство $A:v\rightarrow tv$ - $F$-линейное отображение $V\rightarrow V$ \textit{оператор}. Линейные операторы образуют кольцо (сумма - поточечно, умножение - композиция). $A(v)$ или $Av$.
\[
    (a_0+a_1t+\ldots+a_nt^n)V=a_0v+a_1Av+\ldots+a_nA^nv
\]
$V$ - векторное порстранство с оператором, значит, $F[t]$ - модуль.\ 

Пусть $a$ - матрица $n\times n$ $F^n\rightarrow F^n$, $F[t]$ - модуль на $F^n$. $F[t]$ - как модуль над собой векторное пространство со счётным базисом.\ 


\begin{stat}
    Пусть $V$ возьмём конечнопорождённый модуль над $F[t]$, тогда $V$ - конечномерное векторное пространство над $F$ тогда и только тогда, когда $V=\Tors (V)$ (как $F[t]$-модуль).
\end{stat}

\begin{proof}
    $F[t]^n\oplus F[t]/(f_i)\oplus\ldots\oplus F[t]/(f_k)$, где $f_i\neq 0$. Если $n\neq 0$, то в $V$ есть бесконечномерное подпространство $F[t]$. Если $n=0$, то $\dim _F F[t]/(f_i)=\deg f_i < \infty$.
\end{proof}

Теперь рассмотрим матрицы. Пусть $\dim V =n $, $A: V\rightarrow V$. Если зафиксировать базис в $V$, получается матрица $a$ $n\times n$. Взали другой базис, получим матрицу перехода $c$. $V\rightarrow V$ посредством $A$, причём стороны соответственно изоморфны вот таким вещам (по центру, я не умею так круто чертить, загляните в лекцию) $F^n\xrightarrow{c^{-1}}F^n\xrightarrow{a}F^n\xrightarrow{c}F^n$. И, кстати, $a\sim c^{-1}ac$ (сопряжённая матрица).\\

Рассмотрим модуль $F[t]/(f)$, что также есть $V$, $A$. Поймём, что такое $f$. Он обладает таким свойством: $(f)=\Ker(F[t\rightarrow F[t/(f)]])=\{g(t)\vert \: g(t)\cdot v=0\: \forall v\in V\}$. Однако последнее равенство неочевидно. По определению там может быть написано $\{g(t)\vert \: g(t)\cdot [1]=0\}$, но $[h(t)]=h(t)\cdot 1$, поэтому он обнуляется $g(t)$: $g(t)\cdot[h(t)]=h(t)\cdot g(t)\cdot [1]=0$, откуда и получаем искомое. \ 

Давайте теперь запишем это в терминах оператора. Если 

\[
    g(t)=a_0+a_1t+\ldots+a_kt^k, 
\]

тогда 

\[
    g(t)\cdot v = a_0v+a_1Av+\ldots+a_kA^kv.
\]

Каждый раз писать такие длинные вещи неудобно, поэтому введём следующее обозначение: 

\[
    g(A):= a_0v+a_1A+\ldots+a_kA^k.
\]

В силу того, что $A$ коммутирует с собой, то такая запись корректна. Тогда мы можем переписать:

\[
    \{g(t)\vert \: g(t)\cdot v=0\: \forall v\in V\}=\{g(t)\vert \: g(A)v=0\: \forall v\in V\},
\]

но если последнее выполнено для любого $v\in V$, то получаем, что оператор - тождественный нуль, получаем $\{g(t)\vert \: g(A) = 0\}$. \ 

Также можно пойти и в обратныую сторону, то есть, пусть мы знаем $A$, рассмотрим $\{g(t)\vert \: g(A)=0\}$. Это - идеал в $F[t]$, скажем, что это $(f(t))$, тогда $f(t)$ мы будем называть \textit{минимальным многочленом} оператора $A$. Можно заметить, что минимальный многочлен не равен нулю, если у нас имеется конечномерное пространство, не может быть такого, что никакой многочлен $A$ не обнуляет. Покажем это. \ 

Найдём некую линейную зависимость между степенями $A$. Рассмотрим $\text{Id}, A, A^2, \ldots$ - элементы кольца операторов. Рассмотрим это кольцо как векторное пространство над $F$. Если $\dim V = т$, то у полученного пространства размерность есть $n^2$, то есть, конечна. Потому бесконечной линейно независимой системы быть не может, тогда когда-то мы получим линейную зависимость:

\[
    a_0+a_1 A+\ldots+a_k A^k=0, 
\]
тогда отсюда мы и нашли требуемый многочлен.

\section{Лекция 31.}

Начинаем опять с оператора. Рассматриваем векторной пространство $V$ над каким-то полем $F$ и мы действуем на него оператором $A:V\rightarrow V$. Мы его также рассматривали как $F[t]$-модуль, $t\cdot v=Av$. Мы определили минимальный многочлен $A$ такой, что $\{g(t)\in F[t]\vert g(a)=0\}\triangleleft F[t]$, причём $F[t]=(f(t))$ - идеал унитарного (нуо) многочлена. Такой $f(t)$ и называется минимальным многочленом. \
    
Теперь немного понятнее на языке модулей. Рассмотрим $V$ - $F[t]$-модуль, а также $\Ann(V):=\{r\in V\vert rv=0, \: \forall v\in V\}$. Это - идеал в $R$, причём даже двусторонний (можно будет потом записать проверку). Причём получаем, что $\Ann(V)=(f(t))$, легко заметить, что они совпадают.\ 

$g(A)v=0$, но тогда 

\[
    g=a_0+a_1t+\ldots+a_kt^k
\]
\[
    g=a_0+a_1Av+\ldots+a_kA^tv=0,
\]
что также и равно $g(t)\cdot v$. Тогда $f(A)v=g(t)\cdot v$ как оператор и из структуры модуля соответственно. Тогда $g(A)=0$ $\Leftrightarrow$ $g(A)\cdot v=0$ для любого $v\in V$ $\Leftrightarrow$ $g(t)\cdot v=0$ $\forall v\in V$ $\Leftrightarrow$ $g(t)\in \Ann (v)$.\ 

Мы уже начинали рассматривать такой модуль: $F[t]/(f(t))$ - $F[t]$-модуль, имеем также $V$, $Av=t\cdot v$. Мы хотим придумать базис $V$, в которм матрица $A$ имеет простой вид. Возьмём такой базис: $[1], [t], \ldots, [t^{k-1}]$, тогда $[t^k]=-a_0[1]-\ldots-a_{k-1}[t^{k-1}]$. Как выглядит матрица $A$ в этом базисе?

\begin{equation*}
    \begin{pmatrix}
        0 & 0 & \dots & 0 & -a_0\\
        1 & 0& \dots & 0 & -a_1\\
        \vdots & \vdots &\ddots & \vdots  & \vdots  \\
        \vdots & \vdots & \dots & 0 & -a_{k-2}\\
        0 & 0 & \dots & 1 & -a_{k-1}
    \end{pmatrix}
\end{equation*}

Такая матрица называется \textit{фробениусовой клеткой}. А вообще, в итоге мы получили, что если $V$ - циклический $F[t]$-модуль, то $A$ в некотором базисе записывается фробениусовой клеткой, причём последним столбцом будут коэффициенты минимального многочлена, только со знаком ''минус''.\ 

А если модуль не циклический (произвольный и с конечномерным $V$), то мы можем его разложить в прямую сумма циклических: 

\[
    F[t]/(f_1(t))\oplus F[t]/(f_2(t))\oplus\ldots\oplus F[t]/(f_m(t)), 
\]
причём мы можем даже потребовать, чтобы $f_1\vert f_2\vert\ldots\vert f_n$. \ 

Умножение на $t$ будет действовать поккординатно. \ 

Для каждого слагаемого мы умеем выписывать матрицу оператора $A$ в подходящем базисе. Матрица $A$ тогда выглядит на всём пространстве как цепочка фробениусовых клеток, расставленных по порядку по диагонали.\ 

Зададимся теперь вопросом: чему же в таком случае равен минимальный многочлен? Ответ таков:
\[
    A=f_m(t),
\]
причём принципиально условие цепочки делений. \

Как считать инвариантные факторы (то есть, $f_1(t), \ldots,f_n(t)$)? Рассмотрим $V$ и $F[t]$. $e_1, \ldots, e_n$ - базис $V$ как векторное пространство над $F$, а тем более, это система образующих $V$ как $F[t]$-модуля. Какими соотношениями обладает этот набор? $t\cdot e_i=Ae_i$ - линейная комбинация $e_1, \ldots, e_n$. Это соотношение между $e_i$ с коэффициентами из $f(t)$, получаем $(t\cdot I-A)e_i=0$.\ 

Мы имеем $n$ образующих и $n$ таких последних соотношений. Рассмотрим матрицу $(t\cdot I-A)$, она имеет размер $n\times n$ над $F[t]$ и выглядит так:

\begin{equation*}
    \begin{pmatrix}
        t-a_{11} & -a_{12} & \dots & -a_{1n}\\
        -a_{21} & t-a_{22} & \dots & -a_{2n}\\
        \vdots & \vdots &\ddots   & \vdots  \\
        -a_{n1} & -a_{n2} & \dots & -a_{nn}
    \end{pmatrix}
\end{equation*}

Домножим её слева и справа на обратимые над $F[t]$ матрица и приведём её к диагональному виду, а на диагонали будут расставлены $f_1, \ldots, f_m$ (перед которыми $n-m$ единиц). Последний многочлен будет минимальным многочленом $A$. \ 

Сравним определители этих матриц. Определитель обратимой матрицы лежит в $F[t]^*=F^*$. Идеал, порождённый в $F[t]$ определителем, не поменяется, тогда 

\[
    (\det (t\cdot I-A))=(f_1(t)\ldots f_n(t)), 
\]
тогда $\det (t\cdot I-A))\in F[t]$ мы будем называть \textit{характеристическим многочленом} матрицы $A$ (обозначаем $\chi_A(t)$). Имеет он степень $n$, причём он ещё и унитарный в силу того, что максимальная степень будет содержаться в $(t-a_{11})(t-a_{22})\ldots(t-a_{nn})$.\ 

Причём тогда мы можем получить такое равенство из того, что и характеристический многочлен, и призведение $f_i$ унитарно: 

\[
    \chi_a(t)=f_1(t)\cdot\ldots\cdot f_n(t),
\]

откуда минимальный многочлен делит характеристический многочлен, а характеристический делит минимальный в степени $n$. \ 

Наборы неприводимых делителей у минимаьлного и характеристического многочленов совпадают. В частности, наборы корней без учёта кратности совпадают.\\

\begin{theorem}
    (Теорема Гамильтона-Кэли). Минимальный многочлен делит характеристичесий, имеет такие же корни [и у них совпадают неприводимые делители].
\end{theorem}\

Приступим теперь к рассмотрению \textit{нильпотентным} операторам.

\begin{defn}
    $A:V\rightarrow V$ - \textit{нильпотентный}, если $A^k=0$ для некоторого $k$.
\end{defn}

Нужно теперь научиться понимать, когда это выполнено. Берём $k:A^k=0, A^{k-1}\neq 0$ (наименьшее возможное?). Минимальный многочлен у $A$ - $t^k$, потому что он подходит, и никакой его делитель не подходит. Какой же характеристический многочлен у $A$? Это есть $t^n$, где $n=\dim V$ из теоремы Гамильтона-Кэли. \ 

Пусть $A^k=0$ - минимальная такая степень. Рассмотрим $V$ как $F[t]$-модуль.
\[
F[t]/(t^{k_1})\oplus F[t]/(t^{k_2})\oplus\ldots\oplus F[t]/(t^{k_m}), \: k_1\leq k_2\leq\ldots\leq k_m=k,
\]
а само $k$ мы называем \textit{степенью нильпотентности}. Кстати, фробениусова клетка нильпотентного оператора теперь выглядит ещё лучше, весь правый столбец теперь состоит из нулей (в подходящем базисе). В общем случае, она составлена из квадратиков такого вида. Получили мы матрицу строгонижнетреугольного вида.

\begin{defn}
    \textit{Нижнетреугольная матрица} - всё, выше главной диагонали - нули. \textit{Строгонижнетреугольная матрица} - ещё и диагональ - нули.
\end{defn}

Как найти такой базис (без формы Смита)? Запишем по индукции:

\begin{align*}
    V[t] & =\{v\in V \vert tv=0\}=\Ker (A),  \\ 
    V[t^2] & =\{v\in V \vert t^2v=0\}=\Ker (A^2),\\
    &\dots \\
    V[t^{k-1}] & =\Ker (A^{k-1}),\\
    V[t^k] & =\Ker (A^k) = V.
\end{align*}

Рассмотрим цепочку вложенных пространств:
\[
    0<\Ker(A)\leq \Ker(A^2)\leq\ldots\leq\Ker(A^{k-1})<V.
\]

Посмотрим на образ $A$ (то есть, $\Imf A$), он попадёт в $\Ker (A^{k-1})$, а вот $A(\Ker(A^{k-2}))\leq\Ker(A^{k-2})$. \ 

Осталось найти тот самый базис, в котором матрица $A$ имеет нужный вид. Рассмотрим фактор-пространство $V/\Ker(A^{k-1})$, и выберем в нём базис. Это даёт нам относительный базис $V$ относительно $\Ker (A^{k-1})$ (скажем, это $e_1, \ldots, e_s$). Тогда что с ними происходит: $e_1. Ae_1, \ldots, A^{k-1}e_1$, причём получается, что все они не равны нулю, так как они не лежат в классе нуля. \ 

Рассмотрим $\langle e_1. Ae_1, \ldots, A^{k-1}e_1\rangle$ - $A$ переводит его в себя. Рассмотрим матрицу $A$ в данном базисе, это как раз будет фробениусова клетка размера $k$. Так проделаем для каждого элемента базиса и получим $s$ фробениусовых клеток размера $k$, где $s$ также было размерностью отфакторизованного пространства, тогда $s=\dim V-\dim \Ker (A^{k-1})$. \ 

Теперь рассмотрим $\Ker(A^{k-1})/(\Ker A^{k-2}+\Imf A)$ -  подпространство, порождённое $\Ker A^{k-2}$ и $\Imf A$. Возьмём относительный базис $e_{1, 1}, \ldots, e_{s_1, 1}$, опять перейдём к $\langle e_{1, 1}. Ae_{1, 1}, \ldots, A^{k-1}e_{1, 1}\rangle$ - тут $A$ имеет матрицу в виде фробениусовой клетки размера $k-1$ (если фробениусовых клеток такого размера нет, это пространство равно нулю). $s_1$ - количество таких клеток. \ 

И, наконец, клетки размера $k-i$: $\Ker(A^{k-i})/(\Ker(A^{k-i-1})+\Imf A^i)$, рассмотрим тут базис и проделаем аналогичные операции. \ 


\section{Лекция 32.}


\begin{remark}
    В предыдущей лекции была допущены небольшая ошибка, в месте, где записано $\Ker A^i/(\Ker A^{i-1}+\Imf A^{n-i})$, нужно записать $\Ker A^i/(\Ker A^{i-1}+(\Imf A\cap \Ker A^i))$.
\end{remark}

Допустим, у нас есть два разных поля: пусть раньше мы рассуждали над полем $K$, а сейчас есть ещё $L\geq K$. Над $K$ было векторное пространство $V$ с базисом $e_1, \ldots, e_n$. Мы можем рассмотреть такое же пространство над $L$, размерности тоже $n$. Рассмотрим $V_L$ - пространство, натянутое на $e_1, \ldots, e_n$ над $L$, то есть, все линейные комбинации вида $\{\alpha_1e_1+\ldots+\alpha_ne_n\}$. То есть, $\dim V = \dim V_L = n$. Тогда понятно, если у нас есть оператор $A: V\rightarrow V$, то мы можем его продолжить до оператора $A_L:V_L\rightarrow V_L$. \ 

Представить себе это можно по-разному. Представим себе матрицу изначального оператора в этом базисе, это какая-то матрица $M_n(K)\subseteq M_n(L)$ - можем ''расширить'', и получим, что первое - подкольцо второго. И тогда можно написать оператор с точно такой же матрицей на $L$. Можно также сказать, что мы рассматриваем $A(\alpha_1e_1+\ldots+\alpha_ne_n)$, тогда раскроем по линейности $\alpha_1A(e_1)+\ldots+\alpha_nA(e_n)$, и посчитаем необходимые элементы внутри первого кольца. \ 

Что же меняется при переходе от ожного поля к другому? У нас есть инвариантные факторы, например, если у нас есть оператор $A$, то для него есть многочлены $f_1, \ldots, f_m\in K[t]$ (последний - минимальный). Тогда для $A_L$ они также инвариантны, причём даже минимальный многочлен такой же. Давайте вспомним, как они строятся в терминах оператора $A$. \ 

Пусть у нас имеется матрица $a$ (перехода $A$), рассмотрим матрицу $a-t\cdot I$, тогда инвариантные факторы - $\frac{\text{НОД}(\text{все миноры порядка } i-1)}{\text{НОД}(\text{все миноры порядка } i)}$. А наибольший общий нелитель не зависит от того, в каком поле ме его рассматривали. Значит, инвариантные факторы не изменятся. \ 

Мы знаем, что в каком-то базисе матрицу $A$ можно привести к фробениусовой форме (на диагонали - квадратики, последняя клетка - соответствующая $f_m$). 

\begin{defn}
    $\End(V)=\{A:V\rightarrow V\}$ - множество всех линейных операторов (эндоморфизмы $V$). Кстати, это кольцо (поточечное сложение и композиция), которое изоморфно $M_n(K)$, посредством выбора базиса. 
\end{defn}

Пусть $c$ - матрица перехода при изменении базиса и $A$ - операотр с матрицей $a$, тогда в новом базисе у него будет матрица $c^{-1}ac$ - \textit{сопряжённая} к $a$ матрица.

\begin{defn}
    $A, \: B$ - \textit{сопряжённые}, если существует $C$ - обратимый $B=C^{-1}AC$. 
\end{defn}

Сформулируем такую теорему, которую мы уже по сути доказали: \\

\begin{theorem}
    $A$, $B$ сопряжены тогда и только тогда, когда у них одинаковые инвариантные факторы.
\end{theorem}

\begin{proof}
    Найдём базис, в котором матрица $A$ записывается в фробениусовой нормальной форме. Существует какой-то другой базис, в котором матрица $B$ записывается точно также. Тогда нужно взять просто матрицу, которая переведёт один базис в другой. В обратную сторону - если $A$ известно в какой фробениусовой форме, то легко определить, что $f_j$ - инвариантные факторы.
\end{proof}

\begin{cons}
    $A,\: B$ сопряжены тогда и только тогда, когда $A_L, \: B_L$ сопряжены. Анаолгично можно записать и для матриц из изоморфности колец.
\end{cons}

Приведём другое доказательство второго пункта в случае бесконечного $K$. Мы хотим найти такую обратимую матрицу, что $ac=cb$. Пусть $с$ - матрица с неизвестными коэффициентами. Тогда у нас имеется система однородных линейных уравнений на $x_{i, j}$, где $c=(x_{i,j})$. Она имеет нетривиальное решение над $L$, причём набор решений образует подпространство $L^{n^2}$ размерности $K$. Тогда над базовым полем $K$ размерность подпространства будет точно такая же, поскольку метод Гаусса не зависит от поля, над которым мы работаем, поэтому он выдаст одинаковые ответы для $K$ и для $L$. \ 

Возьмём базис в этом подпространстве: $c_1, \ldots, c_k$. Рассмотрим всевозможные комбинации $\{\lambda_1c_1+\ldots+\lambda_kc_k\}$, и будме искать такую линейную комбинацию, определитель которой не равен нулю. Мы знаем, что над $L$ такие существуют, потому что над $L$ у нас есть решение. Но определитель - суть многочлен от $\lambda_i$, причём ненулевой, поскольку над $L$ можно найти такие $\lambda_i$, значение при которые не нуль. А поскольку поле $K$ бесконечное, то можно такие коэффициенты найти и над $K$ (индукция по $k$). Доказательство завершили.

Рассмотрим теперь за место $L$ алгебраическое замыкание $K$, $K\leq \overline{K}$. Мы уже знаем, что есть фробениусова нормальная форма, но она не очень удобна. Рассмотрим оператор $A_{\overline{K}}$. Применим для кольца $\overline{K}[t]$ теорему о строении модулей над кольцами главных идеалов, но сначала применим примарное разложение. То есть, возьмём какой-то неприводимый многочлен над алгебраически замкнутым полем, он линейный ($t-\lambda$). И начинаем теперь образовывать блоки. У нас есть $V_{\overline{K}}[t-\lambda]= \{v:\: (t-\lambda)v=0\}=\{v: \: Av=\lambda V\}$ - собственное подпространство, соответствующее собственному числу $\lambda$. $v\neq 0$ из этого множества - собственные векторы, соответствующие собственному числу $\lambda$. \ 

Нас интересуют в качестве $\lambda$ - корни минимального многослена (корни характеристического) (чтобы мы получали ненулевые множества), но можно и проще, преобразуем к $(A-\lambda I)v=0$, но это раносильно тому, что $\det(A-\lambda I)=0$, что и равносильно первому. Далее мы смотрим на степени $V_{\overline{K}}[(t-\lambda)^2]=\{v:\: (t-\lambda)^2v=0\}$, и так далее, а затем берём объединенение $V_\lambda=\bigcup_{i\geq 1}V_{\overline{K}}[(t-\lambda)^i]$, это - корневое подпространство, отвечающее собственному числу $\lambda$. Не стоит путать это с собственным подпространством (по сути, первый и последний член цепи). \ 

Из общей теории мы теперь знаем, что $V_{\overline{K}}=\oplus V_\lambda$, где суммируем по $\lambda$ - собственным числам. Мы получили корневре разложение. Посмотрим теперь, что происходит на каком-то корневом подпространстве. Ограничим $A|_{V_\lambda}$, тогда минимальный многочлен этого ограничения - $(t-\lambda)^k$. А если рассмотреть оператор $A-\lambda I$, то его минимальный многочлен будет $t^k$, то есть, ограничение такой вещи нильпотентное, то есть, матрица будет состоять из квадратиков по диагонали, на диагонали которых нули, а под ними - диагональ из единиц. А вот если мы вернёмся к изначальному сужению, то мы получим матрицу, состоящую из \textit{жордановых блоков}, это то же самое, что и предыдущая матрица, только на главной диагонали везде $\lambda$. \ 
 
Сама матрица $A$ тогда будет состоять из кучи таких блоков для всех $\lambda$ по диагонали, и целиком такое представление $A$ будет называться \textit{жордановой нормальной формой}. Таким образом, для любого ооператора существует базис, в котором матрица выглядит в такой форме. \ 

Рассмотрим такой важный частный случай. Пусть имеется характеристический многочлен $\chi_A(t)$ (по сути, $\det(\lambda I - A)$), и степень его тогда есть размерность пространства ($n$). Предположим, что он раскладывается в произведение линейных $(t-\lambda_1)\ldots(t-\lambda_n)$ (это какое-то условие на характеристический многочлен ($\text{НОД}(\chi_A(t), \chi_A'(t))=1$)). Но всё же, если они все различны, то жорданова форма просто диагональная, так как на диагонали под ними просто ничего не поместится, квадратики единичные. В каком же базисе матрица имеет такой вид? Для того, чтобы это понять, достаточно решить $Av_i=\lambda_iv_i$, $v_i=\neq0$. Такой базис, который мы найдём, \textit{диагонализирует} матрицу. Пока что всё это происходило над алгебраическим замыканием. \ 

Перейдём к случаю $K=\RR$, $\overline{K}=\CC$ и придумаем вещественную жорданову форму (именно её, а не фробениусову, потому что она удобнее). Для этого, нам нужно разобрать немного подробнее процедуру переходу из поля в замыкание с одним и тем же базисом ($V\rightarrow V_{\CC}$). Как нам тогда восстановить $V_{\CC}$? Вообще, никак, но если ввести некую дополнительную структуру это можно сделать. Хочется ввести на $V_{\CC}$ какую-то инволюцию - аналог комплексного сопряжения. Если у нас уже есть базис, то пусть $\overline{z_1e_1+\ldots+z_ne_n}=\overline{z_1}e_1+\ldots+\overline{z_n}e_n $, это операция из $V_{\CC}$ в $V_{\CC}$, которая не будет линейной, а будет \textit{полулинейной}, то есть, выполнено $\overline{zv}=\bar{z}\bar{v}$, а не $\overline{zv}=z\bar{v}$. С суммой же всё нормально. Получили мы \textit{полулинейный оператор}, который является инволюцией. \ 

А само $V$ тогда восстанавливается: $V=\{v\in V_{\CC}: \: \bar{v}=v\}$. Это уже пространство над $\RR$ такой же размерности. \ 

Итого, вещественные векторные пространства по сути есть комплексные векторные пространства такой же размерности с полулинейной инволюцией. Придумаем теперь вещественный аналог жордановой формы. Пусть есть $V$, $A: \: V\rightarrow V$, тогда $V_{\CC}$ назовём \textit{комплексификацией} $V$, а наоборот - \textit{овеществлением}. Также мы можем рассмотреть и комплексификацию $A_{\CC}: \: V_{\CC}\rightarrow V_{\CC}$. Есть базис, в котором он представляется жордановой начальной формой. У $A$ есть характеристический многочлен $f(t)\in \RR[t]$, у которого есть вещественные корни $\alpha_i$ и мнимые $\lambda_j$ вместе со своими сопряжёнными парами. Пусть $\lambda$ - комплексный корень этого многочлена, тогда у нас сеть корневые пространства $V_{\lambda}$ и $V_{\bar{\lambda}}$, тогда их сумма устойчива относительно нашей полулинейной инволюции. Потому что, допустим, $Av=\lambda v$, тогда $\overline{AV}=\bar{\lambda\bar{v}}$, но про $A$ мы знаем, что у его матрицы коэффициенты вещественные, поэтому $\overline{Av}=A\bar{v}$ (можно расписать умножение матрицы на столбец для наглядности). \ 

Таким образом, если $v$ - собственный вектор, отвечающий $\lambda$, то $\bar{v}$ - собственный вектор, отвечающий $\bar{\lambda}$, и со степенью, конечно, то же самое: $(A-\lambda I)^iv=0$, тогда $\overline{(A-\lambda I)^i}\bar{v}=0=(A-\bar{\lambda} I)^iv$. То есть, вся эта прямая сумма относительно $\lambda$ и его сопряжённого, будет устойчивой относительно нашей полуинволюции. Мы знаем, что вещественное векторное пространство это то же самое, что и комплексное с полуинволюцией, тогда все пары $\lambda_i$ и их сопряжённый будут объединяться в пары и давать вещественные подпространства. 



\end{document}