\documentclass[a4paper,100pt]{article}

\usepackage[utf8]{inputenc}
\usepackage[unicode, pdftex]{hyperref}
\usepackage{cmap}
\usepackage{mathtext}
\usepackage{multicol}
\setlength{\columnsep}{1cm}
\usepackage[T2A]{fontenc}
\usepackage[english,russian]{babel}
\usepackage{amsmath,amsfonts,amssymb,amsthm,mathtools}
\usepackage{icomma}
\usepackage{euscript}
\usepackage{mathrsfs}
\usepackage{geometry}
\usepackage[usenames]{color}
\hypersetup{
     colorlinks=true,
     linkcolor=magenta,
     filecolor=,
     citecolor=magenta,      
     urlcolor=magenta,
     }
\usepackage{fancyhdr}
\pagestyle{fancy} 
\fancyhead{} 
\fancyhead[LE,RO]{\thepage} 
\fancyhead[CO]{\hyperlink{t2}{к списку объектов}}
\fancyhead[LO]{\hyperlink{t1}{к содержанию}} 
\fancyhead[CE]{текст-центр-четные} 
\fancyfoot{}
\newtheoremstyle{indented}{0 pt}{0 pt}{\itshape}{}{\bfseries}{. }{0 em}{ }

%\geometry{verbose,a4paper,tmargin=2cm,bmargin=2cm,lmargin=2.5cm,rmargin=1.5cm}

\title{Теория вероятностей. Конспект 2 сем.}
\author{Мастера Конспектов\\ \\ (по материалам лекций Давыдова Ю. А.,\\ а также других источников)}
\date{16 февраля 2021 г.}

\theoremstyle{indented}
\newtheorem{theorem}{Теорема}
\newtheorem{lemma}{Лемма}

\theoremstyle{definition} 
\newtheorem{defn}{Определение}
\newtheorem{exl}{Пример(ы)}

\theoremstyle{remark} 
\newtheorem{remark}{Примечание}
\newtheorem{cons}{Следствие}
\newtheorem{stat}{Утверждение}

\DeclareMathOperator{\Ker}{Ker}
\DeclareMathOperator{\Tors}{Tors}
\DeclareMathOperator{\Frac}{Frac}
\DeclareMathOperator{\Imf}{Im}
\DeclareMathOperator{\cont}{cont}
\DeclareMathOperator{\id}{id}
\DeclareMathOperator{\ev}{ev}
\DeclareMathOperator{\lcm}{lcm}
\DeclareMathOperator{\chard}{char}
\DeclareMathOperator{\CC}{\mathbb{C}}
\DeclareMathOperator{\ZZ}{\mathbb{Z}}
\DeclareMathOperator{\RR}{\mathbb{R}}
\DeclareMathOperator{\NN}{\mathbb{N}}
\DeclareMathOperator{\PP}{\mathbb{P}}
\DeclareMathOperator{\FF}{\mathcal{F}}
\DeclareMathOperator{\Rho}{\mathcal{P}}
\DeclareMathOperator{\codim}{codim}
\DeclareMathOperator{\rank}{rank}
\DeclareMathOperator{\ord}{ord}
\DeclareMathOperator{\adj}{adj}

\begin{document}

\newcommand{\resetexlcounters}{%
  \setcounter{exl}{0}%
} 

\newcommand{\resetremarkcounters}{%
  \setcounter{remark}{0}%
} 

\newcommand{\reseconscounters}{%
  \setcounter{cons}{0}%
} 

\newcommand{\resetall}{%
    \resetexlcounters
    \resetremarkcounters
    \reseconscounters%
}

\maketitle 

\newpage

\hypertarget{t1}{Некоторые} записи по теории вероятностей.
\tableofcontents

\newpage


\section{Лекция 1.}

Начинаем мы с самого базового - аксиоматики и введения определений.

\begin{defn}
    $\Omega$ - \textit{пространство элементарных событий} или \textit{множество элементарных исходов}, есть множество, состоящее из $\omega_i$, \textit{элементарных событий}. Нам важно лишь, чтобы это множество было непустым. $\FF\subseteq \Rho(\Omega)$ - некоторая совокупность подмножеств $\Omega$, есть \textit{множество событий}, элементы которого есть $A_i$ - события.
\end{defn}

\begin{defn}
    $\PP$ - вероятность $A\Rightarrow \PP(A)$ - вероятность события $A$.
\end{defn}

\begin{defn}
    Вся же тройка $(\Omega, \FF, \PP)$ называется \hypertarget{n1}{\textcolor{magenta}{\textit{вероятностным пространством}}}.
\end{defn}

Для вероятностей существует несколько аксиом: 

\begin{itemize}
    \item $0\leq \PP(a\leq 1)$ для любого события, 
    \item $\PP(\Omega)=1$, 
    \item для любого счётного набора попарно непересекающихся события $\{A_i\}_{i\in N}\subseteq\FF$ выполнена \textit{счётная аддитивность}:
    \[
        \PP\biggl( \bigsqcup_{n\in N} A_n\biggr)=\sum_{n\in N}\PP(A_n).
    \]
\end{itemize}

Некоторые свойства вероятностей:

\begin{itemize}
    \item $A\subset B \Rightarrow \PP(A)\leq \PP(B)$;
    \item $\PP(A)=1-\PP(A^c)$;
    \item $\forall A, B \Rightarrow \PP(A\cup B)=\PP(A)+\PP(B)-\PP(A\cap B)$;
    \item $\PP(\bigcup_n A_n)\leq \sum_n \PP(A_n)$.
\end{itemize}

Теперь перейдём к некоторым примерам вероятностных пространств:

\begin{exl}

    Пространство $(\Omega, \FF, \PP)$ \hypertarget{n2}{\textcolor{magenta}{\textit{дискретно}}}, если $\Omega$ не более, чем счётно. $\FF=\Rho(\Omega)$, элементы $\{\omega\}$ также считаем событиями.
    \begin{stat}
        Несколько предложений:

        \begin{itemize}
            \item Пусть $\PP$ - вероятность в $(\Omega, \FF, \PP)$. Тогда $\PP(A)=\sum_{\omega\in A}p_\omega$, где $p_\omega=\PP\{\omega\}$. При этом $p_\omega\geq 0$, $\sum_\omega p_\omega = 1$. 
            \item Предположим, что $\{p_\omega\}_{\omega\in\Omega}$ такие, что выполнено последнее предложение предыдущего пункта, тогда $\PP(A)=\sum_{\omega\in A}p_\omega$ - вероятность.
        \end{itemize}
    \end{stat}
\end{exl}

Также, можно упомянуть про \hypertarget{n3}{\textcolor{magenta}{\textit{равновероятные исходы}}}, из названия понятно, что это. Если $\vert \Omega\vert <\infty$ и $p_\omega=p$ для любого $\omega$, тогда $\PP(A)=\frac{\vert A\vert}{\vert \omega \vert}$. \ 

Начнём теперь разбираться с понятием \textit{условная вероятность}.

\begin{exl}
    Начнём с такого примера. Пусть у нас есть события $A, B$, причём их пересечение в вероятностном пространстве пусть. Тогда если исполнится $B$, то $A$ уже исполнится не может.
\end{exl}

\begin{defn}
    \hypertarget{n4}{\textcolor{magenta}{\textit{Условная вероятность}}}: $\PP_B(A)=\frac{\PP(A\cap B)}{\PP(B)}$ (при $\PP(B)>0$).
\end{defn}

\begin{stat}
    Для условной вероятности выполнены аксиомы вероятности.
\end{stat}

А  теперь - несколько утверждений, которые касаютсся условной вероятности.

\begin{stat}
    $(B_n)$ - \textit{разбиение} $\Omega$ (дизъюнктный набор, который в объединении даёт всё множество). Тогда для любого $A$ $\PP(A)=\sum_k\PP(B_k)\PP_{B_k}(A)$. 
\end{stat}

\begin{proof}
    \[
        \PP(A)=\PP(A\cap \Omega)=\sum_n \PP(A\cap (\bigcup_n B_n)) = \PP(\bigcup_n (A\cap B_n))=\sum_n \PP(A\cap B_n)=\PP(B_n)\cdot \PP_{B_n}(A).
    \]
\end{proof}

\begin{stat}
    \hypertarget{n5}{\textcolor{magenta}{\textit{Формула Байеса}}}. Пусть мы знаем событие $A$, имеется разбиение $(B_n)$, тогда 
    \[
    \PP_A(B_k)=\frac{\PP(A\cap B_k)}{\PP(A)}=\frac{\PP(B_k)\PP_{B_k}(A)}{\sum_n\PP(B_n)\PP_{B_n}(A)}
    \]
\end{stat}

\begin{stat}
    \hypertarget{n6}{\textcolor{magenta}{\textit{Формула умножения}}}. 
    \[
        \PP(\bigcap_1^n A_k)=\PP(A_1)\cdot \PP_{A_1}(A_2)\cdot\PP_{A_1\cap A_2}(A_3)\cdot\ldots
    \]
\end{stat}

Перейдём к \textit{независимости событий}. Начнём рассуждения с двух событий: $A$ и $B$. Если $\PP(B)=\PP_A(B)$, $\PP(A)=\PP_B(A)$, или, что равносильно им обоим $\PP(A\cap B)=\PP(A)\PP(B)$, то события называются \textit{независимыми}.\ 

Пусть теперь имеется не два, а больше событий $\{A_q, \ldots, A_n\}$. Нельзя сказать, что нам хватает попарной независимости для независимости совокупной.

\begin{exl}
    (Пирамида Бернштейна). Рассмотрим тетраедр, у которого стороны покрашены таким образом: белый, синий, красный и флаг России. Рассматриваем события: $A_1$ - на выпавшем основании есть белый цвет, и так далее $A_2$ и $A_3$. Эти события попарно независмы, но не независимы в совокупности.
    \[
        \PP(A_i)=\frac{1}{2}, \: \PP(A_1\cap A_2)=\frac14=\frac12\cdot\frac12=\PP(A_1)\PP(A_2), 
    \]
    но тогда 
    \[
        \PP((A_1\cap A_2)\cap A_3)=\frac14\neq\frac18.
    \]
\end{exl}

Таким образом, нужно ввести корректное определение.

\begin{defn}
    События $A_1, \dots, A_n$ \hypertarget{n7}{\textcolor{magenta}{\textit{независимы}}}, если выполнено:
    \begin{align*}
        \PP(A_i\cap A_j)&=\PP(A_i)\PP(A_j), \: \forall i\neq j,  \\ 
        \PP(A_i\cap A_j\cap A_k)&=\PP(A_i)\PP(A_j)\PP(A_k), \: \forall i\neq j\neq k,  \\
        &\dots \\
        \PP(\bigcup_1^n A_i)&=\prod_1^n \PP(A_i).
    \end{align*}
\end{defn}

\begin{theorem}
    Пусть имеется $T_1, \ldots, T_m$ - разбиение $\{1, \ldots, n\}$, независимые события $A_1, \ldots, A_n$, $\{B_j\}_m$ - комбинация (всякие действия между элементами) событий $\{A_s, s\in T_j\}$. Тогда $\{B_j\}$ - независимы.
\end{theorem}

\begin{proof}
    По индукции.
\end{proof}

\begin{defn}
    \hypertarget{n8}{\textcolor{magenta}{\textit{Случайная величина}}} - это функция $X:\Omega\rightarrow R$. 
\end{defn}

\begin{exl}
    Число выпавших решек на $n$ бросках.
\end{exl}

Теперь немного о \textit{распределении случайной величины}. Пусть имеется вероятностное пространство и случайная величина $X$. Нас интересует $\{\omega\vert X(\omega)\in B\subseteq \RR \}$, то есть, мы хотим исследовать попадания случайной величины в те или иные зоны на прямой. Такую вероятность можно рассматривать как вероятность от множества $B$, но это слишком сложно, поэтому продолжим на таких двух пунктах:

\begin{itemize}
    \item значения $X$, $X(\Omega)=\{a_1, \ldots\}$, $\{a_k\}$ - значение $X$, 
    \item $A_k=\{\omega\vert X(\omega)=a_k\}$; $p_k=\PP(A_k)$, причём каждая $p_k\geq0$, а их сумма равна единице.
\end{itemize}

Тогда мы можем сделать вывод, что $\PP\{X\in B\}=\sum_{k\vert a_k\in B}p_k$, так как левая часть есть $\PP\{\bigcup_{k\vert a_k\in B}\}$, что равно $\PP\{\bigcup_{k\vert a_k\in B}\{x=a_k\}\}=\sum_{k\vert a_k\in B}\PP\{x=a_k\}$, что уже и равно левой части.\

\begin{defn}
    Таки образом, совокупность последовательностей $\frac{\{a_k\}}{\{p_k\}}$ и называется \hypertarget{n9}{\textcolor{magenta}{\textit{распеределением случайной величины}}}. 
\end{defn}

\begin{exl}
    Приведём примеры распределений:

    \begin{itemize}
        \item \textit{вырожденное}: $X(\omega)=a$ для любого $\omega$.
        \item \textit{распределение Бернулли}: $B(1, p)$, $p\in[0,1]$, причём единица принимается с вероятностью $p$, $0$ - иначе.
        \item \hypertarget{n10}{\textcolor{magenta}{\textit{биномиальное}}}: $B(n, p)$, $p\in[0,1]$, $X\sim B(n, p)$, если принимаются значения от $0$ до $n$, причём $\PP\{X=k\}=\PP_n(k)$ (просто обозначение), и равно $C_n^kp^k(1-p)^{n-k}$.
    \end{itemize}

\end{exl}

\section{Лекция 2.}

Приведём ещё несколько примеров распределений: 

\begin{exl}
    \begin{itemize}
        \item \textit{геометрическое} $X = 1, 2, \ldots$, $p\in[0,1]$. $P\{X=k\}=(1-p)^{k-1}p$.
        \item \textit{плацоновское} $X\sim \Rho(\alpha), \alpha>0$, $X=0, 1, \ldots$, $P\{X=k\}=\frac{\alpha^k}{k!}e^{-\alpha}$. 
    \end{itemize}
\end{exl}

Независимость случайных величин.

\begin{exl}
    Пусть $X\sim \Rho(\alpha)$, $Y\sim \Rho(\beta)$ - независимы (плацоновские распределения). Найдём тогда распределение $X+Y$, $X+Y=0, 1, 2, \ldots$. 
   
    \begin{equation*}
        \begin{aligned}
            \PP\{X+Y=n\} & =\PP\{\sqcup_{k=0}^n\{X+Y=n, X=k\}\} =\sum_{k=0}^n\PP\{X+Y=n, X=k\}= \\ 
             & = \sum_{k=0}^n\{X=k, Y=n-k\} =\sum_{k=0}^n\PP\{X=K\}\PP\{Y=n-k\}= \\ 
             & = \sum_{k=0}^n\frac{\alpha^k}{k!}e^{-\alpha}\cdot \frac{\beta^{n-k}}{k(n-))!}e^{-\beta} = e^{-(\alpha+\beta)}\frac{1}{n!}\sum_{k=0}^n\frac{n!}{k!(n-k)!}\alpha^k\beta^{n-k}= \\ 
             & = e^{-(\alpha+\beta)}\frac{1}{n!}(\alpha+\beta)^n =\frac{(\alpha+\beta)^n}{n!}(\alpha+\beta)^n=\frac{(\alpha+\beta)^n}{n!}e^{-(\alpha+\beta)}. 
        \end{aligned}
    \end{equation*}
        

    Таким образом, получаем вывод: $X+Y\sim\Rho(\alpha+\beta)$ - плацоновское распределение.
\end{exl}

Испытание Бернулли. 

\begin{defn}
    Рассмотрим последовательность $\varepsilon_k$ - независимых бернуллиевских величин (два исхода), $\varepsilon_k\sim B(1, p)$, 
    \begin{equation*}
        \varepsilon_k=
        \begin{cases}
            1, \: p \\ 
            0, \: q=1-p
        \end{cases}
    \end{equation*}
    $\{\varepsilon_k=1\}$ означает успех в каком-то испытании. То есть, \textit{испытание Бернулли} - последовательность из однотипных испытаний.
\end{defn}

$\nu_n$ - число успехов в $n$ испытаниях., что равно $\sum_{k=1}^n\varepsilon_k$. Покажем, что распределение биномиально. $\nu_n=0, 1, \ldots, n$.

\begin{equation*} 
    \begin{aligned}
        C_n^kp^kq^{n-k} & =^?\PP\{\nu_n=l\}=\PP\{\omega\vert \text{ в цепочке }\varepsilon_1(\omega), \ldots, \varepsilon_n(\omega)\text{ в точности } k \text{единиц}\}= \\ 
        & = \PP\{\sum_1^n\varepsilon_j=k\}=\sum_{\{i_1, \ldots, i_n\}| \text{ в } \bar{i} \: k \text{ единиц }}\PP\{\varepsilon_1= i_1, \ldots, \varepsilon_n=i_n\}= \\ 
        & = \sum_{-//-}p^kq^{n-k}=C_n^k p^kq^{n-k}.
    \end{aligned}
\end{equation*} 

Теперь встаёт вопрос: существует ли независимая бернуллиевская случайная величина. Пусть $\Omega=\{(i_1, i_2, \ldots, i_n)|i_k=0 \text{ или }1\}$; $|\Omega|=2^k$. $p_{\omega}= p^k(1-p)^{n-k}$ для $\omega=(i_1, \ldots, i_n)$, в которой ровно $k$ единиц. $\sum p_n=1$, $\varepsilon_k(\omega)=\varepsilon_k(i_1, \ldots, i_n)=i_k$ .\

$P_n(k)=\PP\{\nu_n=k\}=C_n^kp^kq^{n-k}$ - можем ли мы как-то упростить это? \\

\begin{theorem}
    $n$ испынаний $\text{Б}$ с веростностями успеха $p=p_n$, $n\cdot p_n\rightarrow_{n\rightarrow \infty} \alpha$, тогда $\forall k\in \NN$ $P_n(k)\rightarrow \pi_k=\frac{\alpha^k}{k!}e^{-\alpha}$. 
\end{theorem}

\begin{proof}
    \begin{equation*}
        \begin{split}
            P_n(k)=\frac{n!}{k!(n-k)!}p^kq^{n-k}=\frac{1}{k!}\frac{n(n-1)\ldots(n-k+1)}{n^k}(np^k)q^{n-k}= \\ 
            = \frac{1}{k!}{(1-\frac{1}{n})\ldots(1-\frac{k-1}{n})}(np)^kq^{n-k}\sim\frac{\alpha^k}{k!}q^{n-k}
        \end{split}
    \end{equation*}

    Если что, $k$ - константа. 

    \begin{equation*}
        \begin{split}
            q^{n-k}=(1-p)^{n-k}=(1-\frac{n p_n}{n})^n(1-\frac{n p_n}{n})^{-k} \\ 
            \sim (1-\frac{n p_n}{n})^n \sim e^{-n p_n} = e^{-\alpha}. 
        \end{split}
    \end{equation*}

    В итоге, $P_n(k)\sim\frac{\alpha^k}{k!}e^{-\alpha}$. 

\end{proof}

А теперь - несколько слов о точности аппроксимации. Оценим $|P_n(k)-\pi_k|$. $\pi_k=\frac{(np)^k}{k!}e^{-np}$. \\

\begin{theorem}
    $\sum_{k=0}^\infty|P_n(k)-\pi_n|\leq 2np^2$.
\end{theorem}

\newpage

\hypertarget{t2}{И в заключение...}

\section{Пофамильный указатель всех важных предметов}

\begin{multicols}{2}
    [
    Быстрый список для особо ленивого поиска.
    ]

    \hyperlink{n1}{биномиальное распределение}\
    
    \hyperlink{n1}{вероятностное пространство}\
    
    \hyperlink{n2}{дискретное пространство}\

    \hyperlink{n7}{независимые события}\

    \hyperlink{n3}{равновероятные исходы}\

    \hyperlink{n1}{распределение случайной величины}\

    \hyperlink{n8}{случайная величина}\

    \hyperlink{n4}{условная вероятность}\

    \hyperlink{n5}{формула Байеса}\

    \hyperlink{n6}{формула умножения}\



\end{multicols}

\end{document}