\documentclass[a4paper]{article}

%plastikovye pakety

\usepackage[12pt]{extsizes}
\usepackage[utf8]{inputenc}
\usepackage[unicode, pdftex]{hyperref}
\usepackage{cmap}
\usepackage{mathtext}
\usepackage{multicol}
\setlength{\columnsep}{1cm}
\usepackage[T2A]{fontenc}
\usepackage[english,russian]{babel}
\usepackage{amsmath,amsfonts,amssymb,amsthm,mathtools}
\usepackage{icomma}
\usepackage{euscript}
\usepackage{mathrsfs}
\usepackage[dvipsnames]{xcolor}
\usepackage[left=2cm,right=2cm,
    top=2cm,bottom=2cm,bindingoffset=0cm]{geometry}
\usepackage[normalem]{ulem}
\usepackage{graphicx}
\usepackage{makeidx}
\makeindex
\graphicspath{{pictures/}}
\DeclareGraphicsExtensions{.pdf,.png,.jpg}
%\usepackage[usenames]{color}
\hypersetup{
     colorlinks=true,
     linkcolor=coralpink,
     filecolor=coralpink,
     citecolor=black,      
     urlcolor=coralpink,
     }
\usepackage{fancyhdr}
\pagestyle{fancy} 
\fancyhead{} 
\fancyhead[LE,RO]{\thepage} 
\fancyhead[CO]{\hyperlink{uk}{к списку объектов}}
\fancyhead[LO]{\hyperlink{sod}{к содержанию}} 
\fancyfoot{}
\newtheoremstyle{indented}{0 pt}{0 pt}{\itshape}{}{\bfseries}{. }{0 em}{ }

\renewcommand\thesection{}
\renewcommand\thesubsection{}

%\geometry{verbose,a4paper,tmargin=2cm,bmargin=2cm,lmargin=2.5cm,rmargin=1.5cm}

\title{Конспект лекций по матанализу}
\author{Горбунов Леонид \\ 
 при участии и редакторстве @keba4ok \\ 
    на основе лекций Любарского Ю. И.}
\date{13 сентября 2021г.}


%envirnoments
    \theoremstyle{indented}
    \newtheorem{theorem}{Теорема}
    \newtheorem{lemma}{Лемма}
    \newtheorem{alg}{Алгоритм}

    \theoremstyle{definition} 
    \newtheorem{defn}{Определение}
    \newtheorem{exl}{Пример(ы)}
    \newtheorem{prob}{Задача}

    \theoremstyle{remark} 
    \newtheorem{remark}{Примечание}
    \newtheorem{cons}{Следствие}
    \newtheorem{exer}{Упражнение}
    \newtheorem{stat}{Утверждение}
%esli ne hochetsa numeracii - nuzhno prisunut' zvezdochku-pezsochku

\definecolor{coralpink}{rgb}{0.19, 0.55, 0.91}

%declarations
        %arrows_shorten
            \DeclareMathOperator{\la}{\leftarrow}
            \DeclareMathOperator{\ra}{\rightarrow}
            \DeclareMathOperator{\lra}{\leftrightarrow}
            \DeclareMathOperator{\llra}{\longleftrightarrow}
            \DeclareMathOperator{\La}{\Leftarrow}
            \DeclareMathOperator{\Ra}{\Rightarrow}
            \DeclareMathOperator{\Lra}{\Leftrightarrow}
            \DeclareMathOperator{\Llra}{\Longleftrightarrow}

        %letters_different
            \DeclareMathOperator{\CC}{\mathbb{C}}
            \DeclareMathOperator{\ZZ}{\mathbb{Z}}
            \DeclareMathOperator{\RR}{\mathbb{R}}
            \DeclareMathOperator{\NN}{\mathbb{N}}
            \DeclareMathOperator{\HH}{\mathbb{H}}
            \DeclareMathOperator{\LL}{\mathscr{L}}
            \DeclareMathOperator{\KK}{\mathscr{K}}
            \DeclareMathOperator{\GA}{\mathfrak{A}}
            \DeclareMathOperator{\GB}{\mathfrak{B}}
            \DeclareMathOperator{\GC}{\mathfrak{C}}
            \DeclareMathOperator{\GD}{\mathfrak{D}}
            \DeclareMathOperator{\GN}{\mathfrak{N}}
            \DeclareMathOperator{\Rho}{\mathcal{P}}
            \DeclareMathOperator{\FF}{\mathcal{F}}

        %common_shit
            \DeclareMathOperator{\Ker}{Ker}
            \DeclareMathOperator{\Frac}{Frac}
            \DeclareMathOperator{\Imf}{Im}
            \DeclareMathOperator{\cont}{cont}
            \DeclareMathOperator{\id}{id}
            \DeclareMathOperator{\ev}{ev}
            \DeclareMathOperator{\lcm}{lcm}
            \DeclareMathOperator{\chard}{char}
            \DeclareMathOperator{\codim}{codim}
            \DeclareMathOperator{\rank}{rank}
            \DeclareMathOperator{\ord}{ord}
            \DeclareMathOperator{\End}{End}
            \DeclareMathOperator{\Ann}{Ann}
            \DeclareMathOperator{\Real}{Re}
            \DeclareMathOperator{\Res}{Res}
            \DeclareMathOperator{\Rad}{Rad}
            \DeclareMathOperator{\disc}{disc}
            \DeclareMathOperator{\rk}{rk}
            \DeclareMathOperator{\const}{const}
            \DeclareMathOperator{\grad}{grad}
            \DeclareMathOperator{\Aff}{Aff}
            \DeclareMathOperator{\Lin}{Lin}
            \DeclareMathOperator{\Prf}{Pr}
            \DeclareMathOperator{\Iso}{Iso}

        %specific_shit
            \DeclareMathOperator{\Tors}{Tors}
            \DeclareMathOperator{\form}{Form}
            \DeclareMathOperator{\Pred}{Pred}
            \DeclareMathOperator{\Func}{Func}
            \DeclareMathOperator{\Const}{Const}
            \DeclareMathOperator{\arity}{arity}
            \DeclareMathOperator{\Aut}{Aut}
            \DeclareMathOperator{\Var}{Var}
            \DeclareMathOperator{\Term}{Term}
            \DeclareMathOperator{\sub}{sub}
            \DeclareMathOperator{\Sub}{Sub}
            \DeclareMathOperator{\Atom}{Atom}
            \DeclareMathOperator{\FV}{FV}
            \DeclareMathOperator{\Sent}{Sent}
            \DeclareMathOperator{\Th}{Th}
            \DeclareMathOperator{\supp}{supp}
            \DeclareMathOperator{\Eq}{Eq}
            \DeclareMathOperator{\Prop}{Prop}


%env_shortens_from_hirsh            
    \newcommand{\bex}{\begin{example}\rm}
    \newcommand{\eex}{\end{example}}
    \newcommand{\ba}{\begin{algorithm}\rm}
    \newcommand{\ea}{\end{algorithm}}
    \newcommand{\bea}{\begin{eqnarray*}}
    \newcommand{\eea}{\end{eqnarray*}}
    \newcommand{\be}{\begin{eqnarray}}
    \newcommand{\ee}{\end{eqnarray}}
    \newcommand{\abs}[1]{\lvert#1\rvert}
        \newcommand{\bp}{\begin{prob}}
        \newcommand{\ep}{\end{prob}}

    
\begin{document}
%ya_ebanutyi
\newcommand{\resetexlcounters}{%
  \setcounter{exl}{0}%
} 
\newcommand{\resetremarkcounters}{%
  \setcounter{remark}{0}%
} 
\newcommand{\reseconscounters}{%
  \setcounter{cons}{0}%
} 
\newcommand{\resetall}{%
    \resetexlcounters
    \resetremarkcounters
    \reseconscounters%
}

\newcommand{\cursed}[1]{\textit{\textcolor{coralpink}{#1}}}
\newcommand{\de}[3][2]{\index{#2}{\textbf{\textcolor{coralpink}{#3}}}}
\newcommand{\re}[3][2]{\hypertarget{#2}{\textbf{\textcolor{coralpink}{#3}}}}
\newcommand{\se}[3][2]{\index{#2}{\textit{\textcolor{coralpink}{#3}}}}

\maketitle 

\newpage

\hypertarget{sod}
\tableofcontents

\newpage


\section{Теория меры}

\subsection{Алгебраические структуры подмножеств}

Пусть нам дано множество $\mathcal{X}$ произвольной природы и система его подмножеств $\GA$.

\begin{defn}
    $\GA$ - \se{Полукольцо множеств}{полукольцо множеств}, если для любых $A$, $B$ $\in \GA$ их пересечение $A \cap B$ тоже лежит в $\GA$, а их разность $A \backslash B$ представляется в виде конечного объединения попарно дизъюнктных множеств из $\GA$.
\end{defn}

\begin{remark}
    Легко понять, что любое полукольцо содержит пустое множество.
\end{remark}

\begin{defn}
    $\GA$ - \se{Кольцо множеств}{кольцо множеств}, если для любых $A$, $B$ $\in \GA$ их пересечение $A \cap B$, объединение $A \cup B$ и разность $A \backslash B$ лежат в $\GA$
\end{defn}

\begin{remark}
    Легко понять, что тогда и $A \triangle B$ лежит в $\GA$. Тогда если на элементах кольца множеств определить операции сложения $+:= \triangle$ и умножения $\times := \cap$, то оно превратится в алгебраическое кольцо. 
\end{remark}

\begin{defn}
    $\GA$ - \se{Алгебра множеств}{алгебра множеств}, если оно кольцо, и для любого $A \in \GA$ множество $X \backslash A$ тоже лежит в $\GA$
\end{defn}

\begin{stat}
    Пусть $\GA \subseteq \Rho(X)$ и $\GB \subseteq \Rho(Y)$ - полукольца. Тогда $\GA \times \GB \subseteq \Rho(X \times Y)$ - тоже полукольцо.
\end{stat}

\begin{stat}
    Пусть множества $A$, $B_1$, ... $B_n$ принадлежат какому-то полукольцу. Тогда $A \backslash (B_1 \cup ... \cup B_n)$ представляется в виде объединения конечного числа элементов этого полукольца.
\end{stat}

\begin{proof}
    $A \backslash (B_1 \cup ... \cup B_n)=(A \backslash B_1) \cap ... \cap (A \backslash B_n)=(\bigsqcup_{i=1}^{k_1}C_{1, i}) \cap ... \cap (\bigsqcup_{i=1}^{k_n}C_{n, i}) = \bigsqcup_{i_1, ... i_n} (C_{1, i_1} \cap ... \cap C_{n, i_n})$. В последнем выражении все множества попарно дизъюнктны, так как если бы, например, $(C_{1, i_1} \cap ... \cap C_{n, i_n}) \cap C_{1, j_1} \cap ... \cap C_{n, j_n} \ni x$, то для каждого $k$ от $1$ до $n$ $x \in C_{k, i_k} \cap C_{k, j_k}$, что возможно только при $i_k=j_k$, но для всех $k$ это равенство быть верным не может.
\end{proof}

\begin{exl}
    $P(\RR)=\{[a, b) | a, b, \in \RR \cup \{\pm \infty\}\}$ - \se{Полукольцо ячеек}{полукольцо ячеек}
    \\
    $P(\RR^n)=\{[a_1, b_1) \times ... \times [a_n, b_n) | a_i, b_i, \in \RR \cup \{\pm \infty\}\}$ - тоже полукольцо ячеек, только многомерных
\end{exl}

\subsection{Вводим меру}

Пусть $\mathfrak{X}$ - множество произвольной природы, $\GA \subseteq \Rho(\mathfrak{X})$. 

\begin{defn}
    Функция $\mu: \GA \ra \RR_{\geq 0} \cup \{+\infty\}$ называется \se{Мера}{мерой}, если для любых попарно дизъюнктных множеств $A_1$, ... $A_k$ $\in \GA$ и таких, что $\bigsqcup_{i=1}^k A_i \in \GA$, верно равенство $\mu(\bigsqcup_{i=1}^k A_i)=\sum_{i=1}^k \mu(A_i)$
\end{defn}

\begin{remark}
    Данное свойство называется \textit{аддитивностью}
\end{remark}

\begin{exl}
    \
    \begin{itemize}
        \item $\mathfrak{X}$ - дискретное пространство, и для любого $x \in \mathfrak{X}$ $\mu({x})=1$. Тогда $\mu(A)=\sum_{x \in A} 1$
        \item $\mathfrak{X}$ - дискретное пространство, и для любого $x \in \mathfrak{X}$ $\mu({x})=p_x$, причём $\sum_{x \in \mathfrak{X}} p_x=1$. Тогда мы получаем в точности вероятностное пространство.
        \item $\mathfrak{X}=\RR$, $\GA$ - полукольцо конечных ячеек. Тогда $\mu([a, b))=b-a$ - мера.
        \item То же, что и в предыдущем примере, только теперь $\mu([a, b))=f(b)-f(a)$, где $f$ - монотонно возрастающая функция.
    \end{itemize}
\end{exl}

\begin{stat}
    Мера, определённая на полукольце, монотонна: если $A$, $B$ $\in \GA$, и $B \subseteq A$, то $\mu(B) \leq \mu(A)$.
\end{stat}

\begin{proof}
    $\mu(A)=\mu(B)+\mu(A \backslash B) = \mu(B)+\mu(\bigsqcup_{i=1}^n C_i)=\mu(B)+\sum_{i=1}^n \mu(C_i) \geq \mu(B)$
\end{proof}

\subsection{Простые функции}

\begin{defn}
    Пусть $\GA$ - полукольцо, и $A \in \GA$. Определим \se{Функция-индикатор}{функцию-индикатор} (или \se{Характеристическая функция}{характеристическую функцию}): 

\begin{equation*}
    \chi_A(x)=
  \begin{cases}
     1 , \: \text{если} \: x \in A, \\
     0 , \: \text{если} \: x \notin A\\
  \end{cases}
\end{equation*}

\end{defn}

\begin{defn}
    \se{Простая функция}{Простая функция} - это функция вида $f(x)=\sum_{i=1}^n a_i \chi_{A_i}(x)$, где $A_i \in \GA$ и $a_i \in \RR$
\end{defn}

\begin{remark}
    Сумма и произведение простых функций - простые функции.
\end{remark}

\subsection{Элементарный интеграл}

Пусть мы имеем $\GA$ - полукольцо, $\mu$ - меру и $f$ - простую функцию (всё пока что конечно). Можем тогда ввести следующее понятие:

\begin{defn}
    \se{Интеграл!элементарный}{Элементарным интегралом} называется 

    \[
        \int f(x)dx = \sum a_i \mu(A_i)
    \]
\end{defn}

\begin{stat}
    Определение корректно.
\end{stat}

\begin{remark}
    Я не понял, что тут рассказывает Юрий Ильич, поэтому доказательство найдено в других источниках. Суть просто в попарном подразбиении и перегуппировке.
\end{remark}

\begin{proof}
    Пусть $f = \sum \alpha_i \cdot \chi(a_i) = \sum \beta_j \cdot \chi(b_j)$, рассмотрим тогда $c_{ij} = a_i \cap b_j$.  

    \[
        \sum \mu(a_j) \cdot \alpha_j = \sum \mu(c_{ij}) \cdot \alpha_i = \sum \mu(c_{ij}) \cdot \beta_j = \sum \mu(b_j) \beta_j
    \]
\end{proof}

\begin{stat}[Техническое замечение]
    \[
        \int \chi_A = \mu(A).
    \]
\end{stat}

\begin{stat}
    Рассмотрим свойства интеграла: 

    \begin{itemize}
        \item Линейность. Если у нас есть две простые функции: $f$ и $g$, а также два числа: $\alpha, \beta \in \RR$, тогда 
        \[
            \int \alpha f + \beta g = \alpha \int f + \beta \int g.
        \]
        \item Монотонность. Пусть $f$ и $g$ - простые функции, а также $f \leq g$. Тогда 
        \[
            \int f \leq \int g.
        \]
    \end{itemize}
\end{stat}

\begin{remark}
    Для доказательства практически всего нужно просто рассмотреть дизъюнктное подразбиение данных функций.
\end{remark}

\subsection{Включаем бесконечность}

Пусть у нас, по прежнему, имеется кольцо, и простая функция $f$. Выделим тогда у неё положительную и отрицательную часть ($f^{+}$ и $f^{-}$). Такие, что положительная часть во всех положительных значениях остаётся таковой, а при отрицательных - обнуляется. Почти аналогично с отрицательной, только мы рассмотриваем модуль того, что останется. Таким образом,

\[
    f = f^{+} - f^{-}. 
\]

Определим тогда $I_{+}(f) = \int f_{+}$, и аналогично $I_{-}$. Мы хотим определить интеграл от функции, как $I_{+}(f) - I_{-}(f)$. Но нам мешает то, что обе эти функции могут быть бесконечными. Так что в случае, когда оба интеграла равны бесконечности, у нас ничего не получится, и этот случай мы попросу запрещаем. И рассмотриваем мы теперь только функции, который могут быть бесконечны максимум в одну сторону.

\begin{remark}
    Монотонность и линейность останутся при данном определении (последнее, конечно, опять таки при конечности хотя бы одного из интегралов). 
\end{remark}



\hypertarget{dex}
    \printindex


%staryi_variant
%\hypertarget{uk}{Основные понятия.}

%\begin{multicols}{2}
%    \hyperlink{}{} \ 
%\end{multicols}



%novyi_variant


\end{document}