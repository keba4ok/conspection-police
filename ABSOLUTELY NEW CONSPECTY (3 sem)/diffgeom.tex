\documentclass[a4paper]{article}

%plastikovye pakety

\usepackage[12pt]{extsizes}
\usepackage[utf8]{inputenc}
\usepackage[unicode, pdftex]{hyperref}
\usepackage{cmap}
\usepackage{mathtext}
\usepackage{multicol}
\setlength{\columnsep}{1cm}
\usepackage[T2A]{fontenc}
\usepackage[english,russian]{babel}
\usepackage{amsmath,amsfonts,amssymb,amsthm,mathtools}
\usepackage{icomma}
\usepackage{euscript}
\usepackage{mathrsfs}
\usepackage[dvipsnames]{xcolor}
\usepackage[left=2cm,right=2cm,
    top=2cm,bottom=2cm,bindingoffset=0cm]{geometry}
\usepackage[normalem]{ulem}
\usepackage{graphicx}
\usepackage{makeidx}
\makeindex
\graphicspath{{pictures/}}
\DeclareGraphicsExtensions{.pdf,.png,.jpg}
%\usepackage[usenames]{color}
\hypersetup{
     colorlinks=true,
     linkcolor=magenta,
     filecolor=magenta,
     citecolor=black,      
     urlcolor=magenta,
     }
\usepackage{fancyhdr}
\pagestyle{fancy} 
\fancyhead{} 
\fancyhead[LE,RO]{\thepage} 
\fancyhead[CO]{\hyperlink{uk}{к списку объектов}}
\fancyhead[LO]{\hyperlink{sod}{к содержанию}} 
\fancyfoot{}
\newtheoremstyle{indented}{0 pt}{0 pt}{\itshape}{}{\bfseries}{. }{0 em}{ }

\renewcommand\thesection{}
\renewcommand\thesubsection{}

%\geometry{verbose,a4paper,tmargin=2cm,bmargin=2cm,lmargin=2.5cm,rmargin=1.5cm}

\title{Дифференциальная геометрия}
\author{%Кабашный Иван (@keba4ok) \\ 
    %на основе хуйни
    }
\date{%01 ноября 2002г.
}


%envirnoments
    \theoremstyle{indented}
    \newtheorem{theorem}{Теорема}
    \newtheorem{lemma}{Лемма}
    \newtheorem{alg}{Алгоритм}

    \theoremstyle{definition} 
    \newtheorem{defn}{Определение}
    \newtheorem*{exl}{Пример(ы)}
    \newtheorem{prob}{Задача}

    \theoremstyle{remark} 
    \newtheorem*{remark}{Примечание}
    \newtheorem*{cons}{Следствие}
    \newtheorem*{exer}{Упражнение}
    \newtheorem*{stat}{Утверждение}
%esli ne hochetsa numeracii - nuzhno prisunut' zvezdochku-pezsochku

\definecolor{coralpink}{rgb}{0.97, 0.51, 0.47}

%declarations
        %arrows_shorten
            \DeclareMathOperator{\la}{\leftarrow}
            \DeclareMathOperator{\ra}{\rightarrow}
            \DeclareMathOperator{\lra}{\leftrightarrow}
            \DeclareMathOperator{\llra}{\longleftrightarrow}
            \DeclareMathOperator{\La}{\Leftarrow}
            \DeclareMathOperator{\Ra}{\Rightarrow}
            \DeclareMathOperator{\Lra}{\Leftrightarrow}
            \DeclareMathOperator{\Llra}{\Longleftrightarrow}

        %letters_different
            \DeclareMathOperator{\CC}{\mathbb{C}}
            \DeclareMathOperator{\ZZ}{\mathbb{Z}}
            \DeclareMathOperator{\RR}{\mathbb{R}}
            \DeclareMathOperator{\NN}{\mathbb{N}}
            \DeclareMathOperator{\HH}{\mathbb{H}}
            \DeclareMathOperator{\LL}{\mathscr{L}}
            \DeclareMathOperator{\KK}{\mathscr{K}}
            \DeclareMathOperator{\GA}{\mathfrak{A}}
            \DeclareMathOperator{\GB}{\mathfrak{B}}
            \DeclareMathOperator{\GC}{\mathfrak{C}}
            \DeclareMathOperator{\GD}{\mathfrak{D}}
            \DeclareMathOperator{\GN}{\mathfrak{N}}
            \DeclareMathOperator{\Rho}{\mathcal{P}}
            \DeclareMathOperator{\FF}{\mathcal{F}}

        %common_shit
            \DeclareMathOperator{\Ker}{Ker}
            \DeclareMathOperator{\Frac}{Frac}
            \DeclareMathOperator{\Imf}{Im}
            \DeclareMathOperator{\cont}{cont}
            \DeclareMathOperator{\id}{id}
            \DeclareMathOperator{\ev}{ev}
            \DeclareMathOperator{\lcm}{lcm}
            \DeclareMathOperator{\chard}{char}
            \DeclareMathOperator{\codim}{codim}
            \DeclareMathOperator{\rank}{rank}
            \DeclareMathOperator{\ord}{ord}
            \DeclareMathOperator{\End}{End}
            \DeclareMathOperator{\Ann}{Ann}
            \DeclareMathOperator{\Real}{Re}
            \DeclareMathOperator{\Res}{Res}
            \DeclareMathOperator{\Rad}{Rad}
            \DeclareMathOperator{\disc}{disc}
            \DeclareMathOperator{\rk}{rk}
            \DeclareMathOperator{\const}{const}
            \DeclareMathOperator{\grad}{grad}
            \DeclareMathOperator{\Aff}{Aff}
            \DeclareMathOperator{\Lin}{Lin}
            \DeclareMathOperator{\Prf}{Pr}
            \DeclareMathOperator{\Iso}{Iso}

        %specific_shit
            \DeclareMathOperator{\Tors}{Tors}
            \DeclareMathOperator{\form}{Form}
            \DeclareMathOperator{\Pred}{Pred}
            \DeclareMathOperator{\Func}{Func}
            \DeclareMathOperator{\Const}{Const}
            \DeclareMathOperator{\arity}{arity}
            \DeclareMathOperator{\Aut}{Aut}
            \DeclareMathOperator{\Var}{Var}
            \DeclareMathOperator{\Term}{Term}
            \DeclareMathOperator{\sub}{sub}
            \DeclareMathOperator{\Sub}{Sub}
            \DeclareMathOperator{\Atom}{Atom}
            \DeclareMathOperator{\FV}{FV}
            \DeclareMathOperator{\Sent}{Sent}
            \DeclareMathOperator{\Th}{Th}
            \DeclareMathOperator{\supp}{supp}
            \DeclareMathOperator{\Eq}{Eq}
            \DeclareMathOperator{\Prop}{Prop}


%env_shortens_from_hirsh            
    \newcommand{\bex}{\begin{example}\rm}
    \newcommand{\eex}{\end{example}}
    \newcommand{\ba}{\begin{algorithm}\rm}
    \newcommand{\ea}{\end{algorithm}}
    \newcommand{\bea}{\begin{eqnarray*}}
    \newcommand{\eea}{\end{eqnarray*}}
    \newcommand{\be}{\begin{eqnarray}}
    \newcommand{\ee}{\end{eqnarray}}
    \newcommand{\abs}[1]{\lvert#1\rvert}
        \newcommand{\bp}{\begin{prob}}
        \newcommand{\ep}{\end{prob}}
    
\begin{document}
%ya_ebanutyi
\newcommand{\resetexlcounters}{%
  \setcounter{exl}{0}%
} 
\newcommand{\resetremarkcounters}{%
  \setcounter{remark}{0}%
} 
\newcommand{\reseconscounters}{%
  \setcounter{cons}{0}%
} 
\newcommand{\resetall}{%
    \resetexlcounters
    \resetremarkcounters
    \reseconscounters%
}
\newcommand{\cursed}[1]{\textit{\textcolor{coralpink}{#1}}}
\newcommand{\de}[3][2]{\index{#2}{\textbf{\textcolor{coralpink}{#3}}}}
\newcommand{\re}[3][2]{\hypertarget{#2}{\textbf{\textcolor{coralpink}{#3}}}}
\newcommand{\se}[3][2]{\index{#2}{\textit{\textcolor{coralpink}{#3}}}}
\maketitle
\newpage
\hypertarget{sod}
\tableofcontents
\newpage
%main_content


\section{Разделы курса.}

\noindent
\textcolor{blue}{Алгебраическая топология}
\begin{itemize}
    \item Фундаментальная группа.
    \item Накрытия.
    \item Приложения.
\end{itemize}

\noindent
\textcolor{blue}{Дифференциальная геометрия}
\begin{itemize}
    \item Гладкие кривые и поверхности.
    \item Гладкие многообразия.
    \item Римановы многообразия.
\end{itemize}

\noindent
\textcolor{blue}{Литература}
\begin{itemize}
    \item Виро и др – Элементарная топология.
    \item Munkres – Topology
\end{itemize}



\section{Лекция 1.}
% похожая лекция:
% https://pdmi.ras.ru/~svivanov/uni/2020_2/lecture25_notes.pdf

\begin{defn}
    \se{Ретракция}{Ретракция} — непрерывное отображение $f : X \to A$, где $A \subset X$, такое, что $f |_A = id_A$. \\
        Если существует ретракция $f : X \to A$, то $A$ называется ретрактом пространства $X$.
\end{defn}

\begin{exl}\
    
    \begin{itemize}
        \item Всякое одноточечное подмножество является ретрактом.
        \item Никакое двухточечное подмножество прямой не является её ретрактом.
    \end{itemize}
\end{exl}



\begin{theorem}
    Подмножество $A$ топологического просmрансmва $X$ является его 
    ретрактом $\Llra$ всякое нenpepывнoe отображение $g : A \to Y$ в 
    произвольное пространство $Y$ можно продолжить до нenpepывного
    отображения $X \to Y$.
\end{theorem}

\begin{proof}\
    
    \noindent
    ''$\Ra$'' пусть $\rho: X \to A$ ретракция, тогда $g\circ\rho$ искомое продолжение. \\
    (композиция непрерывных отображений непрерывна; действует так: $X\to A \to Y$; на множестве $A$: $g\circ\rho|_A = g\circ id_A = g$).\\
    ''$\La$'' рассмотрим $\rho$ - непрерывное продолжение $g=id_A$ ($Y=A$), тогда $\rho$ - ретракция. ($\rho: X \to A$ непрерывно, $\rho|_A = id_A$)  
\end{proof}

$A \subset X, in: A \to X$ - включение $(\forall a \in A:\ in(a)=a)$.
\begin{lemma}
    Если $\rho: X \to A$– ретракция, $in: A \to X$ – включение и $x_0 \in A$, то
    \begin{itemize}
        \item $\rho_*: \pi_1(X,x_0) \to \pi_1(A,x_0)$ - сюръекция;
        \item $in_*: \pi_1(A,x_0) \to \pi_1(X,x_0)$ - инъекция;
    \end{itemize}
\end{lemma}

\begin{proof}
   $$ \rho \circ in = id \Ra (\rho \circ in)_* = \rho_* \circ in_* = id_*$$
\end{proof}

\begin{theorem}
    (\se{Теорема!Борсука (в размерности 2)}{Теорема Борсука (в размерности 2)}) \\
    Не существует ретракции из $D^2$ на $S^1$.
\end{theorem}

\begin{proof}
    От противного. Пусть $\rho: D^2 \to S^1$ – ретракция, $x_0 \in S^1$.
$\pi_1(D^2,x_0)=\mathbb{Z}, \pi_1(S^1,x_0)=0$, тогда по лемме $in: \mathbb{Z} \to 0$ - инъекция. Противоречие.
\end{proof}

\begin{defn}
    Точка $a\in X$ называется \se{Неподвижная точка отображения}{неподвижной точкой отображения} \\
    $f: X \to X$, если $f(a)=a$.
\end{defn}

\begin{remark}
    Говорят, что пространство обладает свойством неnодвижной точки,
    если всякое непрерывное отображение $f: X \to X$ имеет неподвижную
    точку.
\end{remark}

\begin{exl}
    Отрезок $[a;b]$ обладает свойством неподвижной точки.
\end{exl}

\begin{theorem}
    (\se{Теорема!Брауэра о неподвижной точке}{Теорема Брауэра о неподвижной точке}) 
    Любое непрерывное отображение $f: D^2 \to D^2$ имеет неподвижную точку.
\end{theorem}

\begin{proof}
    От противного, пусть $f (x) \not= x$ для всех $x \in D^2$.\\
Построим $g : D^2 \to S^1$ так: $g(x)$ - точка пересечения луча,
начинающегося в $f(x)$ и проходящего через $x$, с окружностью.
Это $g$ противоречит теореме Борсука.
$$|(x-f(x))t+x|=1$$
\end{proof}

\begin{defn}
    $X$ и $Y$ гомотопически эквивалентны ($X \sim Y$), если существуют непрерывные отображения $f : X \to Y$ и $g : Y \to X$ такие, что $g \circ f \sim id_X$ и $f \circ g \sim id_Y$. Такие $f$ и $g$ называются \se{Гомотопически обратные отображения}{гомотопически обратными отображениями}.
    Каждое из $f$ и $g$ называется \se{Гомотопическая эквивалентность}{гомотопической эквивалентностью}.
\end{defn}

\begin{remark}
    Отображения бывают гомотопными, а пространства - гомотопически
эквивалентными.
\end{remark}

\begin{exl}
    $\mathbb{R}^n$ гомотопически эквивалентно $\{0\}$.
\end{exl} 

\begin{defn}
    Ретракция $f : X \to A$ называется \se{Ретракция!деформационная}{деформационной ретракцией}, если её композиция с включением $in: A \to X$ гомотопна тождественному отображению, т.е. $$in \circ f \sim id_X$$
Если существует деформационная ретракция $X$ на $A$, то $A$ называется деформационным ретрактом пространства $X$
\end{defn}

\begin{theorem}
    Деформационная ретракция является гомотопической эквивалентностью.
\end{theorem}

\begin{proof}
    Деформационная ретракция и включение – гомотопически
обратные отображения.
\end{proof}

\begin{exl}\

    \begin{itemize}
        \item $\mathbb{R}^n \backslash \{0\} \sim S^{n-1}$. ($f: \mathbb{R}^n \backslash \{0\} \to S^{n-1},  f(x)=\frac{x}{|x|}$).
        \item Лента Мёбиуса (или кольцо) $\sim S^1$.
        \item Плоскость без $n$ точек $\sim$ букет $n$ окружностей.
        \item Тор с дыркой $\sim$ букет двух окружностей.
    \end{itemize}
\end{exl}

\begin{remark}
    В примерах правое пространство - деформационный ретракт левого.
\end{remark}

\begin{theorem}
    Гомотопическая эквивалентность — отношение эквивалентности
между топологическими пространствами.
\end{theorem}

\begin{proof}\

    Рефлексивность $X \sim X$: $f=g=id_X$.
    
    Симметричность $X \sim Y \Ra Y \sim X$: ($f\circ g$ и $g\circ f$) $\to$ ($g\circ f$ и $f\circ g$).
    
    Транзитивность: $f_{1}: X\to Y, g_{1}: Y\to X$ гомотопически обратны, $f_{2}: Y\to Z$, \\$g_{2}: Z\to Y$ гомотопически обратны $\Ra$ $f_{2} \circ f_{1}$ и $g_{1} \circ g_{2}$ гомотопически обратны, т.к.:
    $$ f_{2} \circ (f_{1} \circ g_{1}) \circ g_{2} 
    \sim f_{2} \circ id_Y \circ g_{2} \sim  f_{2} \circ g_{2} 
    \sim id_Z$$
    $$g_{1} \circ (g_{2} \circ f_{2}) \circ f_{1}
    \sim g_{1} \circ id_Y \circ f_{1} \sim  g_{1} \circ f_{1} 
    \sim id_X$$
\end{proof}

\begin{defn}
    Класс пространств, гомотопически эквивалентных данному X, называется его \se{Гомотопический тип}{гомотопическим типом}.
    Свойства (характеристики) топологических пространств, одинаковые у гомотопически эквивалентных, - \se{Гомотопические свойства(инварианты)}{гомотопические свойства (гомотопические инварианты)}.
\end{defn}

\begin{exer}
    Число компонент линейной связности - гомотопический инвариант.
\end{exer}

%nuzhno dopisat lekciyou 1 and 2 
\section{Лекция 2.}
\newpage
%2ая лекция
Изоморфмность фундаментальных групп
\begin{theorem}
    Гомотопическая эквивалентность индуцирует изоморфизм фундаментальных групп.
\end{theorem}

\begin{lemma}
    Пусть $f,g: X \to Y$ - гомотопные отображения, $H: X \times I \to Y$ - гомотопия между ними. $f(x_0)=y_0, g(x_0) = y_1, \gamma (t) = H(x_0, t)$ - путь от $y_0$ к $y_1$. Тогда $g_* = T_\gamma \circ f_*$
\end{lemma}

\begin{defn}
    Топологическое пространтсов X стягиваемо, если оно гомотопически эквивалентно точке.
\end{defn}

\begin{exl} Переформулировки стягиваемости:
    \begin{itemize}
        \item тождественное отображение гомотопно постоянному
        \item некоторая точка - деформационный ретракт
    \end{itemize}
\end{exl}

\begin{lemma}
    Пусть $h: S^1 \to X$ - непрерывное отображение. Следующие утверждения эквивалентны:
    \begin{enumerate}
        \item $h$ гомотопно постоянному отображению.
        \item $h$ продолжается до непрерывного от. $D^2 \to X$.
        \item $h_*$ - тривиальный гомоморфизм фундаментальных групп.
    \end{enumerate}
\end{lemma}

%ukazatel'. chto ne vidno blyat'?
\newpage
\hypertarget{dex}
    \printindex
%staryi_variant
%\hypertarget{uk}{Основные понятия.}
%\begin{multicols}{2}
%    \hyperlink{}{} \ 
%\end{multicols}
%novyi_variant
\end{document}