\documentclass[a4paper,100pt]{article}

\usepackage[utf8]{inputenc}
\usepackage[unicode, pdftex]{hyperref}
\usepackage{cmap}
\usepackage{mathtext}
\usepackage{multicol}
\setlength{\columnsep}{1cm}
\usepackage[T2A]{fontenc}
\usepackage[english,russian]{babel}
\usepackage{amsmath,amsfonts,amssymb,amsthm,mathtools}
\usepackage{icomma}
\usepackage{euscript}
\usepackage{mathrsfs}
\usepackage{geometry}
\usepackage[usenames]{color}
\hypersetup{
     colorlinks=true,
     linkcolor=magenta,
     filecolor=green,
     citecolor=black,      
     urlcolor=cyan,
     }
\usepackage{fancyhdr}
\pagestyle{fancy} 
\fancyhead{} 
\fancyhead[LE,RO]{\thepage} 
\fancyhead[CO]{\hyperlink{t2}{к списку объектов}}
\fancyhead[LO]{\hyperlink{t1}{к содержанию}} 
\fancyfoot{}
\newtheoremstyle{indented}{0 pt}{0 pt}{\itshape}{}{\bfseries}{. }{0 em}{ }

%\geometry{verbose,a4paper,tmargin=2cm,bmargin=2cm,lmargin=2.5cm,rmargin=1.5cm}

\title{Геометрия и топология. Факты 2 сем.}
\author{Кабашный Иван (@keba4ok)\\ \\ (по материалам лекций Фоминых Е. А.,\\ практик, а также других источников)}
\date{26 марта 2021 г.}

\theoremstyle{indented}
\newtheorem{theorem}{Теорема}
\newtheorem{lemma}{Лемма}

\theoremstyle{definition} 
\newtheorem{defn}{Определение}
\newtheorem{exl}{Пример(ы)}

\theoremstyle{remark} 
\newtheorem{remark}{Примечание}
\newtheorem{cons}{Следствие}
\newtheorem{stat}{Утверждение}

\DeclareMathOperator{\la}{\leftarrow}
\DeclareMathOperator{\ra}{\rightarrow}
\DeclareMathOperator{\lra}{\leftrightarrow}
\DeclareMathOperator{\La}{\Leftarrow}
\DeclareMathOperator{\Ra}{\Rightarrow}
\DeclareMathOperator{\Lra}{\Leftrightarrow}
\DeclareMathOperator{\Llra}{\Longleftrightarrow}
\DeclareMathOperator{\Ker}{Ker}
\DeclareMathOperator{\Tors}{Tors}
\DeclareMathOperator{\Frac}{Frac}
\DeclareMathOperator{\Imf}{Im}
\DeclareMathOperator{\Real}{Re}
\DeclareMathOperator{\cont}{cont}
\DeclareMathOperator{\id}{id}
\DeclareMathOperator{\ev}{ev}
\DeclareMathOperator{\lcm}{lcm}
\DeclareMathOperator{\chard}{char}
\DeclareMathOperator{\CC}{\mathbb{C}}
\DeclareMathOperator{\ZZ}{\mathbb{Z}}
\DeclareMathOperator{\RR}{\mathbb{R}}
\DeclareMathOperator{\NN}{\mathbb{N}}
\DeclareMathOperator{\PP}{\mathbb{P}}
\DeclareMathOperator{\FF}{\mathcal{F}}
\DeclareMathOperator{\Rho}{\mathcal{P}}
\DeclareMathOperator{\codim}{codim}
\DeclareMathOperator{\rank}{rank}
\DeclareMathOperator{\ord}{ord}
\DeclareMathOperator{\adj}{adj}
\DeclareMathOperator{\const}{const}
\DeclareMathOperator{\grad}{grad}
\DeclareMathOperator{\Aff}{Aff}
\DeclareMathOperator{\Lin}{Lin}
\DeclareMathOperator{\Prf}{Pr}
\DeclareMathOperator{\Iso}{Iso}

\begin{document}

\newcommand{\resetexlcounters}{%
  \setcounter{exl}{0}%
} 

\newcommand{\resetremarkcounters}{%
  \setcounter{remark}{0}%
} 

\newcommand{\reseconscounters}{%
  \setcounter{cons}{0}%
} 

\newcommand{\resetall}{%
    \resetexlcounters
    \resetremarkcounters
    \reseconscounters%
}

\maketitle 

\newpage

\hypertarget{t1}{Основные} (по моему мнению) определения и факты из топологии (на самом деле, почти всё, что можно).
\tableofcontents

\newpage


\section{Аффинные пространства.}

\subsection{Начальные определения и свойства.}

\begin{defn}
    \textit{\textcolor{magenta}{\hypertarget{s1}{Аффинное пространство}}} - тройка $(X, \vec{X}, +)$, состоящая из непустого множества \textit{\textcolor{magenta}{\hypertarget{s2}{точек}}}, векторного пространства над $\RR$ (\textit{\textcolor{magenta}{\hypertarget{s3}{присоединённое}}}) и операцией $+:X\times \vec{X}\ra X$ \textit{\textcolor{magenta}{\hypertarget{s4}{откладывания вектора}}}. \ 

    Налагаемые условия - для любых точек $x, y\in X$ существует единственный вектор $v\in \vec{X}$ такой, что $x+v=y$ ($\vec{xy}$), а также ассоциативность откладывания вектора.
\end{defn}

\begin{defn}
    \textit{\textcolor{magenta}{\hypertarget{s5}{Начало отсчёта}}} аффинного пространства - произвольная фиксированная точка $o\in X$. 
\end{defn}

\begin{lemma}
    Начало отсчёта $o\in X$ задаёт биекцию $\varphi_o: X\ra \vec{X}$ по правилу:
    \[
        \varphi_o(x)=\vec{ox} \: \forall x \in X. 
    \]
    Такая биекция называется \textit{\textcolor{magenta}{\hypertarget{s6}{векторизацией}}} аффинного пространства.
\end{lemma}

\begin{defn}
    \textit{\textcolor{magenta}{\hypertarget{s7}{Линейная комбинация}}} $\sum t_i p_i$ точек с коэффициентами относительно начала отсчёта $o\in X$ - вектор $v = \sum t_i \vec{op_i}$, или точка $p=o+v$. Комбинация называется \textit{\textcolor{magenta}{\hypertarget{s8}{барицентрической}}}, если сумма коэффициентов равна единице, и \textit{\textcolor{magenta}{\hypertarget{s9}{сбалансированной}}}, если сумма коэффициентов равна нулю.
\end{defn}

\begin{theorem}
    Барицентрическая комбинация точек - точка, не зависящая от начала отсчёта. Сбалансированная комбинация точек - вектор, не зависящий от начала отсчёта.
\end{theorem}

\subsection{Материальные точки.}

\begin{defn}
    Пусть $x$ — некоторая точка аффинного пространства и $m$ — ненулевое число. \textit{\textcolor{magenta}{\hypertarget{s10}{Материальной точкой}}} $(x,m)$ называется пара: точка $x$ с вещественным числом $m$, причем число m называется \textit{\textcolor{magenta}{\hypertarget{s11}{массой}}} материальной точки $(x,m)$, а точка $x$ — носителем этой материальной точки.
\end{defn}

\begin{defn}
    \textit{\textcolor{magenta}{\hypertarget{s12}{Центром масс}}} системы материальных точек $(x_i, m_i)$ называется такая точка $z$ (притом единственная), для которой имеет место равенство
    \[
        m_1 \cdot \vec{zx_1} + \ldots + m_n \cdot \vec{zx_n} = 0. 
    \]
\end{defn}

\subsection{Аффинные подпространства и оболочки.}

\begin{defn}
    Множество $Y\subset X$ - \textit{\textcolor{magenta}{\hypertarget{s13}{аффинное подпространство}}}, если существуют такие линейное подпространство $V\subset \vec{X}$ и точка $p\in Y$, что $Y=p+V$. $V$ называется\textit{\textcolor{magenta}{\hypertarget{s14}{направлением}}} $Y$. Определение подпространства не зависит от выбора точки в нём.
\end{defn}

\begin{defn}
    \textit{\textcolor{magenta}{\hypertarget{s15}{Размерность}}} $\dim X$ афинного пространства есть размерность его присоединённого векторного пространства.
\end{defn}

\begin{defn}
    \textit{\textcolor{magenta}{\hypertarget{s16}{Параллельный перенос}}} на вектор $v\in \vec{X}$ - отображение $T_v:X\ra X$, заданное равенством $T_v(x)=x+v$. 
\end{defn}

\begin{defn}
    Аффинные подпространства одинаковой размерности \textit{\textcolor{magenta}{\hypertarget{s17}{параллельны}}}, если их направления совпадают.
\end{defn}

\begin{defn}
    \textit{\textcolor{magenta}{\hypertarget{s18}{Прямая}}} - аффинное подпространство размерности 1, \textit{\textcolor{magenta}{\hypertarget{s19}{гиперплоскость}}} в $X$ - аффинное подпространство размерности $\sim X - 1$. 
\end{defn}

\begin{stat}
    Две различные гиперплоскости не пересекаются тогда и только тогда, когда они параллельны.
\end{stat}

\begin{defn}
    \textit{\textcolor{magenta}{\hypertarget{s20}{Суммой аффинных подпространств}}} называется наименьшее аффинное подпространство, их содержащее.
\end{defn}

\begin{theorem}
    Пересечение любого набора аффинных подпространств - либо пустое мноежство, либо аффинное подпространство.
\end{theorem}

\begin{defn}
    \textit{\textcolor{magenta}{\hypertarget{s21}{Аффинная оболочка}}} $\Aff{A}$ непустого множества $A\subset X$ - пересечение всех аффинных подпространств, содержащих $A$. Как следствие, это - наименьшее аффинное подпространство, содержащее $A$. 
\end{defn}

\begin{theorem}
    $\Aff(A)$ - множество всех барицентрических комбинаций точек из $A$. 
\end{theorem}

\begin{defn}
    Точки $p_1, \ldots, p_k$ \textit{\textcolor{magenta}{\hypertarget{s22}{аффинно зависимы}}}, если существуют такие коэффициенты $t_i \in \RR$, не все равные нулю, что $\sum t_i = 0$ и $\sum t_ip_i = 0$. Если такой комбинации нет, то точки \textit{\textcolor{magenta}{\hypertarget{s23}{аффинно независимы}}}.
\end{defn}

\begin{theorem}
    (Переформулировки аффинной независимости.) Для $p_1, \ldots, p_k \in X$ следующие свойства эквивалентны: 

    \begin{itemize}
        \item они аффинно независимы; 
        \item векторы $p_1 p_i$, $i\in \{2, 3, \ldots, k\}$, линейно независимы; 
        \item $\dim \Aff (p_1, \ldots, p_k)=k-1$; 
        \item каждая точка из $\Aff(p_1, \ldots, p_k)$ единственным образом представляется в виде барицентрической комбинации $p_i$. 
    \end{itemize}
\end{theorem}

\subsection{Базисы и отображения.}

\begin{defn}
    \textit{\textcolor{magenta}{\hypertarget{s24}{Аффинный базис}}} - набор $n+1$ точке в $X$, пространстве размерности $n$, являющийся аффинно независимым. Или же, это - точке $o\in X$ и базис $e_0, \ldots, e_n$ пространства $\vec{X}$. 
\end{defn}

\begin{defn}
    Каждая точка однозначно записывается в виде барицентрической комбинации $\sum_{i=0}^n t_i e_i$, а числа $t_i$ называют \textit{\textcolor{magenta}{\hypertarget{s25}{барицентрическими координатами}}} этой точки.
\end{defn}

\begin{defn}
    (Говно-определение). Отображение $F:X\ra Y$ называется \textit{\textcolor{magenta}{\hypertarget{s26}{аффинным}}}, если отображение $\tilde{F}_p$ линейно для некоторой точки $p\in X$. Отображение $\tilde{F}_p:\vec{X}\ra\vec{Y}$ индуцируется из любого отображения $F:X\ra Y$ посредством формулы $\forall v\in \vec{X}$ $\tilde{F}_p(v)=\overrightarrow{F(p)F(q)}$, где $q=p+v$. 
\end{defn}

\begin{defn}
    Отображение $\tilde{F}$ называется \textit{\textcolor{magenta}{\hypertarget{s27}{линейной частью}}} аффинного отображения $F$.
\end{defn}

\begin{defn}
    (Нормальное определение.) Отображение $F:X\ra Y$ называется \textit{\textcolor{magenta}{\hypertarget{s28}{аффинным}}}, если существует такое линейное $L:\vec{X}\ra\vec{Y}$, что для любых $q, p\in X$, $\overrightarrow{F(p)F(q)}=L(\vec{pq})$. 
\end{defn}

\begin{theorem}
    Пусть $x\in X$, $y\in Y$, $L:\vec{X}\ra \vec{Y}$ линейно. Тогда существует единственное аффинное отображение $F:X\ra Y$ такое, что $\tilde{F}=L$ и $F(x)=y$. 
\end{theorem} \

\begin{lemma}
    Пусть $p_1, \ldots, p_n$ - аффинно независимые точки в аффинном пространстве $X$, $q_1, \ldots, q_n$ - точки в аффинном пространстве $Y$. Тогда существует такое аффинное отображение $F:X\ra Y$, что $F(p_i)=q_i$ $\forall i$. Кроме того, если $\dim X = n-1$, то такое отображение единственно.
\end{lemma} \ 

\begin{lemma}
    Аффинное отображение сохраняет барицентрические комбинации.
\end{lemma} \ 

\begin{lemma}
    Композиция аффинных отображений - аффинное отображение. При этом линейная часть композиции - композиция линейных частей.
\end{lemma}

\begin{stat}
    Образ и прообраз аффинного подпространства - аффинное подпространство. Образы (прообразы) параллельных подпространств параллельны.
\end{stat}

\begin{theorem}
    Параллельный перенос - аффинное отображение, его линейная часть тождественна. Верно также и обратное.
\end{theorem}

\begin{defn}
    Аффинное отображение $F:X\ra X$ такое, что $\tilde{F}=k \id$ для некоторого $k \in \RR\backslash \{0, 1\}$, называется \textit{\textcolor{magenta}{\hypertarget{s29}{гомотетией}}}, а $k$ называют \textit{\textcolor{magenta}{\hypertarget{s30}{коэффициентом растяжения}}} гомотетии $F$. Такое отображение имеет ровно одну неподвижную точку, называемую центром.
\end{defn}

\begin{theorem}
    (\textit{\textcolor{magenta}{\hypertarget{s31}{Основная теорема аффинной геометрии.}}}) Пусть $X, Y$ - аффинные пространства, $\dim X \geq 2$. Пусть $F: X\ra Y$ - инъективное отображение, и для любой прямой $l\subset X$ её образ $F(l)$ - тоже прямая. Тогда $F$ - аффинное отображение.
\end{theorem}

\section{Проективные пространства.}

\subsection{Начальные определения и свойства.}

\begin{defn}
    Пусть $V$ - векторное пространство над полем $K$. На множестве $V \backslash \{0\}$ введём отношение эквивалентности
    \[
        x \sim y \Llra \exists \lambda \in K: x = \lambda y. 
    \]
    Тогда фактор $V$ по этому отношению называют \textit{\textcolor{magenta}{\hypertarget{s32}{проективным пространством}}} ($\PP(V)$), порождённым векторным $V$. Само отображение из векторного пространства в соответствующее проективное называют \textit{\textcolor{magenta}{\hypertarget{s33}{проективизацией}}}.
\end{defn}

\begin{remark}
    \textit{\textcolor{magenta}{\hypertarget{s34}{Размерность}}} $\PP(V)$ по определению равна $\dim V -1$. 
\end{remark}

\begin{theorem}
    Пусть $Y, Z\subset X$ - подпространства, $\dim Y + \dim Z \geq \dim X$, тогда

    \begin{itemize}
        \item $Y\cap Z \neq \emptyset$; 
        \item $Y \cap Z$ - подпространство; 
        \item $\dim(Y \cap Z)\geq \dim Y + \dim Z - \dim X$. 
    \end{itemize}
\end{theorem}

\begin{defn}
    Пусть $W$ - непустое векторное подпространство $V$. Тогда $\PP(W)$ называется \textit{\textcolor{magenta}{\hypertarget{s35}{проективным подпространством}}} $\PP(V)$.
\end{defn}

\begin{defn}
    Пусть $X=\PP(V)$ - проективное пространство размерности $n$. Числа $x_0, x_1, \ldots, x_n$, являющиеся координатами вектора $v$, порождающего $p\in \PP(V)$, называются \textit{\textcolor{magenta}{\hypertarget{s36}{однородными координатами}}}.
\end{defn}

\begin{defn}
    $\hat{X}=\PP(V)$ - \textit{\textcolor{magenta}{\hypertarget{s37}{проективное пополнение}}} аффинного пространства $X$, а множество $X_\infty = \PP(\vec{X}\times 0)\subset \hat{X}$ - \textit{\textcolor{magenta}{\hypertarget{s38}{бесконечно удалённые точки}}}. Также, множество этих точек есть гиперплоскость в $\hat{X}$, которая называется \textit{\textcolor{magenta}{\hypertarget{s39}{бесконечно удалённой гиперплоскостью}}}.
\end{defn}

\begin{defn}
    Пусть $V$ - векторное пространство, $W\subset V$ - линейная гиперплоскость, $X$ - гиперплоскость ей пареллельная. Тогда биекцию $\PP(V) \backslash \PP(W) \ra X$ называют \textit{\textcolor{magenta}{\hypertarget{s40}{картой}}} пространства $\PP(V)$.
\end{defn}

\begin{defn}
    Пусть на $\RR p^1$ (прямая с бесконечно удалённой точкой) выбрана аффинная система координат, в которой $A=a$, $B=b$, $C=c$ и $D=d$. Определим \textit{\textcolor{magenta}{\hypertarget{s41}{двойное отношение}}} четвёрки точек $(A, B, C, D)$ формулой
    \[
        [A, B, C, D]=\frac{a-c}{a-d}\cdot \frac{b-c}{b-d}.
    \]
\end{defn}

\begin{stat}
    Данное определение инвариантно относительно выбора карты, а само отношение сохраняется при проективных преобразованиях.
\end{stat}

\subsection{Проективные отображения.}

\begin{lemma}
    Пусть $V, W$ - векторные пространства и $L:V\ra W$ - инъективное линейное отображение. Тогда существует единственное отображение $F:\PP(V) \ra \PP(W)$ такое, что 
    \[
        P_W\circ L = F \circ P_V, 
    \]
    где $P_V$, $P_W$ - проекции из $V \backslash \{0\}$ и $W \backslash \{0\}$ в $\PP(V)$ и $\PP(W)$ соответственно.
\end{lemma}

\begin{defn}
    Отображение $F$ из леммы выше называется \textit{\textcolor{magenta}{\hypertarget{s42}{проективизацией}}} $L$, и обозначается как $F=\PP(L)$. 
\end{defn}

\begin{defn}
    Отображение из $\PP(V)$ в $\PP(W)$ - \textit{\textcolor{magenta}{\hypertarget{s43}{проективное}}}, если оно является проективизацией некоторого линейного $L:V\ra W$. 
\end{defn}

\begin{stat}
    Проективное отображение переводит проективные подпространства (в том числе, всё пространство) в проективные пространства той же размерности.
\end{stat}

\begin{theorem}
    Пусть $X, Y$ - аффинные пространства, $\hat{X}$, $\hat{Y}$ - их проектиыне пополнения, $F: X\ra Y$ - инъективное аффинное отображение. Тогда существует единственное проективное отображение $\hat{F}: \hat{X} \ra \hat{Y}:\hat{F}|_X = F$. Причём оно переводит бескоречно удалённые в бесконечно удалённые.
\end{theorem}

\begin{defn}
    Пусть $H_1, H_2 \subset X$ - гиперплоскости ($X$ - проективное), $p \in X \backslash (H_1 \cup H_2)$. \textit{\textcolor{magenta}{\hypertarget{s44}{Центральная проекция}}} $H_1$ и $H_2$ с центром $p$ - проективное отображение $F:H_1 \ra H_2$, определяемое так: пусть $x\in H_1$, тогда $F(x)$ - точка пересечения прямой $(px)$ и гиперплоскости $H_2$.
\end{defn}

\begin{defn}
    Пусть $X = \PP(V)$ - проективное пространство, размерности $n$. \textit{\textcolor{magenta}{\hypertarget{s45}{Проективный базис}}} $X$ - набор из $n+2$ точек, никакие $n+1$ из которых не лежат в одной проективной гиперплоскости.
\end{defn}

\begin{lemma}
    Можно выбрать такие векторы $v_1, \ldots, v_{n+2}\in V\backslash \{0\}$, порождающие проективный базис $p_1, \ldots, p_{n+2}$, что $v_{n+2}=\sum_{i=1}^{n+1}v_i$. 
\end{lemma} \

\begin{theorem}
    Пусть $X, Y$ - проективные пространства, размерностей $n$, $p_1, \ldots, p_{n+2}\in X$ и $q_1, \ldots, q_{n+2}\in Y$ - проективные базисы. Тогда существует единственное проективное отображение $F: X\ra Y$ такое, что $F(p_i)=q_i$ для всех $i$. 
\end{theorem}

\begin{remark}
    Проективное преобразование прямой с бесконечно удалёнными точками имеет вид 
    \[
        [x:y]\mapsto [ax+by:cx+dy], 
    \]
    где $a, b, c, d\in \RR$, $ad-bc \neq 0$. Для $\hat{\RR}$ это - дробно-линейная функция
    \[
        f(x)=\frac{ax+b}{cx+d}. 
    \]
\end{remark}

\subsection{Проективные и аффинные теоремы.}

\begin{theorem}
    (\textit{\textcolor{magenta}{\hypertarget{s46}{Теорема Паппа (аффинная)}}}). Пусть $X$ - аффинная плоскость, $l$, $l'$ - рразличные прямые в $X$. $x, y, z\in l$, $x', y', z'\in l'$ - различные точки, отличные от $l \cap l'$. Тогда из $(xy')||(x'y)$, $(yz')||(y'z)$ следует, что $(xz')||(x'z)$. 
\end{theorem} \ 

\begin{theorem}
    (\textit{\textcolor{magenta}{\hypertarget{s47}{Теорема Паппа (проективная)}}}). Пусть $\PP(E)$ - проективная плоскость, $l, l'$ - различные прямые в $\PP(E)$, $a, b, c\in l$, $a', b', c'\in l'$ - различные точки, отличные от $l \cap l'$. Тогда три точки - $\gamma = (ab')\cap(a'b)$, $\alpha=(bc')\cap (b'c)$ и $\beta = (ac')\cap (a'c)$ лежат на одной прямой.
\end{theorem}

\begin{defn}
    \textit{\textcolor{magenta}{\hypertarget{s48}{Треугольник}}} - тройка точек (\textit{\textcolor{magenta}{\hypertarget{s49}{вершин}}}), не лежащих на одной прямой. \textit{\textcolor{magenta}{\hypertarget{s50}{Стороны}}} треугольника - прямые, содержащие пары вершин.
\end{defn}

\begin{theorem}
    (\textit{\textcolor{magenta}{\hypertarget{s51}{Теорема Дезарга (аффинная)}}}). Пусть $\triangle abc$ и $\triangle a'b'c'$ - треугольники на аффинной плоскости, и их вершины и стороны все различны. Если прямые $(aa')$, $(bb')$ и $(cc')$ пересекаются в одной точке или параллельны, и $(ab)||(a'b')$, $(bc)||(b'c')$, то $(ac)||(a'c')$. 
\end{theorem} \ 

\begin{theorem}
    (\textit{\textcolor{magenta}{\hypertarget{s52}{Теорема Дезарга (проективная)}}}). Пусть $\triangle abc$ и $\triangle a'b'c'$ - треугольники на проективной плоскости, и их вершины и стороны все различны. Если прямые $(aa')$, $(bb')$ и $(cc')$ пересекаются в одной точке, то три точки $\gamma = (ab)\cap (a'b')$, $\alpha = (bc)\cap (b'c')$ и $\beta = (ac)\cap (a'c')$ лежат на одной прямой.
\end{theorem}

\section{Евклидовы пространства.}

\subsection{Начальные определения и свойства.} 

\begin{defn}
    \textit{\textcolor{magenta}{\hypertarget{s53}{Скалярное произведение}}} на векторном пространстве $X$ - функция 
    \[
        \langle \cdot, \cdot \rangle: X\times X \ra \RR, 
    \]
    удовлетворяющая условиям симметричности, линейности по каждому аргументу и неотрицательности $\langle x, x \rangle$ (равно нулю только при $x=0$). \ 

    \textit{\textcolor{magenta}{\hypertarget{s54}{Евклидово протранство}}} - векторное пространство с заданным на нём скалярным произведением.
\end{defn}

\begin{defn}
    \textit{\textcolor{magenta}{\hypertarget{s55}{Длина}}} (норма) вектора $x \in X$ - $|x|=\sqrt{\langle x, x \rangle}$, \textit{\textcolor{magenta}{\hypertarget{s56}{расстояние}}} между $x, y\in X$ - $d(x, y)=|x-y|$.
\end{defn}

\begin{theorem}
    (\textit{\textcolor{magenta}{\hypertarget{s57}{Неравенство КБШ}}}). Для любых $x, y \in X$, 
    \[
        |\langle x, y \rangle|\leq |x|\cdot|y|. 
    \]
    Причём неравенство обращается в равенство тогда и только тогда, когда $x$ и $y$ линейно зависимы.
\end{theorem}

\begin{cons}
    Для любых $x, y\in X$ $|x+y|\leq |x|+|y|$, причём равенство выполняется тогда и только тогда, когда один из векторов равен нулю или они сонаправленны.
\end{cons}

\begin{cons}
    Для любых $x, y, z\in X$, $d(x, z)\leq d(x, y)+d(y, z)$, причём равенство выполняется тогда и только тогда, когда векторы $x-y$ и $y-z$ сонаправлены или один из них равен нулю.
\end{cons}

\begin{defn}
    Пусть $X$ - евклидово пространство. \textit{\textcolor{magenta}{\hypertarget{s58}{Угол}}} между ненулевыми векторами $x$ и $y$ - это $\angle (x, y) = \arccos \frac{\langle x, y \rangle}{|x|\cdot |y|}$.
\end{defn}

\begin{theorem}
    (\textit{\textcolor{magenta}{\hypertarget{s59}{Теорема косинусов}}}). 
    \[
        |x-y|^2 = |x|^2+|y|^2-2|x|\cdot |y| \cos \angle (x, y).
    \]
\end{theorem} \ 

\begin{theorem}
    (\textit{\textcolor{magenta}{\hypertarget{s60}{Неравенство треугольника для углов}}}). Для любых ненулевых $x, y, z\in X$, 
    \[
        \angle(x, z)\leq \angle (x, y)+ \angle (y, z). 
    \]
\end{theorem}

\begin{cons}
    Для любых ненулевых $x, y, z\in X$, 
    \[
        \angle(x, y)+\angle(y, z)+\angle(z, x)\leq 2\pi. 
    \]
\end{cons}

\subsection{Ортогональность.}

\begin{defn}
    Векторы $x, y\in X$ \textit{\textcolor{magenta}{\hypertarget{s61}{ортогональны}}}, если $\langle x, y \rangle = 0$. Обозначается как $x\perp y$. 
\end{defn}

\begin{theorem}
    (\textit{\textcolor{magenta}{\hypertarget{s62}{Теорема Пифагора}}}). Если $x \perp y$, то $|x+y|^2=|x|^2+|y|^2$. 
\end{theorem}

\begin{cons}
    Если векторы $v_1, \ldots, v_n$ попарно ортогональны, то 
    \[
        |v_1+\ldots+v_n|^2=|v_1|^2+\ldots+|v_n|^2. 
    \]
\end{cons}

\begin{defn}
    \textit{\textcolor{magenta}{\hypertarget{s63}{Ортонормированный}}} набор векторов - такой, в котором каждые два вектора ортогональны и все имеют длину 1.
\end{defn}

\begin{theorem}
    Пусть $v_1, \ldots, v_n$ - ортонормированный набор, $x=\sum \alpha_i v_i$, $y=\sum \beta_i v_i$ $(\alpha_i, \beta_i \in \RR)$. Тогда $\langle x, y \rangle = \sum \alpha_i \beta_i$, а $|x|^2=\sum a_i^2$. 
\end{theorem} \

\begin{theorem}
    Любой ортонормированный набор линейно независим.
\end{theorem} \

\begin{theorem}
    (\textit{\textcolor{magenta}{\hypertarget{s64}{Об ортогонализации по Граму-Шмидту}}}). Для любого линейно-независимого набора векторов $v_1, \ldots, v_n$ существует единственный ортонормированный набор $e_1, \ldots, e_n$ такой, что для каждого $k\in \{1, \ldots, n\}$ $\Lin (e_1, \ldots, e_k)=\Lin(v_1, \ldots, v_k)$ и $\langle v_k, e_k \rangle > 0$. 
\end{theorem}

\begin{cons}
    Пусть $X$ - конечномерное евклидово пространство. Тогда в $X$ существует ортонормаированный базис, и любой ортонормаированный набор можно дополнить до ортономированного базиса.
\end{cons}

\subsection{Изоморфизмы.}

\begin{defn}
    Евклидовы пространства $X$ и $Y$ \textit{\textcolor{magenta}{\hypertarget{s65}{изоморфны}}}, если существует линейная биекция $f:X\ra Y$, сохраняющая скалярное произведение: 
    \[
        \langle f(v), f(w) \rangle_Y = \langle v, w \rangle_X
    \]
    для любых $v, w\in X$. Такое $f$ называется \textit{\textcolor{magenta}{\hypertarget{s66}{изоморфизмом}}} (евклидовых пространств).
\end{defn}

\begin{theorem}
    Пусть $X, Y$ - конечномерные евклидовы пространства одинаковой размерности. Тогда $X$ и $Y$ изоморфны.
\end{theorem}

\begin{cons}
    Любое евклидово пространство размерности $n$ изоморфно $\RR^n$. 
\end{cons}

\subsection{Продолжение ортогональности.}

\begin{defn}
    Пусть $X$ - евклидово пространство, $A$ - его подмножество. \textit{\textcolor{magenta}{\hypertarget{s67}{Ортогональное дополнение}}} множества $A$ это - 
    \[
        A^{\perp}=\{x\in X: \forall v\in A \: \langle x, v \rangle = 0\}. 
    \]
\end{defn}

\begin{stat}
    Ортогональное дополнение - линейное пространсто. Если $A\subset B$, то $B^{\perp} \subset A^{\perp}$. Наконец, $A^{\perp} = \Lin(A)^{\perp}$. 
\end{stat}

\begin{theorem}
    Пусть $X$ - конечномерное евклидово пространство, $V\subset X$ - линейное подпространство. Тогда $X= V \oplus V^{\perp}$, и $(V^{\perp})^{\perp}=V$.
\end{theorem}

\begin{defn}
    \textit{\textcolor{magenta}{\hypertarget{s68}{Ортогональная проекция}}} $x$ на $V$ ($\Prf_V(x)$) - такой вектор $y\in V$, что $x-y\in V^{\perp}$. 
\end{defn}

\begin{defn}
    \textit{\textcolor{magenta}{\hypertarget{s69}{Нормаль}}} линейной гиперплоскости $H$ - любой ненулевой вектор $v\in H^{\perp}$. 
\end{defn}

\begin{theorem}
    (\textit{\textcolor{magenta}{\hypertarget{s70}{Конечномерная теорема Рисса}}}). Пусть $X$ - конечномерное евклидово пространство, $L:X\ra \RR$ - линейное отображение. Тогда существует единственный вектор $v\in X$ такой, что $L(x)=\langle v, x \rangle$ для всех $x\in X$. 
\end{theorem} \

\begin{theorem}
    Любая линейная гиперплоскость имеет вид $\ker L$, где $L:X\ra \RR$ - линейное отображение, $L\neq 0$. Также, $L$ определена однозначно с точностью домножения на константу.
\end{theorem} \ 

\begin{theorem}
    (\textit{\textcolor{magenta}{\hypertarget{s71}{Расстояние до гиперплоскости}}}). Пусть $x=v^{\perp}$. Тогда расстояние от $x$ до $H$ равно 
    \[
        d(x, H)=\frac{|\langle v, x \rangle|}{|v|}, 
    \]
    или в координатах, где $a_1, \ldots, a_n$ - координаты $v$, 
    \[
        d(x, H)=\frac{|a_1x_1+\ldots+a_n x_n|}{\sqrt{a_1^2+\ldots+a_n^2}}.
    \]
\end{theorem}

\begin{defn}
    \textit{\textcolor{magenta}{\hypertarget{s72}{Изометрическое отображение}}} $X$ в $Y$ (евклидовы пространства) - линейное отображение, сохраняющее скалярное произведение. \textit{\textcolor{magenta}{\hypertarget{s73}{Ортогональное преобразование}}} пространства $X$ - изометрическое отображение из $X$ в себя.
\end{defn}

\begin{defn}
    \textit{\textcolor{magenta}{\hypertarget{s74}{Ортогональная группа}}} порядка $n$ - группа ортогональных преобразований $\RR^n$. Обозначается как $O(n)$. 
\end{defn}

\begin{stat}
    Линейное отображение изометрическое тогда и только тогда, когда оно сохраняет длины векторов. Или же, линейное отображение изометрическое тогда и только тогда, когда оно переводит какой-нибудь ортонормаированный базис в ортонормированный набор.
\end{stat}

\subsection{Немного о матрицах отображений.}

\begin{theorem}
    Пусть $f:X\ra Y$ линейно, $A$ - его матрица в ортономированных базисах $X$ и $Y$. Тогда $f$ изометрическое тогда и только тогда, когда $A^TA=E$. 
\end{theorem}

\begin{cons}
    При совпадении размерностей, это равносильно тому, что $AA^T=E$ или $A^T=A^{-1}$.
\end{cons}

\begin{defn}
    \textit{\textcolor{magenta}{\hypertarget{s75}{Ортогональная матрица}}} - квадратная матрица $A$, для которой $A^TA=AA^T=E$. 
\end{defn}

\begin{theorem}
    Если $A$ - ортонормаированная матрица, то $\det A = \pm 1$. 
\end{theorem}

\begin{defn}
    \textit{\textcolor{magenta}{\hypertarget{s76}{Специальная ортогональная группа}}} $SO(n)$ - группа ортогональных преобразований с определителем 1.
\end{defn}

\subsection{Много какой-то хуйни.}

\begin{defn}
    \textit{\textcolor{magenta}{\hypertarget{s77}{Инвариантное подпространство}}} линейного отображения $f: X\ra X$ - линейное подпространство $Y\subset X$ такое, что $f(Y)\subset Y$. 
\end{defn}

\begin{stat}
    Если $V$ - инварантное подпространство ортогонального преобразования, то $V^{\perp}$ - тоже инвариантное.
\end{stat}

\begin{theorem}
    Пусть $f:X\ra X$ - ортогональное преобразование. Тогда существует разложение $X$ в ортогональную прямую сумму
    \[
        X=X_+ \oplus X_- \oplus \Pi_1 \oplus \ldots \oplus \Pi_m \: \: (m\geq 0)
    \]
    инвариантных подпространств таких, что $f|_{X_+}=\id$, $f|_{X_-}=-\id$, а $\dim \Pi_i=2$, $f|_{\Pi_i}$ - поворот.
\end{theorem}

\begin{defn}
    Два базиса \textit{\textcolor{magenta}{\hypertarget{s78}{одинаково ориентированы}}}, если матрица перехода между ними имеет положительный определитель.
\end{defn}

\begin{remark}
    Одинаковая ориентированность базисов - отношение эквивалентности. Классов эквивалентности ровно два (кроме случая нулевой размерности).
\end{remark}

\begin{defn}
    \textit{\textcolor{magenta}{\hypertarget{s79}{Ориентированное векторное пространство}}} - векторное пространство, в котором выделен один их двух классов одинаково ориентированных базисов. Выделенные базисы - \textit{\textcolor{magenta}{\hypertarget{s80}{положительно ориентированные}}} (положительные), остальные - \textit{\textcolor{magenta}{\hypertarget{s81}{отрицательно ориентированные}}} (отрицательные).
\end{defn}

\begin{defn}
    Пусть $X$ - ориентированное евклидово пространство размерности $n$, $v_1, \ldots, v_n \in X$. \textit{\textcolor{magenta}{\hypertarget{s82}{Смешанное произведение}}} $v_1, \ldots, v_n$ - определитель матрицы из координат $v_i$ в произвольном положительном ортонормированном базисе. Обозначается как $[v_1, \ldots, v_n]$. 
\end{defn}

\begin{theorem}
    Определение корректно, то есть, не зависит от выбора базиса.
\end{theorem}

\begin{defn}
    Пусть $X$ - трёхмерное ориентированное евклидово пространство, $u, v\in X$. Их \textit{\textcolor{magenta}{\hypertarget{s83}{векторное произведение}}} - такой (единственный по лемме Рисса) вектор $h\in X$, что $\langle h, x\rangle = [u, v, x]$ для любого $x\in X$. Обозначается как $h=u\times v$. 
\end{defn}

\begin{theorem}
    Пусть $u, v$ линейно независимы. Тогда 
    
    \begin{itemize}
        \item $u\times v$ - вектор, ортогональный $u$ и $v$; 
        \item $u, v, u\times v$ - положительный базис; 
        \item $|u\times v|$ равно площади параллелограмма, образованного векторами $u$ и $v$. 
    \end{itemize}
\end{theorem} \ 

\begin{theorem}
    Пусть $e_1, e_2, e_3$ - положительный ортонормированный базис, $x=x_1e_1+x_2e_2+x_3e_3$, $y=y_1e_1+y_2e_2+y_3e_3$. Тогда 
    \[
        x\times y = (x_2y_3-x_3y_2)e_1+(x_3y_1-x_1y_3)e_2+(x_1y_2-x2y_1)e_3, 
    \]
    или в псевдо-матричной записи:
    \begin{equation*}
        x\times y = 
        \begin{vmatrix}
            x_1 & x_2 & x_3 \\
            y_1 & y_2 & y_3 \\
            e_1 & e_2 & e_3 
        \end{vmatrix}
    \end{equation*}
\end{theorem}

\subsection{Движения евклидова аффинного пространства.}

\begin{defn}
    \textit{\textcolor{magenta}{\hypertarget{s84}{Евклидово аффинное пространство}}} - аффинное пространство $X$ с заданным на $\vec{X}$ скалярным произведением. \textit{\textcolor{magenta}{\hypertarget{s85}{Расстояние}}} в таком пространстве: $d(x, y)=|x-y|$. 
\end{defn}

\begin{defn}
    \textit{\textcolor{magenta}{\hypertarget{s86}{Движение}}} евклидова аффинного пространства $X$ - биекция из $X$ в $X$, сохраняющая расстояния. Группа движений обозначается как $\Iso(X)$.
\end{defn}

\begin{theorem}
    Любое движение - аффинное преобразование, линейная часть которого - ортогональное преобразование, и обратно.
\end{theorem} \

\begin{lemma}
    Пусть $X$ - аффинное пространство, $F:X\ra X$ - аффинное отображение, и его линейная часть не имеет неподвижных ненулевых векторов, то есть, $\vec{F}(v)\neq v$ для всех $v\in \vec{X}\backslash \{0\}$. Тогда $F$ имеет неподвижную точку. 
\end{lemma}

\begin{cons}
    Если линейная часть движения плоскости - поворот на ненулевой угол, то и само движение - поворот на этот угол относительно некоторой точки.
\end{cons}

\begin{cons}
    Композиция поворотов - поворот или параллельный перенос.
\end{cons}

\newpage

\section{Указатель.}

\hypertarget{t2}{Он самый.}

\begin{multicols}{2}

    \hyperlink{s24}{аффинный базис} \ 

    \hyperlink{s22}{аффинная зависимость} \ 
    
    \hyperlink{s23}{аффинная независимость} \ 
    
    \hyperlink{s21}{аффинная оболочка} \ 
    
    \hyperlink{s26}{аффинное отображение(1)} \ 
    
    \hyperlink{s28}{аффинное отображение(2)} \ 
    
    \hyperlink{s13}{аффинное подпространство} \ 
    
    \hyperlink{s1}{аффинное пространство} \
    
    \hyperlink{s8}{барицентрическая лк} \
    
    \hyperlink{s25}{барицентрические координаты} \ 
    
    \hyperlink{s38}{бесконечно удалённые точки} \ 

    \hyperlink{s39}{бесконечно удалённая гп} \
    
    \hyperlink{s6}{векторизация (ап)} \ 

    \hyperlink{s83}{векторное произведение} \ 

    \hyperlink{s49}{вершина} \ 
    
    \hyperlink{s19}{гиперплоскость} \ 
    
    \hyperlink{s29}{гомотетия} \ 

    \hyperlink{s86}{движение} \ 

    \hyperlink{s41}{двойное отношение} \

    \hyperlink{s54}{евклидово пространство} \ 

    \hyperlink{s84}{евклидово ап} \ 

    \hyperlink{s72}{изометрическое отображение} \ 

    \hyperlink{s65}{изоморфность (еп)} \ 

    \hyperlink{s66}{изоморфизм (еп)} \ 

    \hyperlink{s77}{инвариантное подпространство} \ 

    \hyperlink{s40}{карта} \
    
    \hyperlink{s30}{коэффициент растяжения} \ 
    
    \hyperlink{s7}{линейная комбинация} \ 
    
    \hyperlink{s27}{линейная часть (ао)} \ 
    
    \hyperlink{s11}{масса} \ 
    
    \hyperlink{s10}{материальная точка} \ 
    
    \hyperlink{s14}{направление афинного п/п} \ 
    
    \hyperlink{s5}{начало отсчёта (ап)} \ 

    \hyperlink{s57}{неравенство КБШ} \ 

    \hyperlink{s60}{неравенство тр-ка для углов} \ 

    \hyperlink{s55}{норма} \ 

    \hyperlink{s69}{нормаль} \ 

    \hyperlink{s78}{одинаковая ориентированность} \ 
    
    \hyperlink{s36}{однородные координаты} \ 

    \hyperlink{s79}{ориентированное вп} \ 

    \hyperlink{s74}{ортогональная группа} \ 

    \hyperlink{s75}{ортогональная матрица} \ 

    \hyperlink{s68}{ортогональная проекция} \ 

    \hyperlink{s67}{ортогональное дополнение} \ 

    \hyperlink{s73}{ортогональное прообразование} \ 

    \hyperlink{s61}{ортогональные векторы} \ 

    \hyperlink{s63}{ортонормированный набор векторов} \ 
    
    \hyperlink{s31}{основная теорема аг} \ 
    
    \hyperlink{s4}{откладывание вектора} \ 

    \hyperlink{s81}{отр. ориент. базис} \ 
    
    \hyperlink{s17}{параллельные (ап/п)} \ 
    
    \hyperlink{s16}{параллельный перенос} \ 

    \hyperlink{s80}{положит. ориент. базис} \ 
    
    \hyperlink{s3}{присоединённое (вп)} \ 
    
    \hyperlink{s33}{проективизация(1)} \
    
    \hyperlink{s42}{проективизация(2)} \ 

    \hyperlink{s43}{проективное отображение} \ 
    
    \hyperlink{s35}{проективное подпространство} \ 
    
    \hyperlink{s37}{проективное пополнение} \ 
    
    \hyperlink{s32}{проективное пространство} \ 

    \hyperlink{s45}{проективный базис} \ 
    
    \hyperlink{s18}{прямая} \ 
    
    \hyperlink{s15}{размерность (ап)} \ 
    
    \hyperlink{s34}{размерность (пп)} \ 

    \hyperlink{s56}{расстояние (еп)} \ 

    \hyperlink{s85}{расстояние (еап)} \ 

    \hyperlink{s71}{расстояние до гиперплоскости} \ 
    
    \hyperlink{s9}{сбалансированная лк} \ 

    \hyperlink{s82}{смешанное произведение} \ 

    \hyperlink{s53}{скалярное произведение} \ 

    \hyperlink{s76}{спец. ортогональная группа} \ 

    \hyperlink{s50}{сторона} \ 
    
    \hyperlink{s20}{сумма (ап/п)} \ 

    \hyperlink{s51}{теорема Дезарга (а)} \
    
    \hyperlink{s52}{теорема Дезарга (п)} \ 

    \hyperlink{s59}{теорема косинусов} \ 

    \hyperlink{s64}{теорема об ортогонализации} \ 

    \hyperlink{s46}{теорема Паппа (а)} \
    
    \hyperlink{s47}{теорема Паппа (п)} \ 

    \hyperlink{s62}{теорема Пифагора} \ 

    \hyperlink{s70}{теорема Рисса} \ 
    
    \hyperlink{s2}{точка} \ 

    \hyperlink{s48}{треугольник} \ 

    \hyperlink{s58}{угол} \ 
    
    \hyperlink{s12}{центр масс} \ 

    \hyperlink{s44}{центральная проекция} \ 

\end{multicols}



\end{document}