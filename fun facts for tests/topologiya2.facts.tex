\documentclass[a4paper,100pt]{article}

\usepackage[utf8]{inputenc}
\usepackage[unicode, pdftex]{hyperref}
\usepackage{cmap}
\usepackage{mathtext}
\usepackage{multicol}
\setlength{\columnsep}{1cm}
\usepackage[T2A]{fontenc}
\usepackage[english,russian]{babel}
\usepackage{amsmath,amsfonts,amssymb,amsthm,mathtools}
\usepackage{icomma}
\usepackage{euscript}
\usepackage{mathrsfs}
\usepackage{geometry}
\usepackage[usenames]{color}
\hypersetup{
     colorlinks=true,
     linkcolor=magenta,
     filecolor=green,
     citecolor=black,      
     urlcolor=cyan,
     }
\usepackage{fancyhdr}
\pagestyle{fancy} 
\fancyhead{} 
\fancyhead[CO]{\hyperlink{t2}{к списку объектов}}
\fancyhead[LO]{\hyperlink{t1}{к содержанию}} 
\fancyfoot{}
\newtheoremstyle{indented}{0 pt}{0 pt}{\itshape}{}{\bfseries}{. }{0 em}{ }

%\geometry{verbose,a4paper,tmargin=2cm,bmargin=2cm,lmargin=2.5cm,rmargin=1.5cm}

\title{Геометрия и топология. Факты 2 сем.}
\author{Кабашный Иван (@keba4ok)\\ \\ (по материалам лекций Фоминых Е. А.,\\ практик, а также других источников)}
\date{26 марта 2021 г.}

\theoremstyle{indented}
\newtheorem{theorem}{Теорема}
\newtheorem{lemma}{Лемма}

\theoremstyle{definition} 
\newtheorem{defn}{Определение}
\newtheorem{exl}{Пример(ы)}

\theoremstyle{remark} 
\newtheorem{remark}{Примечание}
\newtheorem{cons}{Следствие}
\newtheorem{stat}{Утверждение}

\DeclareMathOperator{\la}{\leftarrow}
\DeclareMathOperator{\ra}{\rightarrow}
\DeclareMathOperator{\lra}{\leftrightarrow}
\DeclareMathOperator{\La}{\Leftarrow}
\DeclareMathOperator{\Ra}{\Rightarrow}
\DeclareMathOperator{\Lra}{\Leftrightarrow}
\DeclareMathOperator{\Llra}{\Longleftrightarrow}
\DeclareMathOperator{\Ker}{Ker}
\DeclareMathOperator{\Tors}{Tors}
\DeclareMathOperator{\Frac}{Frac}
\DeclareMathOperator{\Imf}{Im}
\DeclareMathOperator{\Real}{Re}
\DeclareMathOperator{\cont}{cont}
\DeclareMathOperator{\id}{id}
\DeclareMathOperator{\ev}{ev}
\DeclareMathOperator{\lcm}{lcm}
\DeclareMathOperator{\chard}{char}
\DeclareMathOperator{\CC}{\mathbb{C}}
\DeclareMathOperator{\ZZ}{\mathbb{Z}}
\DeclareMathOperator{\RR}{\mathbb{R}}
\DeclareMathOperator{\NN}{\mathbb{N}}
\DeclareMathOperator{\PP}{\mathbb{P}}
\DeclareMathOperator{\FF}{\mathcal{F}}
\DeclareMathOperator{\Rho}{\mathcal{P}}
\DeclareMathOperator{\codim}{codim}
\DeclareMathOperator{\rank}{rank}
\DeclareMathOperator{\ord}{ord}
\DeclareMathOperator{\adj}{adj}
\DeclareMathOperator{\const}{const}
\DeclareMathOperator{\grad}{grad}
\DeclareMathOperator{\Aff}{Aff}

\begin{document}

\newcommand{\resetexlcounters}{%
  \setcounter{exl}{0}%
} 

\newcommand{\resetremarkcounters}{%
  \setcounter{remark}{0}%
} 

\newcommand{\reseconscounters}{%
  \setcounter{cons}{0}%
} 

\newcommand{\resetall}{%
    \resetexlcounters
    \resetremarkcounters
    \reseconscounters%
}

\maketitle 

\newpage

\hypertarget{t1}{Основные} (по моему мнению) факты по топологии.
\tableofcontents

\newpage


\section{Аффинные пространства.}

\subsection{Начальные определения и свойства.}

\begin{defn}
    \textit{\textcolor{magenta}{\hypertarget{s1}{Аффинное пространство}}} - тройка $(X, \vec{X}, +)$, состоящая из непустого множества \textit{\textcolor{magenta}{\hypertarget{s2}{точек}}}, векторного пространства над $\RR$ (\textit{\textcolor{magenta}{\hypertarget{s3}{присоединённое}}}) и операцией $+:X\times \vec{X}\ra X$ \textit{\textcolor{magenta}{\hypertarget{s4}{откладывания вектора}}}. \ 

    Налагаемые условия - для любых точек $x, y\in X$ существует единственный вектор $v\in \vec{X}$ такой, что $x+v=y$ ($\vec{xy}$), а также ассоциативность откладывания вектора.
\end{defn}

\begin{defn}
    \textit{\textcolor{magenta}{\hypertarget{s5}{Начало отсчёта}}} аффинного пространства - произвольная фиксированная точка $o\in X$. 
\end{defn}

\begin{lemma}
    Начало отсчёта $o\in X$ задаёт биекцию $\varphi_o: X\ra \vec{X}$ по правилу:
    \[
        \varphi_o(x)=\vec{ox} \: \forall x \in X. 
    \]
    Такая биекция называется \textit{\textcolor{magenta}{\hypertarget{s6}{векторизацией}}} аффинного пространства.
\end{lemma}

\begin{defn}
    \textit{\textcolor{magenta}{\hypertarget{s7}{Линейная комбинация}}} $\sum t_i p_i$ точек с коэффициентами относительно начала отсчёта $o\in X$ - вектор $v = \sum t_i \vec{op_i}$, или точка $p=o+v$. Комбинация называется \textit{\textcolor{magenta}{\hypertarget{s8}{барицентрической}}}, если сумма коэффициентов равна единице, и \textit{\textcolor{magenta}{\hypertarget{s9}{сбалансированной}}}, если сумма коэффициентов равна нулю.
\end{defn}

\begin{theorem}
    Барицентрическая комбинация точек - точка, не зависящая от начала отсчёта. Сбалансированная комбинация точек - вектор, не зависящий от начала отсчёта.
\end{theorem}

\subsection{Материальные точки.}

\begin{defn}
    Пусть $x$ — некоторая точка аффинного пространства и $m$ — ненулевое число. \textit{\textcolor{magenta}{\hypertarget{s10}{Материальной точкой}}} $(x,m)$ называется пара: точка $x$ с вещественным числом $m$, причем число m называется \textit{\textcolor{magenta}{\hypertarget{s11}{массой}}} материальной точки $(x,m)$, а точка $x$ — носителем этой материальной точки.
\end{defn}

\begin{defn}
    \textit{\textcolor{magenta}{\hypertarget{s12}{Центром масс}}} системы материальных точек $(x_i, m_i)$ называется такая точка $z$ (притом единственная), для которой имеет место равенство
    \[
        m_1 \cdot \vec{zx_1} + \ldots + m_n \cdot \vec{zx_n} = 0. 
    \]
\end{defn}

\subsection{Аффинные подпространства и оболочки.}

\begin{defn}
    Множество $Y\subset X$ - \textit{\textcolor{magenta}{\hypertarget{s13}{аффинное подпространство}}}, если существуют такие линейное подпространство $V\subset \vec{X}$ и точка $p\in Y$, что $Y=p+V$. $V$ называется\textit{\textcolor{magenta}{\hypertarget{s14}{направлением}}} $Y$. Определение подпространства не зависит от выбора точки в нём.
\end{defn}

\begin{defn}
    \textit{\textcolor{magenta}{\hypertarget{s15}{Размерность}}} $\dim X$ афинного пространства есть размерность его присоединённого векторного пространства.
\end{defn}

\begin{defn}
    \textit{\textcolor{magenta}{\hypertarget{s16}{Параллельный перенос}}} на вектор $v\in \vec{X}$ - отображение $T_v:X\ra X$, заданное равенством $T_v(x)=x+v$. 
\end{defn}

\begin{defn}
    Аффинные подпространства одинаковой размерности \textit{\textcolor{magenta}{\hypertarget{s17}{параллельны}}}, если их направления совпадают.
\end{defn}

\begin{defn}
    \textit{\textcolor{magenta}{\hypertarget{s18}{Прямая}}} - аффинное подпространство размерности 1, \textit{\textcolor{magenta}{\hypertarget{s19}{гиперплоскость}}} в $X$ - аффинное подпространство размерности $\sim X - 1$. 
\end{defn}

\begin{stat}
    Две различные гиперплоскости не пересекаются тогда и только тогда, когда они параллельны.
\end{stat}

\begin{defn}
    \textit{\textcolor{magenta}{\hypertarget{s20}{Суммой аффинных подпространств}}} называется наименьшее аффинное подпространство, их содержащее.
\end{defn}

\begin{theorem}
    Пересечение любого набора аффинных подпространств - либо пустое мноежство, либо аффинное подпространство.
\end{theorem}

\begin{defn}
    \textit{\textcolor{magenta}{\hypertarget{s21}{Аффинная оболочка}}} $\Aff{A}$ непустого множества $A\subset X$ - пересечение всех аффинных подпространств, содержащих $A$. Как следствие, это - наименьшее аффинное подпространство, содержащее $A$. 
\end{defn}

\begin{theorem}
    $\Aff(A)$ - множество всех барицентрических комбинаций точек из $A$. 
\end{theorem}

\begin{defn}
    Точки $p_1, \ldots, p_k$ \textit{\textcolor{magenta}{\hypertarget{s22}{аффинно зависимы}}}, если существуют такие коэффициенты $t_i \in \RR$, не все равные нулю, что $\sum t_i = 0$ и $\sum t_ip_i = 0$. Если такой комбинации нет, то точки \textit{\textcolor{magenta}{\hypertarget{s23}{аффинно независимы}}}.
\end{defn}

\begin{theorem}
    (Переформулировки аффинной независимости.) Для $p_1, \ldots, p_k \in X$ следующие свойства эквивалентны: 

    \begin{itemize}
        \item они аффинно независимы; 
        \item векторы $p_1 p_i$, $i\in \{2, 3, \ldots, k\}$, линейно независимы; 
        \item $\dim \Aff (p_1, \ldots, p_k)=k-1$; 
        \item каждая точка из $\Aff(p_1, \ldots, p_k)$ единственным образом представляется в виде барицентрической комбинации $p_i$. 
    \end{itemize}
\end{theorem}

\subsection{Базисы и отображения.}

\begin{defn}
    \textit{\textcolor{magenta}{\hypertarget{s24}{Аффинный базис}}} - набор $n+1$ точке в $X$, пространстве размерности $n$, являющийся аффинно независимым. Или же, это - точке $o\in X$ и базис $e_0, \ldots, e_n$ пространства $\vec{X}$. 
\end{defn}

\begin{defn}
    Каждая точка однозначно записывается в виде барицентрической комбинации $\sum_{i=0}^n t_i e_i$, а числа $t_i$ называют \textit{\textcolor{magenta}{\hypertarget{s25}{барицентрическими координатами}}} этой точки.
\end{defn}

\begin{defn}
    (Говно-определение). Отображение $F:X\ra Y$ называется \textit{\textcolor{magenta}{\hypertarget{s26}{аффинным}}}, если отображение $\tilde{F}_p$ линейно для некоторой точки $p\in X$. Отображение $\tilde{F}_p:\vec{X}\ra\vec{Y}$ индуцируется из любого отображения $F:X\ra Y$ посредством формулы $\forall v\in \vec{X}$ $\tilde{F}_p(v)=\overrightarrow{F(p)F(q)}$, где $q=p+v$. 
\end{defn}

\begin{defn}
    Отображение $\tilde{F}$ называется \textit{\textcolor{magenta}{\hypertarget{s27}{линейной частью}}} аффинного отображения $F$.
\end{defn}

\begin{defn}
    (Нормальное определение.) Отображение $F:X\ra Y$ называется \textit{\textcolor{magenta}{\hypertarget{s28}{аффинным}}}, если существует такое линейное $L:\vec{X}\ra\vec{Y}$, что для любых $q, p\in X$, $\overrightarrow{F(p)F(q)}=L(\vec{pq})$. 
\end{defn}

\begin{theorem}
    Пусть $x\in X$, $y\in Y$, $L:\vec{X}\ra \vec{Y}$ линейно. Тогда существует единственное аффинное отображение $F:X\ra Y$ такое, что $\tilde{F}=L$ и $F(x)=y$. 
\end{theorem} \

\begin{lemma}
    Пусть $p_1, \ldots, p_n$ - аффинно независимые точки в аффинном пространстве $X$, $q_1, \ldots, q_n$ - точки в аффинном пространстве $Y$. Тогда существует такое аффинное отображение $F:X\ra Y$, что $F(p_i)=q_i$ $\forall i$. Кроме того, если $\dim X = n-1$, то такое отображение единственно.
\end{lemma} \ 

\begin{lemma}
    Аффинное отображение сохраняет барицентрические комбинации.
\end{lemma} \ 

\begin{lemma}
    Композиция аффинных отображений - аффинное отображение. При этом линейная часть композиции - композиция линейных частей.
\end{lemma}

\begin{stat}
    Образ и прообраз аффинного подпространства - аффинное подпространство. Образы (прообразы) параллельных подпространств параллельны.
\end{stat}

\begin{theorem}
    Параллельный перенос - аффинное отображение, его линейная часть тождественна. Верно также и обратное.
\end{theorem}

\begin{defn}
    Аффинное отображение $F:X\ra X$ такое, что $\tilde{F}=k \id$ для некоторого $k \in \RR\backslash \{0, 1\}$, называется \textit{\textcolor{magenta}{\hypertarget{s29}{гомотетией}}}, а $k$ называют \textit{\textcolor{magenta}{\hypertarget{s30}{коэффициентом растяжения}}} гомотетии $F$. Такое отображение имеет ровно одну неподвижную точку, называемую центром.
\end{defn}

\begin{theorem}
    (\textit{\textcolor{magenta}{\hypertarget{s31}{Основная теорема аффинной геометрии.}}}) Пусть $X, Y$ - аффинные пространства, $\dim X \geq 2$. Пусть $F: X\ra Y$ - инъективное отображение, и для любой прямой $l\subset X$ её образ $F(l)$ - тоже прямая. Тогда $F$ - аффинное отображение.
\end{theorem}

\section{Проективные пространства.}

\subsection{Начальные определения и свойства.}

\begin{defn}
    Пусть $V$ - векторное пространство над полем $K$. На множестве $V \backslash \{0\}$ введём отношение эквивалентности
    \[
        x \sim y \Llra \exists \lambda \in K: x = \lambda y. 
    \]
    Тогда фактор $V$ по этому отношению называют \textit{\textcolor{magenta}{\hypertarget{s32}{проективным пространством}}} ($\PP(V)$), порождённым векторным $V$. Само отображение из векторного пространства в соответствующее проективное называют \textit{\textcolor{magenta}{\hypertarget{s33}{проективизацией}}}.
\end{defn}

\begin{remark}
    \textit{\textcolor{magenta}{\hypertarget{s34}{Размерность}}} $\PP(V)$ по определению равна $\dim V -1$. 
\end{remark}

\begin{theorem}
    Пусть $Y, Z\subset X$ - подпространства, $\dim Y + \dim Z \geq \dim X$, тогда

    \begin{itemize}
        \item $Y\cap Z \neq \emptyset$; 
        \item $Y \cap Z$ - подпространство; 
        \item $\dim(Y \cap Z)\geq \dim Y + \dim Z - \dim X$. 
    \end{itemize}
\end{theorem}

\begin{defn}
    Пусть $W$ - непустое векторное подпространство $V$. Тогда $\PP(W)$ называется \textit{\textcolor{magenta}{\hypertarget{s35}{проективным подпространством}}} $\PP(V)$.
\end{defn}

\begin{defn}
    Пусть $X=\PP(V)$ - проективное пространство размерности $n$. Числа $x_0, x_1, \ldots, x_n$, являющиеся координатами вектора $v$, порождающего $p\in \PP(V)$, называются \textit{\textcolor{magenta}{\hypertarget{s36}{однородными координатами}}}.
\end{defn}

\begin{defn}
    $\hat{X}=\PP(V)$ - \textit{\textcolor{magenta}{\hypertarget{s37}{проективное пополнение}}} аффинного пространства $X$, а множество $X_\infty = \PP(\vec{X}\times 0)\subset \hat{X}$ - \textit{\textcolor{magenta}{\hypertarget{s38}{бесконечно удалённые точки}}}. Также, множество этих точек есть гиперплоскость в $\hat{X}$, которая называется \textit{\textcolor{magenta}{\hypertarget{s39}{бесконечно удалённой гиперплоскостью}}}.
\end{defn}

\begin{defn}
    Пусть $V$ - векторное пространство, $W\subset V$ - линейная гиперплоскость, $X$ - гиперплоскость ей пареллельная. Тогда биекцию $\PP(V) \backslash \PP(W) \ra X$ называют \textit{\textcolor{magenta}{\hypertarget{s40}{картой}}} пространства $\PP(V)$.
\end{defn}

\begin{defn}
    Пусть на $\RR p^1$ (прямая с бесконечно удалённой точкой) выбрана аффинная система координат, в которой $A=a$, $B=b$, $C=c$ и $D=d$. Определим \textit{\textcolor{magenta}{\hypertarget{s41}{двойное отношение}}} четвёрки точек $(A, B, C, D)$ формулой
    \[
        [A, B, C, D]=\frac{a-c}{a-d}\cdot \frac{b-c}{b-d}.
    \]
\end{defn}

\begin{stat}
    Данное определение инвариантно относительно выбора карты, а само отношение сохраняется при проективных преобразованиях.
\end{stat}

\newpage

\section{Указатель.}

\hypertarget{t2}{Он самый.}

\begin{multicols}{2}

    \hyperlink{s24}{аффинный базис} \ 

    \hyperlink{s22}{аффинная зависимость} \ 
    
    \hyperlink{s23}{аффинная независимость} \ 
    
    \hyperlink{s21}{аффинная оболочка} \ 
    
    \hyperlink{s26}{аффинное отображение(1)} \ 
    
    \hyperlink{s28}{аффинное отображение(2)} \ 
    
    \hyperlink{s13}{аффинное подпространство} \ 
    
    \hyperlink{s1}{аффинное пространство} \
    
    \hyperlink{s8}{барицентрическая лк} \
    
    \hyperlink{s25}{барицентрические координаты} \ 
    
    \hyperlink{s38}{бесконечно удалённые точки} \ 

    \hyperlink{s39}{бесконечно удалённая гп} \
    
    \hyperlink{s6}{векторизация (ап)} \ 
    
    \hyperlink{s19}{гиперплоскость} \ 
    
    \hyperlink{s29}{гомотетия} \ 

    \hyperlink{s41}{двойное отношение} \

    \hyperlink{s40}{карта} \
    
    \hyperlink{s30}{коэффициент растяжения} \ 
    
    \hyperlink{s7}{линейная комбинация} \ 
    
    \hyperlink{s27}{линейная часть (ао)} \ 
    
    \hyperlink{s11}{масса} \ 
    
    \hyperlink{s10}{материальная точка} \ 
    
    \hyperlink{s14}{направление афинного п/п} \ 
    
    \hyperlink{s5}{начало отсчёта (ап)} \ 
    
    \hyperlink{s36}{однородные координаты} \ 
    
    \hyperlink{s31}{основная теорема аг} \ 
    
    \hyperlink{s4}{откладывание вектора} \ 
    
    \hyperlink{s17}{параллельные (ап/п)} \ 
    
    \hyperlink{s16}{параллельный перенос} \ 
    
    \hyperlink{s3}{присоединённое (вп)} \ 
    
    \hyperlink{s33}{проективизация} \ 
    
    \hyperlink{s35}{проективное подпространство} \ 
    
    \hyperlink{s37}{проективное пополнение} \ 
    
    \hyperlink{s32}{проективное пространство} \ 
    
    \hyperlink{s18}{прямая} \ 
    
    \hyperlink{s15}{размерность (ап)} \ 
    
    \hyperlink{s34}{размерность (пп)} \ 
    
    \hyperlink{s9}{сбалансированная лк} \ 
    
    \hyperlink{s20}{сумма (ап/п)} \ 
    
    \hyperlink{s2}{точка} \ 
    
    \hyperlink{s12}{центр масс} \ 

\end{multicols}



\end{document}